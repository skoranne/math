%%%%%%%%%%%%%%%%%%%%%%%%%%%%%%%%%%%%%%%%%%%%%%%%%%%%%%%%%%%%
%% HW3 for Measure Theory
\documentclass{article}
\usepackage{a4wide}
\usepackage{amsmath}
\usepackage{amssymb}
\usepackage{amsmath}
\usepackage{amsfonts}
\usepackage{amsthm}
\usepackage{polynom}

\newtheorem{lem}{Lemma}
\def\RR{\mathbb R}
\def\CC{\mathbb C}
\def\QQ{\mathbb Q}
\def\FF{\mathbb F}
\def\ZZ{\mathbb Z}

\begin{document}
\title{HW3 for MTH513 Real Analysis}
\author{ Sandeep Koranne}
\maketitle

\section*{Problem 7.9} Suppose $f:\RR\to\RR$ is integrable, $a\in \RR$ and
we define
\[
F(x) = \int_a^x f(y) dy
\]
Show that $F$ is a continuous function.
\begin{proof}
Using the fundamental theorem of calculus, we have
\[
F(x)-F(a) = \int_a^x f(y) dy
\]
Since $f$ is integrable, the RHS is finite, moreover we can write $|f(x)|\le M$
over the real line (otherwise the function would not be integrable). Thus we have
\[
-\int_a^x M dy \le \int_a^x f(y) dy \le \int_a^x M dy
\]
Integrating the constant $M$, we get
\[
-M(x-a) \le \int_a^x f(y) dy \le M(x-a)
\]
Substituting from the fundamental theorem, we get
\[
-M(x-a) \le F(x)-F(a) \le M(x-a) \implies |F(x)-F(a)|\le M|(x-a)|
\]
This implies that $\lim_{x\to a} F(x)=F(a)$, which implies continuity of $F$.
\end{proof}

\section*{Problem 7.11}Find the limit
\[
\lim_{n\to\infty} \int_0^n \left( 1 + \frac{x}{n}\right)^{-n} \log(2+\cos(x/n)) dx
\]
\begin{proof}
Consider $f_n(x) = \left( 1 + \frac{x}{n}\right)^{-n} \log(2+\cos(x/n))$.
As $n\to\infty$, clearly $f_n(x)\to \log(2)/e^x$, therefore let us write
$g(x)=\log(2)/e^x$. 
Then we observe that
\[
|f_n(x)| \le g(x)=\frac{\log(2)}{e^x}
\]
We can verify that $g(x)$ is integrable, and thus we
can use the dominated convergence theorem, thus
\[
\lim_{n\to\infty} \int_0^n \left( 1 + \frac{x}{n}\right)^{-n} \log(2+\cos(x/n)) dx
 = \int_0^\infty e^{-x}\log(2) dx = \log(2)
\]
\end{proof}

\section*{Problem 7.13}Prove that the limit exists and find its value
\[
\lim_{n\to\infty} \int_0^1 \frac{1+nx^2}{(1+x^2)^n} \log(2+cos(x/n)) dx 
\]
\begin{proof}
Using Bernoulli's inequality we have $(1+x^2)^n > 1+nx^2$, therefore if
we use $f_n$ to denote the function inside the integral, we observe that
$\lim_{n\to\infty} f_n=0$. Therefore we can define $g(x)=\log(3)$, and 
use the dominated convergence theorem to conclude that the limit exists, and
is 0.
\end{proof}

\section*{Problem 7.15}Let $g:\RR\to\RR$ be integrable and let $f:\RR\to\RR$
be bounded, measurable and continuous at 1. Prove that
\[
\lim_{n\to\infty} \int_{-n}^n f\left( 1 + \frac{x}{n^2}\right) g(x) dx
\]
exists and determine its value.
\begin{proof}
First we observe that $f_n=f\left( 1 + \frac{x}{n^2}\right)$, then
as $n\to\infty$, $f_n\to f(1)$, therefore we can use the dominated convergence
theorem with $f(1)$ playing the role of the usual $g(x)$ for domination.
Since $f(1)$ is constant, it is certainly integrable, and thus we 
can interchange the order of the limits for $f_n$ also, and we get
\[
\lim_{n\to\infty} \int_{-n}^n f\left( 1 + \frac{x}{n^2}\right) g(x) dx =
\int_{-\infty}^\infty f(1)g(x) dx =
f(1)\int_{-\infty}^\infty g(x) dx
\]
Now since $g(x)$ is known to be integrable, the above limit exists, and
since nothing else about $g(x)$ is known, we conclude that this is the limit,
and is finite.
\end{proof}

\section*{Problem 7.25}Let $(X,A,\mu)$ be a measure space and let $f$ be 
non-negative and integrable. Define $\nu$ on $A$ by
\[
\nu(A) = \int_A f d\mu
\]
\subsection*{Problem 7.22(a)} Prove that $\nu$ is a measure.
\begin{proof}
If $A=\emptyset$, then by definition of integral (cf., integral over
a set is the integral multiplied by the characteristic function of the set),
is 0, and thus $\nu(\emptyset)=0$. For the countable additivity, we instead
note that by problem 7.9 above, we have shown that $\nu(A)$ is indeed
continuous function, and continuous functions are subadditive (as discussed
in class this hint to prove measure criteria using continuity).

\end{proof}
\subsection*{Problem 7.22(b)} Prove that if $g$ is integrable with respect
to $\nu$, then $fg$ is integrable with respect to $\mu$ and 
\[
\int g\ d\nu = \int fg\ d\mu
\]
\begin{proof}
We write $\int g d\nu$ using the definition of $\nu(A)$ as
\[
\int g\ d\nu = \int_{A} g\ d\nu = \int_{A} g f\ d\mu
\]
Now since $g$ is known to be integrable w.r.t $\nu$, we can 
exchange the order to get
\[
\int g d\nu = \int_A fg\ d\mu
\]
As required.
\end{proof}

\section*{Problem 8.5}Suppose that $f$ is non-negative integrable
function on a measure space $(X,A,\mu)$. Prove that
\[
\lim_{t\to\infty} t \mu(\{x : f(x) \ge t\})=0
\]
\begin{proof}
Since $f$ is integrable, that implies that it can be written as
\[
\int_A f d\mu < \infty = \int_{f(x)< t} f d\mu + \int_{f(x)\ge t} f d\mu
\]
This implies that
\[
\int_{f(x)< t} f d\mu + \int_{f(x)\ge t} f d\mu < \infty
\]
In particular, then since $|f(x)|\le M$ for some $M$, we have
\[
\int_{f(x)\ge t} f\ d\mu = \int_{f(x)\ge t} f \chi_{(f(x)\ge t)}\ d \mu
\]
Combining the above we get
\[
\int_{f(x)\ge t} f\ d\mu \le Mt \implies t \mu(\chi_{(f(x)\ge t)}) \le M
\]
Taking limit $t\to\infty$ gives the result. Here we essentially
use the fact that if $f(x)$ is bounded then $\mu(\chi(X))\to 0$
as $t\to\infty$.
\end{proof}


\section*{Problem 11.9} Prove that
\[
\int_0^1 \int_0^1 \frac{x^2-y^2}{(x^2+y^2)^{3/4}}\log(4 + \sin x)\ dy dx =
\int_0^1 \int_0^1 \frac{x^2-y^2}{(x^2+y^2)^{3/4}}\log(4 + \sin x)\ dx dy
\]
\begin{proof}
We use Fubini's theorem, by writing $dA=dydx = dxdy$, we have
\[
\int_0^1 \int_0^1 \frac{x^2-y^2}{(x^2+y^2)^{3/4}}\log(4 + \sin x)\ dA
\]
Since the inner integral can be shown to converge, as follows:
\[
F_1(x) = \int_0^1 \left(\frac{x^2-y^2}{(x^2+y^2)^{3/4}}\log(4 + \sin x)\right)\ dy
= \log(4+\sin x) \int_0^1 \frac{x^2-y^2}{(x^2+y^2)^{3/4}} \ dy
\]
Therefore our original integral is
\[
\int_0^1 F_1(x) dx
\]
Which if we expand we get the required expression.
\end{proof}

\section*{Problem 11.10} Let $X=Y=[0,1]$ and let $B$ be the 
Borel $\sigma$-algebra. Let $m$ be the Lebesgue measure and
$\mu$ be counting measure on $[0,1]$.
\subsection*{Problem 11.10(a)}If $D=\{(x,y): x=y\}$, show that
$D$ is measurable with respect to $B\times B$.
\begin{proof}
Consider $D=\{(x,y): x=y\}$ is either a point, in which case
it is measurable with measure 0, or it is a line segment, which
can be put in one-to-one correspondence with a segment on the
unit interval, which we know is in $B$, and thus $B$ measurable.
Thus $D$ is $B\times B$ measurable.
\end{proof}

\subsection*{Problem 11.10(b)} Show that
\[
\int_X \int_Y \chi_D(x,y)\mu(dy)m(dx) \ne \int_Y \int_X \chi_D(x,y)m(dx)\mu(dy)
\]
Why does this not contradict Fubini's theorem.
\begin{proof}
We can see that the counting measure of a single point is 1, while
the Borel measure (which defines the Lebesgue meaure) is 0.
Thats why the above relation is not an equality.
Depending on the order, the non-zero measure might be multiplied
by the number of the points, while in the other case, a zero valued
measure is multiplied.
It does not contradict Fubini's theorem as 
the counting measure $\mu$ defined is not $\sigma$-finite, and hence
Fubini's theorem actually does not apply in this case.
\end{proof}

\section*{Problem 11.11}Let $X=Y=\RR$ and let $B$ be the Borel
$\sigma$-algebra. Define
\[
f(x,y) = \left\{ \begin{array}{cc}
1 & x\ge 0\ \mbox{and}\ x\le y< x+1\\
-1 & x\ge 0\ \mbox{and}\ x+1 \le y < x+2\\
0 & \mbox{otherwise}
\end{array} \right.
\]
Then show that
\[
\int \int f(x,y)\ dy\ dx \ne \int \int f(x,y)\ dx\ dy
\]
Why does this not contradict Fubini's theorem.
\begin{proof}
Consider the absolute value of $|f(x,y)|$, then we note that
the absolute integral is not finite, and therefore the order of
the integral can change the value. 
The integral is not finite, because for every value of $x$ we have
$y$ such that $|f(x,y)|\ne 0$, and thus the integral over $\RR$ is not
finite.
This does not contradict Fubini's
theorem, as Fubini's theorem is not applicable in situations where
the absolute value of $|f(x,y)|$ is not integrable.
\end{proof}





\end{document}
