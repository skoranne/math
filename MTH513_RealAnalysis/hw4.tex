%%%%%%%%%%%%%%%%%%%%%%%%%%%%%%%%%%%%%%%%%%%%%%%%%%%%%%%%%%%%%%%%%%%%%%%%%%%%%%%%
%% File    : skoranne_MTH513_HW4.tex
%% Author  : Sandeep Koranne
%% Purpose : Real Analysis III HW4
%%         : Good luck
%%%%%%%%%%%%%%%%%%%%%%%%%%%%%%%%%%%%%%%%%%%%%%%%%%%%%%%%%%%%%%%%%%%%%%%%%%%%%%%%
\documentclass{article}[12pt]
\usepackage{amsmath,amssymb}
\newtheorem{lem}{Lemma}
\newtheorem{proof}{Proof}

\def\CC{\mathbb C}
\def\FF{\mathbb F}
\def\NN{\mathbb N}
\def\QQ{\mathbb Q}
\def\RR{\mathbb R}
\def\ZZ{\mathbb Z}

\begin{document}
\title{Real Analysis MTH 513 HW5}
\date{May 20 2016}
\author{ Ricardo Noe Gerardo Reyes Grimaldo \& Sandeep Koranne}
\maketitle
\section*{Problem 8.9}
Recall that a function $f:\RR\to \RR$ is \emph{convex} if
\[
f(\lambda x+(1-\lambda)y) \le \lambda f(x) + (1-\lambda)f(y)
\]
(NB: we are using the definition of convex as given in this question,
even though a more appropriate definition might be to stipulate that
the secant line lies on one side of the graph, as convex functions
are also defined when the graph lies completely above the line, with equality
defined similarly.

\subsection*{Problem 8.9(a)}
Prove that if $f$ is convex and $x\in\RR$, there exists a real number $c$ such that
$f(y)\ge f(x)+c(y-x)$ for all $y\in \RR$. Graphically, this says that the graph of $f$
lies above the line with slope $c$ that passes through the point $(x,f(x))$.

\begin{proof}
Consider points on $x$-axis such that $a<s<t<u<b$, then if $f$ is convex we 
can show the following:
\[
f(t) \le \left(\frac{t-s}{u-s}\right)f(u) + \left(\frac{u-t}{u-s}\right)f(s)
\]
Where we are using the notation, $f(a,b)=\frac{f(b)-f(a)}{b-a}$.

Using the definition of convex function we can write 
\[
(u-s)f(t) \le (t-s)f(u) + (u-t)f(s)
\]
which is the same as
\[
(u-s)(f(t)-f(s))\le (t-s)(f(u)-f(s))
\]
This gives us the required result, $f(y)\ge f(x)+c(y-x)$.
Geometrically this means that for $t\in (x,y)$, the slope of $xt \ge xy$, for
all $t$, thus the curve lies above the secant line joining $x$ and $y$.
The converse is also true.  
\end{proof}


\subsection*{Problem 8.9(b)}Jensen's Inequality.Let $(X,A,\mu)$ be a measure space, suppose $\mu(X)=1$,
and let $f:\RR\to\RR$ be convex. Let $g:X\to\RR$ be integrable. Prove Jensen's inequality
\[
f(\int g d\mu) \le \int_X f\circ g d\mu
\]
\begin{proof}
Actually we can show that $g$ must be integrable, by putting $t=\int fd\mu$.
Thus $a<t<b$ since $\mu(X)=1$, and calculating the sup over 
all $(a,b)\subset X$, but it is given that $g$ is integrable. By part (a)
there exists $c$ such that
\[
c \le (f(u)-f(t))/(u-t)
\]
Thus $c(t-u) \ge f(t)-f(u)$ for $u\in [t,b)$. So we have a $\gamma\in (a,b)$
where
\[
f(\gamma) \ge f(t) + c(\gamma-t)
\]
Now put $\gamma=g(x)$ for each fixed $x$ to get
\[
(f \circ g)(x) \ge f(t) + c(g(x)-t)
\]
Now $g$ is measurable and $f$ is continuous, therefore $f\circ g$ is
measurable, but the above RHS is integrable, so we can integrate the above
inequality. Note that $\int g d\mu$ is simply a number, and $f(t)$ bounds 
this value. 
Thus we get the required result
\[
f(\int g d\mu) \le \int_X f\circ g d\mu
\]


\end{proof}


\subsection*{Problem 8.9(c)}Deduce that $(\int g)^2 \le g^2$ and $e^{\int g} \le \int e^g$
\begin{proof}
  Consider $g(x)=e^x$, then
  \[
  e^{\int g}
  \]
  
\end{proof}


\subsection*{Problem 8.9(d)}Prove that if $g$ is convex and $a_1,\ldots,a_n\in \RR$, then
\[
g(\frac{1}{n} \sum_{i=1}^n a_i ) \le \frac{1}{n} \sum_{i=1}^n g(a_i)
\]

\begin{proof}
We use induction, for $n=2$, this is clear because $g$ is convex.
Let it be true for $n$, then
\[
g(\frac{1}{n} \sum_{i=1}^n a_i ) \le \frac{1}{n} \sum_{i=1}^n g(a_i)
\]

Let us extend it to $n+1$, as follows
\[
g(\frac{1}{n+1} \sum_{i=1}^{n+1} a_i ) = 
g(\lambda_1 a_1 + (1-\lambda_1)\sum_{i=2}^{n+1} \frac{\lambda_i}{1-\lambda_1}a_i)
\le \lambda_1 g(a_1) + (1-\lambda_1)g(\sum_{i=2}^{n+1} \frac{\lambda_i}{1-\lambda_1}a_i)
\]
This follows from
\[
\sum_{i=2}^{n+1} \frac{\lambda_i}{1-\lambda_1}=1
\]
Now we can use the inductive hypothesis for the second term (since
the number of terms is $n$), and we get the result.
\end{proof}



\end{document}
