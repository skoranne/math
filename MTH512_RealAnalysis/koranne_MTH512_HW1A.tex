%%%%%%%%%%%%%%%%%%%%%%%%%%%%%%%%%%%%%%%%%%%%%%%%%%%%%%%%%%%%%%%%%%%%%%%%%%%%%%%%
%% MTH512 HW1 TeX source
%% Author : Sandeep Koranne
%% Good luck
%%%%%%%%%%%%%%%%%%%%%%%%%%%%%%%%%%%%%%%%%%%%%%%%%%%%%%%%%%%%%%%%%%%%%%%%%%%%%%%%
\documentclass{article}
\usepackage{mathptmx,amssymb,amsmath}
\usepackage{breqn}
\usepackage[normalem]{ulem}
\usepackage{enumerate}
\usepackage[mathscr]{euscript}
\usepackage{enumerate}
\setlength{\textwidth}{16.5cm}
\setlength{\oddsidemargin}{-0.1cm}
\setlength{\evensidemargin}{-0.1cm}
\setlength{\textheight}{23cm}
\setlength{\topmargin}{-1.3cm}
\newtheorem{lem}{Lemma}
\newtheorem{proof}{Proof}
% Some handy shortcuts. 
\def\ge{\geqslant}
\def\le{\leqslant}
\def\phi{\varphi}
\def\to{\rightarrow}
\def\mapsto{\longmapsto}
\def\la{\langle}
\def\ra{\rangle}
\def\Aut{\operatorname{Aut}}
\def\diam{\operatorname{diam}}
\def\Image{\operatorname{Image}}
\def\Ker{\operatorname{Ker}}
\def\GL{\operatorname{GL}}
\def\SL{\operatorname{SL}}
\def\Perm{\operatorname{Perm}}
\def\nor{\vartriangleleft}
\def\nnor{\vartriangleright}
\def\lcm{\operatorname{lcm}}
\def\gcd{\operatorname{gcd}}
\def\li{\operatorname{li}}
\def\min{\operatorname{min}}
\def\max{\operatorname{max}}
\renewcommand{\mod}{\,\operatorname{mod}\,}
\newcommand{\norm}[1]{\left\lVert#1\right\rVert}

\def\CC{\mathbb C}
\def\FF{\mathbb F}
\def\NN{\mathbb N}
\def\QQ{\mathbb Q}
\def\RR{\mathbb R}
\def\ZZ{\mathbb Z}

\pagestyle{myheadings}
\begin{document}

% Replace Name. This is for the headers on all but the first page.
\markright{\hspace{10pt} Math 512 - Winter 2016 \hspace{100pt} Homework 1 - Sandeep Koranne }
\thispagestyle{empty}

\textbf{Math 512 - Winter 2016 \hfill Real Analysis  \hfill Instructor: Ossiander}

\hrulefill 
\medskip 

% Replace Name. This is for the first page header.
 {Sandeep Koranne \hfill Homework 1 \hfill Due: Januay 15, 2016}
\medskip

\begin{enumerate}

%%%%%%%%%%%%%%%%%%%%%%%%%%%%%%%%%%%%%%%%%%%%%%%%%%%%%%

\item (Car. 5.45) Prove that $\NN$ (with its usual metric) is
homeomorphic to $\{(1/n): n \ge 1\}$ (with its usual metric).

\textbf{Solution:} Consider the function $f:\NN \to \RR$ defined
as $f(n) = 1/n$. Clearly $f$ is continuous for $n\ne 0$, and it is
one-to-one since $1/x'=1/x''$ implies $x'=x''$. Moreover, $f$ is
onto as for every $y\ne 0, f(1/(1/y))=y$. The inverse of $f, f^{-1}$
is also continuous since $f^{-1}(x)=f(x)=1/x$. Thus $f$ is an
homeomorphism, and thus $\NN$ is homeomorphic to 
$\{(1/n): n \ge 1\}$, as required.

\item (Car. 5.53) Suppose that we are given a point $x$ and a
sequence $(x_n)$ in a metric space $M$, and suppose that $f(x_n)\to f(x)$
for every continuous real valued function $f$ on $M$. Prove that
$x_n\to x$ in $M$.

\textbf{Solution:} Note that the metric space $M$ has an associated
distance function, let us call it $d$ which is also real valued
and continuous. Thus, if $f(x_n)\to f(x)$ for every continuous
function, then $d(x_n)\to d(x)$ where $d(x)$ is defined as $d(x,0)$.
Thus given $\delta>0, \exists n>0$ such that $d(f(x_n),f(x)) < \delta$.
But now we use the fact that $f$ and $d$ are continuous, thus
for the given $\epsilon>0$ there exists $\delta>0$ such that
\[
d(f(x_n),f(x)) < \delta \implies d(x_n,x) \le \epsilon
\]
As required.


\item (Car. 5.56) Give example of a continuous function that is not open.\\
\textbf{Solution:} Consider the function $f(x)=x^2$ on the open set
$(-1,1)$ whose image is $[0,1)$ and is thus not open, but obviously
$f(x)=x^2$ is continuous.

(b) Give example of open function that is not continuous.\\
\textbf{Solution:} Here we have to use $(N,\rho)$ and assume that
$(N,\rho)$ has some discrete structure. Consider the \textbf{sgn}
function $sgn(x):\RR\to N$ which maps $\RR$ to +1,0 and -1. This function
maps open sets to open sets (any finite discrete set is open), but
this function is not continuous.

(c) Give example of continuous map that is not closed\\
\textbf{Solution:}  Consider the linear projection operator function
$f(x,y)=x$ for $f:\RR^2\to \RR$. This function is continuous, as it
is a linear mapping between two finite dimensional vector spaces. 
However, it does not map the closed set. This can be shown by
considering that $f(U)$ is open for $U\in\RR^2$, where $U$ is an open
set in $\RR^2$.

(d) Give example of closed map that is not continuous\\
\textbf{Solution:} The \emph{floor} function from $\RR\to \ZZ$ is
closed but not continuous.


\item (Car. 5.57) Let $f:(M,d)\to (N,\rho)$ be one-to-one and onto.
Show that the following are equivalent: (i) $f$ is open, (ii) $f$
is closed, and (iii) $f^{-1}$ is continuous. Consequently, $f$
is an homeomorphism if and only if both $f$ and $f^{-1}$ are 
open (closed).

\textbf{Solution:} Let us first write down the definition
of continuity of $f$ using metric distances:
\[
\forall \epsilon > 0, \exists \delta > 0\ \textrm{s.t.} \forall x \in M\ \
d(x,y) < \delta \implies \rho(f(x),f(y)) \le \epsilon
\]
Topologically, $x$ belongs to the the open ball $B(y,\delta)$
and $f(x)$ belongs to the open ball $B(f(y),\epsilon)$, which is
same as $f(B(y,\delta)) \subseteq B(f(y),\epsilon)$.
\begin{lem}
Let $(M,d)$ and $(N,\rho)$ be metric spaces then $f:M\to N$
be continuous if and only if $f^{-1}(G)$ is open in $M$ for
every open set $G\in N$.
\end{lem}
\begin{proof}
Use Theorem 5.1 (iv) of the textbook.
\end{proof}
We use this to first prove (iii) is equivalent to (i)
\begin{proof}
Since $f^{-1}$ is continuous, then $(f^{-1})^{-1}=f$ preserves open sets from
$M$ to $N$. Here we use the fact that if $f$ is continuous, then $f^{-1}$
preserves open sets. 
The converse is also true, and we use that to show (i) implies (iii).
Since $f$ is open, $f^{-1}$ is continuous.
Proof of (i) equivalent to (ii) follows from
the fact that complement of open sets are closed in $M$ and $N$.
That is $f^{-1}(U')=(f^{-1}(U))'$ where $U'$ denotes the complement
of $U$ in the appropriate metric space.
\end{proof}
Thus if both $f$ and $f^{-1}$ are open, then their inverses are
continuous and preserve open sets, thus $f$ is an homeomorphism.

\item (Optional Car. 4.10) Given $y=(y_n)\in H^\infty, N\in \NN$, and
$\epsilon>0$, shown that $x=(x_n)\in H^\infty: |x_k-y_k|<\epsilon, k=1,\ldots,N$
is open in $H^\infty$.
\begin{proof}
Since $H^\infty$ is a metric space we can define an open ball $U$ in $H^\infty$
containing $x=(x_n)$. Since $|x_k-y_k|<\epsilon$ we can 
find $z=(z_n)\in H^\infty$ such that $|x_k-z_k|=|x_k-y_k|/2 \le \epsilon/2$.
Then the ball $B(z,\epsilon/2)$ is open in $H^\infty$ and is contained in 
$x=(x_n)$ thus $x$ is open.
\end{proof}

\item (Optional Car. 5.51) Let $(M,\rho)$ be a separable metric space
with a countable dense subset. Show that:
\begin{enumerate}
\item{Prove that $f$ is one-to-one and continuous. In fact 
$d(f(x),f(y)) \le \rho(x,y)$
where $d$ is the metric on $H^\infty$.}
\begin{proof}
Consider $f(x)$ and $f(y)$ where $x\ne y$, then it is clear that
$(\rho(x,x_n))_k \ne (\rho(y,x_n))_k$ for any $k$, and thus
$(\rho(x,x_n))_{n=1}^\infty \ne (\rho(y,x_n))_{n=1}^\infty$. Thus $f$ is
one-to-one. Each of $(\rho(x,x_n))_k$ is continuous since $\rho$ is
a metric and thus continuous. But then $(\rho(x,x_n))_{n=1}^\infty$
is continuous since a vector valued function which is coordinatewise
continuous is continuous.
\end{proof}
\item{Show that $f^{-1}$ is continuous, by showing that if we let 
$\epsilon>0, \ \exists \delta>0,\ \rho(x,y)<\epsilon$ whenever
$d(f(x),f(y))<\delta$.
}
\begin{proof}
Consider a open set $U\in H^\infty$ such that $x\in U$, then by the
hypothesis, $\rho(x,y)<\epsilon$ and thus $B_\epsilon(x)$ is open, and thus
$f^{-1}$ is continuous since it maps open sets in $H^\infty$ to $M$.
\end{proof}
\end{enumerate}

\item (Optional Car. 5.64) Given $n\in \NN$ and $f,g \in C(\RR)$, let
$d_n(f,g)=\max_{|t|\le n}|f(t)-g(t)|$. Then $d_n$ defines a pseudo-metric
on $C(\RR)$.\\
\textbf{Solution:} It is clear that $\max |f(t)-g(t)|$ is a metric, 
to show that the restriction to finite $n$ makes it 
a pseudo-metric note that with $\max_{|t|\le n}|f(t)-g(t)|$ it is
possible for $f(k)\ne g(k)$ for $k\ge n$, but $d_n(f,g)=0$. Thus
$d_n(f,g)$ is a pseudo-metric.

To show $d(f,g)=\sum_{n=1}^\infty 2^{-n}\frac{d_n(f,g)}{1+d_n(f,g)}$
is a metric. We need a lemma to show that if $d_n$ is a metric then
$\frac{d_n}{1+d_n}$ is also a metric.
\begin{lem}
If $d(x,y)$ is a metric on $M$ then $d'(x,y)=\frac{d(x,y)}{1+d(x,y)}$
is also a metric on $M$.
\end{lem}
\begin{proof}
Since $d(x,y)\ge 0 \implies d'(x,y)\ge 0$ and moreover $d'(x,y)=0$
only when $d(x,y)=0$. Similarly, $d'(y,x)=d'(x,y)$ based on definition.
Consider a function $F(t)=\frac{t}{1+t}$. Since 
\[
F'(t)=\frac{1}{1+t} - \frac{t}{(1+t)^2} \ge 0
\]
$F(t)$ is an increasing function. Thus $d'(x,y)=F(d(x,y))$ is also
an increasing function, which implies
\begin{eqnarray}
d'(x,y)=F(d(x,y)) & \le & F(d(x,z)+d(z,y)) \nonumber \\
                  & = & \frac{d(x,z)+d(z,y)}{1+d(x,z)+d(z,y)} \nonumber \\
                  & = & \frac{d(x,z)}{1+d(x,z)+d(z,y)} + \frac{d(z,y)}{1+d(x,z)+d(z,y)} \nonumber \\
                  & \le & F(d(x,z)) + F(d(z,y)) \nonumber \\
                  & = & d'(x,z) + d'(z,y) 
\end{eqnarray}
Thus $d'(x,y)$ is a metric on $M$ as required.
\end{proof}
Now we use this Lemma to show that $d(f,g)$ is a metric.
\begin{proof}
Let $D_n(f,g)=\frac{d_n(f,g)}{1+d_n(f,g)}$, we have shown above that $D_n$ is
a metric. Next, consider $d(f,g)=\sum_{n=1}^\infty 2^{-n}D_n(f,g)$, clearly
since $D_n \ge 0$ and $2^{-n}\ge 0$, $d(f,g)\ge 0$ and $d(f,g)=0$ only
when $D_n(f,g)=0$. Similarly, $D_n(f,g)=D_n(g,f)$. Consider the triangle
inequality
\[
d(f,h)+d(h,g) = \sum_{n=1}^\infty 2^{-n} D_n(f,h) + \sum_{n=1}^\infty 2^{-n} D_n(h,g)  = 
\sum_{n=1}^\infty 2^{-n}( D_n(f,h) + D_n(h,g)) \ge
\sum_{n=1}^\infty 2^{-n}( D_n(f,g)) = d(f,g)
\]
Since $D_n$ is a metric.
\end{proof}

\end{enumerate}

\end{document}  
