%%%%%%%%%%%%%%%%%%%%%%%%%%%%%%%%%%%%%%%%%%%%%%%%%%%%%%%%%%%%%%%%%%%%%%%%%%%%%%%%
%% MTH512 HW5 TeX source
%% Author : Sandeep Koranne
%% Good luck
%%%%%%%%%%%%%%%%%%%%%%%%%%%%%%%%%%%%%%%%%%%%%%%%%%%%%%%%%%%%%%%%%%%%%%%%%%%%%%%%
\documentclass{article}
\usepackage{mathptmx,amssymb,amsmath}
\usepackage{breqn}
\usepackage[normalem]{ulem}
\usepackage{enumerate}
%\usepackage{graphicx}
\usepackage[mathscr]{euscript}
\usepackage{enumerate}
\setlength{\textwidth}{16.5cm}
\setlength{\oddsidemargin}{-0.1cm}
\setlength{\evensidemargin}{-0.1cm}
\setlength{\textheight}{23cm}
\setlength{\topmargin}{-1.3cm}
\newtheorem{lem}{Lemma}
\newtheorem{proof}{Proof}
% Some handy shortcuts. 
\def\ge{\geqslant}
\def\le{\leqslant}
\def\phi{\varphi}
\def\to{\rightarrow}
\def\mapsto{\longmapsto}
\def\la{\langle}
\def\ra{\rangle}
\def\Aut{\operatorname{Aut}}
\def\diam{\operatorname{diam}}
\def\Image{\operatorname{Image}}
\def\Ker{\operatorname{Ker}}
\def\GL{\operatorname{GL}}
\def\SL{\operatorname{SL}}
\def\Perm{\operatorname{Perm}}
\def\nor{\vartriangleleft}
\def\nnor{\vartriangleright}
\def\lcm{\operatorname{lcm}}
\def\gcd{\operatorname{gcd}}
\def\li{\operatorname{li}}
\def\min{\operatorname{min}}
\def\max{\operatorname{max}}
\renewcommand{\mod}{\,\operatorname{mod}\,}
\newcommand{\norm}[1]{\left\lVert#1\right\rVert}

\def\CC{\mathbb C}
\def\FF{\mathbb F}
\def\NN{\mathbb N}
\def\QQ{\mathbb Q}
\def\RR{\mathbb R}
\def\ZZ{\mathbb Z}

\pagestyle{myheadings}
\begin{document}

% Replace Name. This is for the headers on all but the first page.
\markright{\hspace{10pt} Math 512 - Winter 2016 \hspace{100pt} Homework 5 - Sandeep Koranne }
\thispagestyle{empty}

\textbf{Math 512 - Winter 2016 \hfill Real Analysis  \hfill Instructor: Ossiander}

\hrulefill 
\medskip 

% Replace Name. This is for the first page header.
 {Sandeep Koranne \hfill Homework 5 \hfill Due: Feb 8, 2016}
\medskip

% chapter 8 (57,67,68,83,86) optional (54,59,70,76,84)
% chapter 9 (34,41,44) optional (17,22,27,28)

\section{Chapter 8 Problem 57}
A function $f:\RR \to \RR$ is said to satisfy a Lipschitz
condition of order $\alpha$ where $\alpha > 0$ if there is a
constant $K < \infty$ such that $|f(x)-f(y)|\le K|x-y|^\alpha$
for all $x,y$. Prove that such a function is uniformly continuous.
\begin{proof}
We have addressed the condition where $\alpha>1$ later in this homework
where we show that $f'(x)=0$ implying that $f$ is a constant function.
Thus we only consider the case when $0<\alpha\le 1$. In this case
let $\epsilon>0$ and let $\delta=(\frac{\epsilon}{K})^{1/\alpha}$. Then
we are given
\[
|x-y|< \delta \implies |f(x)-f(y)|\le K|x-y|^\alpha \le K\delta^\alpha = K(\epsilon)/K = \epsilon
\]
This proves that $f$ is uniformly continuous (as the choice of $\delta$
does not depend on $x,y$).
\end{proof}

\section{Chapter 8 Problem 67} Define $f:l_2\to l_1$ by $f(x)=(x_n/n)_{n=1}^\infty$.
Show that $f$ is uniformly continuous.
\begin{proof}
Since $f:l_2\to l_1$, it is a function mapping sequences from $l_2$ into
sequences in $l_1$. Note that the metric in $l_2$ is Euclidean, while
in $l_1$ we use $\sum_{n=1}^\infty x_n$ as the metric. Consider a Lipschitz
condition on $f$, if we can verify that $f$ is actually Lipschitz, then
it would imply that $f$ is uniformly continuous. To verify that $f$
is Lipschitz consider two sequences $\{x_n\},\{z_n\}$:
\begin{equation}
\|f(x)-f(z)\|_1 = \sum_{n=1}^\infty |\frac{x_n}{n} - \frac{z_n}{n}| =
\sum_{n=1}^\infty \frac{1}{n}|x_n - z_n| \label{eqn1}
\end{equation}
While on the other hand consider the $l_2$ distance between $\{x_n\}$
and $\{z_n\}$
\[
d(x,z) = \sqrt{\sum_{n=1}^\infty (x_n - z_n)^2}
\]
We know $|x|_1 \le \sqrt{n}|x|_2$, and we use this to multiply
each term in Equation~\ref{eqn1} by $n$
\[
\sum_{n=1}^\infty |x_n - z_n| = |f(x)-f(z)|_1 \le |x-z|_2
\]
This shows that $f$ is Lipschitz and hence uniformly continuous.
\end{proof}

\section{Chapter 8 Problem 68}
Fix $y\in l_\infty$ and define $g:l_1\to l_1$ by $g(x)=(x_ny_n)_{n=1}^\infty$.
Show that $g$ is uniformly continuous.
\begin{proof}
The proof follows the same technique as the previous problem.
If we are able to show that $g$ is Lipschitz, then we know
$g$ is uniformly continuous. 
To verify that $f$
is Lipschitz consider two sequences $\{x_n\},\{z_n\}$:
\[
|g(x)-g(z)|_1 = \sum_{n=1}^\infty | g(x)_n - g(z)_n| = 
\sum_{n=1}^\infty | x_ny_n - z_ny_n| = \sum_{n=1}^\infty |y_n||x_n-z_n| \le
y |x-z|_1
\]
Here we have used the fact that $y$ is fixed in $l_\infty$. Thus
we have Lipschitz condition and thus $g$ is uniformly continuous.
\end{proof}

\section{Chapter 8 Problem 83}
If $V$ is any normed vector space, show that $B(V,\RR)$ is always
complete. [Hint: use Banach's characterization, Theorem 7.12.]
\begin{proof}
Consider $S_1, S_2, S_3,\ldots \in B(V,\RR)$ and $x\in V$ we claim that
$S_1x, S_2x,\ldots$ is Cauchy in $\RR$. If $x=0$, there is nothing to
show, so consider $x\ne 0$. Since $S$ is Cauchy in $B(V,\RR)$ we have 
for $n,m>N$ $\|S_n-S_m\| < \frac{\epsilon}{\|x\|}$. Thus
\[
|S_nx-S_mx| \le \|S_n-S_m\|\|x\| \le \epsilon
\]
This shows $S_jx$ is Cauchy in $\RR$. Let $S:V\to\RR$ we will show
that $\lim_{n\to\infty} S_nx= Sx$. We know that $S$ is linear since
\[
S(\alpha x+y) = \lim_{n\to\infty} (S_n\alpha x+ S_ny) = \alpha \lim_{n\to\infty} S_nx 
+ \lim_{n\to\infty} S_ny = \alpha Sx + Sy
\]
Next we shown that $S_n\to S$ as $n\to\infty$, consider
\[
\|S_n-S_m\| < \frac{1}{2}\epsilon\qquad n,m > N
\]
This implies
\[
|S_nx - S_mx| \le \frac{1}{2}\epsilon\|x\| \implies
|Sx -S_nx| = |\lim_{j\to\infty}(S_jx-S_nx)| \le \lim_{j\to\infty} |S_jx-S_nx| \le
\lim_{j\to\infty} |S_j-S_n|\|x\| \le \frac{1}{2}\epsilon \|x\|
\]
Therefore
\[
|Sx| \le |S_nx| + |Sx - S_nx| \le (\|S_n\| + \frac{1}{2}\epsilon)\|x\|
\]
This implies
\[
\|S-S_n\| < \frac{1}{2}\epsilon \forall n > N \implies S_n \to S \in B(V,\RR)
\]
Thus $B(V,\RR)$ is complete as we have showed that Cauchy sequences
have a limit in the metric space.
\end{proof}

\section{Chapter 8 Problem 86}
If $(V,\|\cdot\|)$ is an $n$-dimensional normed vector space, show that
there is a norm $|||\cdot|||$ on $\RR^n$ such that $(\RR^n,|||\cdot|||)$
is linearly isometric to $(V,\|\cdot\|)$.
\begin{proof}
Consider a linear function $f$ from $V\to\RR^{n}$.
Consider the norm on $\RR^n$ given by $|||x||| = \|f^{-1}(x)\|$.
Since $f$ is linear, this is well defined, and we know linear functions
preserve distances, thus there is an isometry.
\end{proof}

\section{Chapter 8 Optional Problem 54}
Let $E$ be a bounded, noncompact subset of $\RR$. Show that
there is a continuous function $f:E \to \RR$ that is not
uniformly continuous.
\begin{proof}

\end{proof}

\section{Chapter 8 Optional Problem 59}
The Lipschitz condition is only interesting only for $\alpha \le 1$;
show that a function satisfying a Lipschitz condition of order $\alpha>1$
is constant.
\begin{proof}
Note that the Lipschitz condition states:
\[
|f(x)-f(y)| \le K|x-y|^\alpha \forall x,y\qquad K < \infty
\]
We can rewrite the above as
\[
\frac{|f(x)-f(y)|}{|x-y|} \le K|x-y|^{\alpha-1}
\]
As $x\to y$, the right hand side approaches 0, and the left side
is $f'(x)$. Thus if $f'(x)=0$, then the function is a constant.
\end{proof}
\section{Chapter 8 Optional Problem 70}
\begin{proof}

\end{proof}

\section{Chapter 8 Optional Problem 76}
\begin{proof}

\end{proof}

\section{Chapter 8 Optional Problem 84}
\begin{proof}
If you see our proof of 83, we have not used any property of $\RR$
except that the $S_1x,S_2x,\ldots$ is Cauchy and thus converges
to a limit. But this is true even when the metric is complete. As required.
\end{proof}

\section{Chapter 9 Problem 34}
Let $M$ be complete, and let $E$ be an $F_\sigma$ set in $M$. Prove
that $E$ is a first category set in $M$ if an only if $E^c$ is dense
in $M$.
\begin{proof}
Consider $G_\delta = \bigcap_n G_n$ each $G_n$ open, dense in $M$.
By the Baire Category Theorem since $M$ is complete then $G_\delta$
is not only non-empty, it is also dense in $M$. Next we show that
every meager set has empty interior. If $F_\sigma$ has non-empty
interior then $\exists G$ such that $G\in F_\sigma$ and $G\cap G_\delta=\emptyset$.
But this is a contradiction, thus we know $F_\sigma$ has empty interior.
We also know that interior of $A^c$ is the complement of the closure
of $A$. But if $\bar{A}^c=\emptyset$, then $\bar{A}=M$, which implies
that $A$ is dense in $M$. We have shown above that if $E=F_\sigma$ is a
meager set (first category) then the complement of the closure of 
$(M\setminus E)$ is empty, which is the same as stating that $E^c$ is
dense in $M$ (as we showed with $A$).

In the other direction assume that $E^c$ is dense in $M$. Let
$G_\delta = \bigcap_n G_n$ and $F_\sigma=(G_\delta)^c = \bigcup_n F_n$.
Since $E^c$ is dense, that implies that $E$ is nowhere dense, and thus
$E=F_\sigma$ is meager, as required.
\end{proof}

\section{Chapter 9 Problem 41}
Let $M$ be a complete metric space. Prove that if $(E_n)$ is a sequence
of closed sets in $M$, each having empty interior, then $\bigcup_{n=1}^\infty E_n$
has empty interior.
\begin{proof}
If $\bigcup_n E_n$ had non-empty interior, then consider 
$G \subset F_\sigma \cap G_\delta=\emptyset$, but we are given that $M$
is complete, thus, this is not possible, hence the interior has to be
empty.
\end{proof}

\section{Chapter 9 Problem 44}
If $M$ is complete, show that the conclusion of Baire's theorem holds
for any \emph{open} subsets of $M$. [Hint: See Exercise 7.30.]
\begin{proof}
Consider $A\subset M$ is nowhere dense in $M$ if every open set
$S\subset M$ contains $S'\subset S$ such that $S'\cap A=\emptyset$.
Given any open set $S$ in $M$, Baire's Theorem is valid because
to show $A\subset S$ is nowhere dense, it is sufficient to consider
$S$ itself, and by definition of open sets, and the diameter of $A$, there
will always exist $\epsilon = \diam(S)-\diam(A)$ and we can construct
another open set of radius $\epsilon$ in $S$. Thus the theorem
holds for any open subset of $M$.
\end{proof}

\section{Chapter 9 Optional Problem 17}
Prove that a complete metric space without any isolated points
is uncountable. In particular this gives another proof that
$\Delta$ is uncountable.
\begin{proof}

\end{proof}

\section{Chapter 9 Optional Problem 22}
\begin{proof}

\end{proof}

\section{Chapter 9 Optional Problem 27}
\begin{proof}

\end{proof}

\section{Chapter 9 Optional Problem 28}
\begin{proof}

\end{proof}


\end{document}  
