%%%%%%%%%%%%%%%%%%%%%%%%%%%%%%%%%%%%%%%%%%%%%%%%%%%%%%%%%%%%%%%%%%%%%%%%%%%%%%%%
%% MTH512 HW3 TeX source
%% Author : Sandeep Koranne
%% Good luck
%%%%%%%%%%%%%%%%%%%%%%%%%%%%%%%%%%%%%%%%%%%%%%%%%%%%%%%%%%%%%%%%%%%%%%%%%%%%%%%%
\documentclass{article}
\usepackage{mathptmx,amssymb,amsmath}
\usepackage{breqn}
\usepackage[normalem]{ulem}
\usepackage{enumerate}
%\usepackage{graphicx}
\usepackage[mathscr]{euscript}
\usepackage{enumerate}
\setlength{\textwidth}{16.5cm}
\setlength{\oddsidemargin}{-0.1cm}
\setlength{\evensidemargin}{-0.1cm}
\setlength{\textheight}{23cm}
\setlength{\topmargin}{-1.3cm}
\newtheorem{lem}{Lemma}
\newtheorem{proof}{Proof}
% Some handy shortcuts. 
\def\ge{\geqslant}
\def\le{\leqslant}
\def\phi{\varphi}
\def\to{\rightarrow}
\def\mapsto{\longmapsto}
\def\la{\langle}
\def\ra{\rangle}
\def\Aut{\operatorname{Aut}}
\def\diam{\operatorname{diam}}
\def\Image{\operatorname{Image}}
\def\Ker{\operatorname{Ker}}
\def\GL{\operatorname{GL}}
\def\SL{\operatorname{SL}}
\def\Perm{\operatorname{Perm}}
\def\nor{\vartriangleleft}
\def\nnor{\vartriangleright}
\def\lcm{\operatorname{lcm}}
\def\gcd{\operatorname{gcd}}
\def\li{\operatorname{li}}
\def\min{\operatorname{min}}
\def\max{\operatorname{max}}
\renewcommand{\mod}{\,\operatorname{mod}\,}
\newcommand{\norm}[1]{\left\lVert#1\right\rVert}

\def\CC{\mathbb C}
\def\FF{\mathbb F}
\def\NN{\mathbb N}
\def\QQ{\mathbb Q}
\def\RR{\mathbb R}
\def\ZZ{\mathbb Z}

\pagestyle{myheadings}
\begin{document}

% Replace Name. This is for the headers on all but the first page.
\markright{\hspace{10pt} Math 512 - Winter 2016 \hspace{100pt} Homework 3 - Sandeep Koranne }
\thispagestyle{empty}

\textbf{Math 512 - Winter 2016 \hfill Real Analysis  \hfill Instructor: Ossiander}

\hrulefill 
\medskip 

% Replace Name. This is for the first page header.
 {Sandeep Koranne \hfill Homework 3 \hfill Due: Januay 29, 2016}
\medskip

%% Chapter 6 : 5,6,7,9,13,24 : Optional 2,10,14,17,25,26

\section{Chapter 7 Problem 28} Suppose that every \emph{countable},
 \emph{closed} subset of $M$ is complete. Prove that $M$ is 
complete.
\begin{proof}
Let $A=\{x_1,x_2,\ldots,\}=\{x_i\}_{i> 0}$ be a Cauchy sequence. 
Note that $A$ is countable.
Since $A$ is Cauchy, it has at most one limit point, and adding that
limit point to $A$ we get $\bar{A}$ which is also countable.
Moreover, $\bar{A}$ is closed, and thus as given in the problem
$\bar{A}$ is complete (since it is a countable, closed subset of $M$).
Thus $\{x_n\}\to x\in \bar{A} \in M$. Hence $M$ is complete (as we
have shown that all Cauchy sequences converge to a limit in $M$).

\end{proof}

\section{Chapter 7 Problem 29}
Prove that $M$ is complete if and only if, for each $r>0$, the closed
ball $\{y\in M:d(x,y)\le r\}$ is complete.
\begin{proof}
$\implies$
We can use Theorem 7.9 of the textbook which states that a subset
$A$ of a complete metric space $M$ is complete if and only if $A$
is closed in $M$. Since the closed ball $B(x,r)$ is closed in $M$,
certainly if $B(x,r)$ is not complete, then $M$ is not complete.


$\impliedby$
For the other direction, consider a Cauchy (thus bounded)
sequence $\{x_n\}\in B(x,r)$ for some $r$. Since the closed ball
$B(x,r)$ is complete $\{x_n\}\to x\in B(x,r)\in M$. Thus $M$ is
also complete.

\end{proof}
\section{Chapter 7 Problem 33}
Let $s$ denote the vector space of all finitely nonzero real sequences;
that is, $x=(x_n)\in s$ if $x_n=0$ for all but finitely many $n$. Show
that $s$ is \emph{not} complete under the $\sup$ norm 
$\| x\|_\infty = \sup_n|x_n|$.
\begin{proof}
In a normed vector space, the distance produced by the norm
can be written as
\[
d(x,y) = \| x-y\|_\infty = \sup_n | x_n - y_n|
\]
where $x$ and $y$ are finitely nonzero sequences.
Consider sequences of the form $x_i=\{0,0,1/i^2,0,\ldots\}$.
Clearly $x_i\in s$ since only one term is non-zero, and moreover
$\|x_i\|_\infty=\sup_n |x_{i_n}|=1/i^2$. 
Now consider a sequence of sequences
\[
y = x_1 + x_2 + \cdots + x_n + \cdots
\]
Clearly $y$ is a Cauchy sequence, since $d(x_m,x_n)=\|x_m-x_n\|_\infty$
which is $\sup|x_m-x_n|$. For any $\epsilon>0$, we can find
$N\in \NN$ such that for $m,n>N$, $|1/m^2 - 1/n^2|<\epsilon$. But if
we look at the limit of $y$ as $n\to\infty$, we can see that 
$y^*=\lim_{n\to\infty} y\not\in s$, 
since $y^*$ has infinitely many non-zero terms (since it was defined
to be pointwise addition of $x_i$, each of which have 1 non-zero term).
Thus $y$ converges to a limit point not in $s$, thus $s$ is not complete. 


\end{proof}
\section{Chapter 7 Problem 35} Prove that a normed vector space $X$
is complete if and only if its closed unit ball $B=\{x\in X:\|x\| \le 1\}$
is complete.
\begin{proof}
Since $B$ is a closed subset of $X$, we have shown in Problem 29 that
$B$ is complete if $X$ is complete and moreover $X$
is not complete if $B$ is not complete. 
To show the other direction, 
consider a Cauchy sequence $(x_n)\in X$.
Moreover, since the sequence is Cauchy it must be bounded we 
know $\exists M: (x_k)< M, \ \forall k$; thus we can scale $(x_n)$ by $M$
to get a new sequence $(y_n)=(x_n/M)$ such that $\|y_k\|\le 1, \forall k$.
Thus $(y_n)\in B$, and since $B$ is complete $(y_n)\to y\in B\implies y\in X$.
Thus $My\in X$, and we have shown that all Cauchy sequences in $X$
converge to a limit in $X$, that implies $X$ is complete, as required.
\end{proof}

\section{Chapter 7 Problem 36} The function $f(x)=x^2$ has two obvious
fixed points: $p_0=0$ and $p_1=1$. Show that there is a $0<\delta<1$
such that $|f(x)-p_0|<|x-p_0|$ whenever $|x-p_0|<\delta, x\ne p_0$.
Conclude that $f^n(x)\to p_0$ whenever $|x-p_0|<\delta, x\ne p_0$.
This means that $p_0$ is an \emph{attracting} fixed point for $f$; every
orbit that starts out near 0 converges to 0. In contrast, find a $\delta>0$
such that if $|x-p_1|<\delta, x\ne p_1$, then $|f(x)-p_1|>|x-p_1|$.
This means that $p_1$ is a \emph{repelling} fixed point for $f$;
orbits that start out near 1 are pushed away from 1. In fact, given any
$x\ne 1$, we have $f^n(x)\not\to 1$.
\begin{proof}
We first analyze the point $p_0$.
Consider $\delta=0.5$, then let $x=p_0+\delta$, then
$f(x)=x^2=(p_0+\delta)^2=\delta^2 < x$. Thus
$|f(x)-p_0|<|x-p_0|$. Hence, any value of $0<\delta<1$ results
in $f(x)$ moving closed to $p_0$.
 For $p_1$, clearly we have to consider $|x|\ge 1$, 
as otherwise $|x-p_0|<\delta$
for some $0<\delta<1$ (and we would be in the first case). For any $|x|\ge 1$
we can consider any $\delta>0$, such as $\delta=0.3$. Then
consider $x=p_1+\delta/2$, then clearly $|x-p_1|=\delta/2<\delta$.
\[
f(x)=x^2=(1+\delta/2)^2=1+\delta^2/4+\delta
\]
Thus $|f(x)-1|=|\delta^2/4+\delta| > |(1+\delta)-1|=|\delta|$.

\end{proof}
\section{Chapter 7 Problem 41}
Let $M$ be complete  and let $f:M\to M$ be continuous. If
$f^k$ is a strict contraction for some integer $k>1$, show that $f$
has a unique fixed point.
\begin{proof}
We know $\exists x_0\in M: f^n(x_0)=x_0, n\ge k$, 
using the contraction mapping theorem.
 Moreover $f(f^n(x_0))=x_0$. Compose both
sides by $f^{-1}$ which we know is well defined as $f$ is continuous
and $x_0$ is closed in $M$, thus $f^{-1}(x_0)$ is closed.
We get $f^n(x_0)=x_0=f^{-1}(x_0)$. But this implies
$f(x_0)=x_0$, as required.
\end{proof}

\section{Chapter 7 Problem 42}
Define $T: C[0,1]\to C[0,1]$ by $(T(f))(x)=\int_0^x f(t)dt$.
Show that $T$ is \emph{not} a strict contraction while $T^2$ is.
What is the fixed point of $T$ ?
\begin{proof}
Let use define the metric on $C[0,1]$ using the $\infty$-norm.
Then $T(f)(x) = \int_0^x f(t)dt$, for any functions $f,g\in C[0,1]$
we can define 
\[
d(T(f),T(g)) = \|\int_0^x f(t)dt - \int_0^x g(t)dt\|_\infty \le 
\int_0^x \|f(t)-g(t)\|_\infty dt \le \|f-g\|_\infty
\]
Hence,
\[
|Tf - Tg| \le \|f-g\|_\infty
\]
But, consider $f=e^x$ and $g=0$ (constant function). 
Then $T(f)(x)=\int_0^x e^tdt=e^x=f(x)$.
Then
$d(Tf-0)=\|f-0\|_\infty$, thus $T$ is not a strict contraction. However,
consider $T(T(f))$ using Fubini's Theorem:
\[
(T(T(f)))(x) = \int_0^x f(x_1)(x-x_1)dx
\]
In general
\[
(T^n(f))(x) = \frac{1}{(n-1)!} \int_0^x f(t)(x-t)^{n-1} dt
\]
Thus 
\begin{eqnarray}
d(((T^n)(f))(x), ((T^n)(g))(x)) & \le & \frac{1}{(n-1)!} 
\left(\int_0^x (x-t)^{n-1} dt\right) \|f-g\|_\infty \nonumber \\
  & = & \frac{x^n}{(n-1)!} \|f-g\|_\infty \nonumber \\
  & \le & \frac{1}{(n-1)!} \|f-g\|_\infty \nonumber 
\end{eqnarray}
since the domain is $C[0,1]$. Thus $T^n, n\ge 2$ is always
a strict contraction.
As we showed above $f(x)=e^x$ is a fixed point for $T$.
\end{proof}

\section{Chapter 7 Problem 47} A function $f:(M,d)\to (N,\rho)$
is said to be {\bf uniformly continuous} if $f$ is continuous
and if, given $\epsilon>0$, there is always a single $\delta>0$
such that $\rho(f(x),f(y))<\epsilon$ for any $x,y\in M$ with
$d(x,y)<\delta$. That is $delta$ is allowed to depend on $f$ and
$\epsilon$ but not on $x$ or $y$. Prove that any Lipschitz map
is uniformly continuous.
\begin{proof}
Note that a function is Lipschitz if
\[
d(f(x),f(y)) \le L d(x,y)
\]
for some constant $L>0$. Let $\epsilon>0$ and 
choose $\delta(\epsilon)=\epsilon/L$, then for $d(x,y)<\delta(\epsilon)$
\[
d(f(x),f(y)) \le L d(x,y) \le L \delta(\epsilon) = L \epsilon/L = \epsilon
\]
As required.
\end{proof}

\section{Chapter 7 Problem 48} Prove that a uniformly
continuous map sends Cauchy sequences into Cauchy sequences.
\begin{proof}
Let $(x_n)$ be a Cauchy sequence. Since $f$ is uniformly
continuous, given $\epsilon>0$, there is always a single $\delta>0$
such that $\rho(f(x),f(y))<\epsilon$ for any $x,y\in M$ with
$d(x,y)<\delta$. Since $(x_n)$ is Cauchy, we know $\exists N(\delta)\in \NN$
such that $d(x_m,x_n)<\delta, m,n>N(\delta)$. But then
\[
m,n>N(\delta) \implies d(x_m,x_n)<\delta \implies \rho(f(x_m),f(x_n)) < \epsilon 
\]
Therefore, $\exists N(\delta)\in\NN$ such that $\rho(f(x_m),f(x_n))<\epsilon$,
therefore $f$ sends Cauchy sequences into Cauchy sequences, as required.
\end{proof}
\section{Optional Chapter 7 Problem 30} If $(M,d)$ is complete, prove
that every open subset $G$ of $M$ is homeomorphic
to a complete metric space.
\begin{proof}
Let $F=M\setminus G$ and consider the metric
\[
\rho(x,y) = d(x,y) + |(d(x,F))^{-1} - (d(y,F))^{-1}|
\]
on $G$. Note that $d(x,F)$ is defined as the infemum of the distance
between $x$ and any $y\in F$. Thus $d(x,F)$ is close to zero near
the boundary of $F$, and thus $\rho(x,y)\to\infty$ for $x$ or $y$
close to the boundary. Now consider a sequence $(x_n)\in G$
with limit not in $G$. We show that $(x_n)$ cannot be Cauchy.
For $\epsilon>0$, for $n,m> N\in\NN$, $\rho(x_n,x_m)$ is given by
\[
\rho(x_n,x_m) = d(x_n,x_m) + | (d(x_n,F))^{-1} - (d(x_m,F))^{-1}|
\]
Let $(x_n)\to x^* \not\in G\implies x^*\in F$, thus
$\rho(x_n,x^*)>\epsilon$, and thus $(x_n)$ is not Cauchy. The converse
follows immediately, that if $(x_n)$ is Cauchy and $(x_n)\to x^*\in G$ then
$\rho(x_m,x_n)<\epsilon$, this implies that Cauchy sequences in $G$
have limits in $G$, and thus $G$ is complete.
\end{proof}

\section{Optional Chapter 7 Problem 37} Suppose that $f:(a,b)\to(a,b)$
has a fixed point $p$ in $(a,b)$ and that $f$ is differentiable at
$p$. If $|f'(p)|<1$, prove that $p$ is an attracting fixed point
for $f$. If $|f'(p)|>1$, prove that $p$ is a repelling fixed point for $f$.
\begin{proof}
Consider a sequence $(x_{n+1})=f(x_n)$, and let $(x_n)\to x^*$ (using
the fixed point assumption)
consider the difference
between $x_{n+1}$ and $p$. We can write
\[
x_{n+1}-p = f(x_n)-f(p) = f'(x_0)[x_n-p]
\]
where $x_0$ is an arbitrary point, using the Mean Value Theorem.
Thus if $f'(p)<1$, then as $x_n\to p$ we get $f(x_n)\to f(p)$. This
is because as $(x_n)\to p$ we have $f'(x_n)\to f'(p)$.
\end{proof}
\section{Optional Chapter 7 Problem 39}
The cubic $x^3-x-1$ has a unique real root $x_0$ with $1<x_0<2$.

%% \begin{figure}[htb]
%% \begin{center}
%% \includegraphics[width=8cm]{banach_cube.eps}
%% \caption{Application of Banach contraction mapping principle.}
%% \label{fig:banach_cube}
%% \end{center}
%% \end{figure}

\begin{proof}
We write $x^3-x-1=0$ as $x=\sqrt[3]{x+1}$, since we want our
iterative approximations to stay within the interval. If we choose
to write the approximation as $x=x^3-1$, it is possible that the 
function iterates will not stay contained within the interval.

Using $x=\sqrt[3]{x+1}$ and starting value of $x=1.5$ we get
the first approximation as shown in Table~\ref{tab:root_difference}.
Hence the root is approximately 1.32472.
\begin{table}
\begin{center}
\begin{tabular}{|c|r|r|r|}
\hline
Iteration & $x$ & $\sqrt[3]{x+1}$ & $x^3-x-1$ \\
\hline
\hline
0	& 1.5	& 1.35721	& 0.875 \\
1	& 1.35721	& 1.33086	& 0.142791 \\
2	& 1.33086	& 1.32588	& 0.0263478 \\
3	& 1.32588	& 1.32494	& 0.00497718 \\
4	& 1.32494	& 1.32476	& 0.000944411 \\
5	& 1.32476	& 1.32473	& 0.000179352 \\
6	& 1.32473	& 1.32472	& 3.40661e-05 \\
7	& 1.32472	& 1.32472	& 6.47069e-06 \\
8	& 1.32472	& 1.32472	& 1.22909e-06 \\
9	& 1.32472	& 1.32472	& 2.33461e-07 \\
\hline
\end{tabular}
\caption{Fixed point iterations for root finding.}
\label{tab:root_difference}
\end{center}
\end{table}


\end{proof}

\section{Optional Chapter 7 Problem 49} Suppose that
$f:\QQ\to\RR$ is Lipschitz. Prove that $f$ extends uniquely
to a continuous function $g:\RR\to\RR$.
\begin{proof}
We only need that $\QQ$ be dense in $\RR$, thus consider a 
Cauchy sequence $(x_n)$
of points in $\QQ$ converging to a limit $c\in\RR$.
Since we proved above in problem 48 that uniformly continuous
functions send Cauchy sequences into Cauchy sequences, we know
that $f(x_n)$ is also Cauchy, and thus converges to a limit,
say $L\in\RR$. Indeed, $L$ depends only on $c$ and not on $(x_n)$.
Let there be another sequence $(y_n)\to c$, then the combined 
sequence $(x_n,y_n)\to c$ as well. This implies that the sequence
$\{f(x_n),f(y_n)\}\to L$
but this implies that the subsequence 
$f(y_n)\to L$, also. Thus the limit $L$
is unique and well defined. Thus, let us construct a new function
$g:\RR\to\RR$ such that $g(c)=L$, and $g(x)=f(x), x\in\QQ$ as $f$
is continuous.

To show uniqueness of this extension, consider two functions
$g,h$ which are extensions of $f$. Consider a
sequence $(x_n)\in\QQ\to c\in\RR$, since both $g$ and $h$
are continuous extensions of $f$ we get $g(x_n)\to g(c)$
as well as $h(x_n)\to h(c)$ as $\lim n\to\infty$. However, due
to the extension, $g(x_n)=h(x_n), \forall n\in\NN$, and we have
previously shown that a continuous functions which agrees on rational
points are equal. Thus the extension is unique.
\end{proof}
\end{document}  
