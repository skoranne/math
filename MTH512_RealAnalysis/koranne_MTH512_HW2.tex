%%%%%%%%%%%%%%%%%%%%%%%%%%%%%%%%%%%%%%%%%%%%%%%%%%%%%%%%%%%%%%%%%%%%%%%%%%%%%%%%
%% MTH512 HW2 TeX source
%% Author : Sandeep Koranne
%% Good luck
%%%%%%%%%%%%%%%%%%%%%%%%%%%%%%%%%%%%%%%%%%%%%%%%%%%%%%%%%%%%%%%%%%%%%%%%%%%%%%%%
\documentclass{article}
\usepackage{mathptmx,amssymb,amsmath}
\usepackage{breqn}
\usepackage[normalem]{ulem}
\usepackage{enumerate}
\usepackage[mathscr]{euscript}
\usepackage{enumerate}
\setlength{\textwidth}{16.5cm}
\setlength{\oddsidemargin}{-0.1cm}
\setlength{\evensidemargin}{-0.1cm}
\setlength{\textheight}{23cm}
\setlength{\topmargin}{-1.3cm}
\newtheorem{lem}{Lemma}
\newtheorem{proof}{Proof}
% Some handy shortcuts. 
\def\ge{\geqslant}
\def\le{\leqslant}
\def\phi{\varphi}
\def\to{\rightarrow}
\def\mapsto{\longmapsto}
\def\la{\langle}
\def\ra{\rangle}
\def\Aut{\operatorname{Aut}}
\def\diam{\operatorname{diam}}
\def\Image{\operatorname{Image}}
\def\Ker{\operatorname{Ker}}
\def\GL{\operatorname{GL}}
\def\SL{\operatorname{SL}}
\def\Perm{\operatorname{Perm}}
\def\nor{\vartriangleleft}
\def\nnor{\vartriangleright}
\def\lcm{\operatorname{lcm}}
\def\gcd{\operatorname{gcd}}
\def\li{\operatorname{li}}
\def\min{\operatorname{min}}
\def\max{\operatorname{max}}
\renewcommand{\mod}{\,\operatorname{mod}\,}
\newcommand{\norm}[1]{\left\lVert#1\right\rVert}

\def\CC{\mathbb C}
\def\FF{\mathbb F}
\def\NN{\mathbb N}
\def\QQ{\mathbb Q}
\def\RR{\mathbb R}
\def\ZZ{\mathbb Z}

\pagestyle{myheadings}
\begin{document}

% Replace Name. This is for the headers on all but the first page.
\markright{\hspace{10pt} Math 512 - Winter 2016 \hspace{100pt} Homework 2 - Sandeep Koranne }
\thispagestyle{empty}

\textbf{Math 512 - Winter 2016 \hfill Real Analysis  \hfill Instructor: Ossiander}

\hrulefill 
\medskip 

% Replace Name. This is for the first page header.
 {Sandeep Koranne \hfill Homework 2 \hfill Due: Januay 22, 2016}
\medskip

%% Chapter 6 : 5,6,7,9,13,24 : Optional 2,10,14,17,25,26

\section{Chapter 6 Problem 5}
If $E$ and $F$ are connected subsets of $M$ with $E\bigcap F\ne \emptyset$,
show that $E\bigcup\ F$ is connected.

We first need a general Lemma.
\begin{lem}
\label{lem1}
Let $C_i$ be a family of connected non-empty subsets of $M$ such that
$\bigcap_{i\in I} C_i\ne\emptyset$, where $I$ is an indexing set; then
$\bigcup_{i\in I} C_i$ is connected.
\end{lem}
\begin{proof}
Let $A=\bigcup_{i\in I} C_i$. If $A$ is not connected then $\exists G_1,G_2$ 
disjoint open sets $(G_1\bigcap G_2=\emptyset)$, 
such that $A\subset G_1 \bigcup G_2$ and $A\bigcap G_1\ne\emptyset$ and
$A\bigcap G_2\ne\emptyset$. Consider one $C_i$ for some $i\in I$, since
$A\subset G_1 \bigcup G_2\implies C_i\subset G_1\bigcup G_2$.
Since $C_i$ is connected, $C_i\bigcap G_1=\emptyset$ or $C_i\bigcap G_2=\emptyset$, otherwise
$C_i$ itself would be disconnected. But this implies that either $C_i\subset G_1$
or $C_i\subset G_2$. But since $\bigcap_{i\in I}C_i \ne\emptyset$, all of $C_i$
are either in $G_1$ or all of $C_i$ are contained in $G_2$. But therefore all of $A$
is contained either in $G_1$ or $G_2$. But this is in contradiction to our assumption
that $A$ is not connected. Hence, $A=\bigcup_{i\in I}C_i$ is connected.
\end{proof}
Now we use the above Lemma to prove the particular case at hand.
\begin{proof}
Use the above lemma (Lemma~\ref{lem1}) with $E=C_1$ and $F=C_2$. Since $E\bigcap F\ne\emptyset$
the criteria for non-empty intersection is met and thus $E\bigcup F$
is connected.
\end{proof}
\section{Chapter 6 Problem 6}
More generally, if $C$ is a collection of connected subsets of $M$, 
all having a point in common, prove that $\bigcup\ C$ is connected. Use this
to give another proof that $\RR$ is connected.
\begin{proof}
Use the above Lemma (Lemma~\ref{lem1}) to immediately prove
the first part that $\bigcup C$ is connected.
 To show that $\RR$ is connected, partition
$\RR=(-\infty,x]\bigcup[x,\infty)$ where the two intervals are connected, and
their intersection is $x$ and thus non-empty, thus $\RR$ is connected.
\end{proof}
\section{Chapter 6 Problem 7}
If every pair of points in $M$ is contained in some connected set, show
that $M$ is itself connected.
\begin{proof}
We prove by contradiction. Let $M$ be not connected, then there exists
two open non-empty disjoint sets $G_1$ and $G_2$ such that $M\subset G_1 \bigcup G_2$.
Since $G_1$ and $G_2$ are both non-empty, there is a point $x\in G_1$ and
$y\in G_2$. But as given, every pair of points in $M$ is connected, this
implies that $G_1$ and $G_2$ are connected. But since $G_1$ and $G_2$ are
open, this implies that $G_1\bigcap G_2\ne\emptyset$. This is a
contradiction to our assumption that $G_1$ and $G_2$ are disjoint. Thus $M$
is connected.
\end{proof}

\section{Chapter 6 Problem 9}
If $A\subset B \subset \bar{A} \subset M$, and if $A$ is
connected, show that $B$ is connected. In particular, $\bar{A}$
is connected.
\begin{proof}
Assume that $B$ is not connected, then there exist two non-empty disjoint
open sets $G_1$ and $G_2$ such that
\[
G_1\bigcap G_2=\emptyset, \ \ B\subset G_1\bigcup G_2,\ \ B\bigcap G_1\ne\emptyset,\ \ B\bigcap G_2\ne\emptyset
\]
This implies $A\subset G_1\bigcup G_2$.
Since $B\bigcap G_1\ne\emptyset, \ \exists x\in B\bigcap G_1$ such that $\exists r>0$
$B(x,r)\subset G_1$. Since $B\subset\bar{A}$, $B(x,r)\bigcap A\ne\emptyset$.
Let $y\in B(x,r)\bigcap A$, thus $y\in A$ as well as $y\in G_1$.
Thus $A\bigcap G_1\ne\emptyset$. Similarly, since $B\bigcap G_2\ne\emptyset$ we can show
that $A\bigcap G_2\ne\emptyset$. But this implies that $A$ itself is not connected.
Which is a contradiction.

For the second part, just take $B=\bar{A}$, then if $A$
is connected $\bar{A}$ is connected, as required.
\end{proof}
\section{Chapter 6 Problem 13}
If $f:[a,b]\to [a,b]$ is continuous, show that $f$ has a fixed point; that
is, show that there is some point $x$ in $[a,b]$ with $f(x)=x$.

We first need a Lemma which is a generalization of the one given
in the textbook.
\begin{lem}
Given a metric space $(X,d)$, a subset $A\subset X$ is 
connected if and only if, $a,b\in X,\ a<b \implies [a,b] \subset X$.
Moreover this implies $[f(a),f(b)]\subset f([a,b])$.
\end{lem}
\begin{proof}
Use Theorem 6.6 of the textbook, and we know $[a,b]$ is connected
in $X$, thus $[f(a),f(b)]$ is connected (written assuming $f(a)<f(b)$,
but this can be assumed without loss of generality).
\end{proof}

Thus $f([a,b])$ is connected.
Now that we have the above Lemma, we also assume without loss of
generality that $f(a) < y < f(b)$, and we can write $f([a,b])=[f(a),f(b)]$.
Let $S=\{x|x\in [a,b]; f(x) < y\}$.
Now $S\ne\emptyset$ since $a\in S$. Let $L=[a,x)$ and $U=(x,b]$.
Note that $L\bigcap U=\emptyset$ and $L\ne\emptyset$ since $f(a)\in L$,
similarly $U\ne\emptyset$ since $f(b)\in U$.
Let $A=f([a,b])\bigcap L$ and $B=f([a,b])\bigcap U$.
Thus $f([a,b])= A\bigcup B$. Now suppose there does not exist a point
$x\in [a,b]$ such that $f(x)=y$, then $A$ and $B$ would be separated, which
would contradict the fact that $[f(a),f(b)]$ is connected. Thus 
$\exists x\in [a,b]: f(x)=y\in [f(a),f(b)]$.
This is the \emph{intermediate value theorem} for general metric spaces.
We use this to prove the \emph{fixed point} theorem.
\begin{proof}
Consider the auxiliary function $g(x)=x-f(x)$, then
$g(a)=a-f(a)$ and $g(b)=b-f(b)$, since $f:[a,b]\to [a,b]$ this
implies $g(a) \le 0 \le g(b)$. Using the intermediate value theorem
we know $\exists x\in [a,b]: g(x)=0$, but this implies $\exists x\in [a,b]$
such that $f(x)=x$ as required.
\end{proof}

\section{Chapter 6 Problem 24}
Show that $(0,1)\times(0,1)$, although an open set in $\RR^2$,
cannot be written as a disjoint union of open balls in $\RR^2$.

Here it is important to note that the metric remains fixed, in
other words, we are not allowed to
 demonstrate a single $\infty$-norm metric ball
which covers $(0,1)\times(0,1)$.
\begin{proof}
We prove by contradiction. Assume that $\RR^2$ is disconnected
and thus can be partitioned into two disjoint open balls
in $\RR^2$.
Let $U,V \in \RR^2$ such that $\RR^2 \subset U \bigcup V$ and
$U\bigcap V=\emptyset$. Let $x=(x_1,x_2)\in U$ and similarly
$y=(y_1,y_2)\in V$. Consider a function 
\[
f(t)=((tx_1+(1-t)y_1), (tx_2+(1-t)y_2))
\] where $f:[0,1]\to \RR^2$. Obviously $f(t)$ is continuous
since
\[
|t_1-t_2|< \delta \implies d(f(t_1),f(t_2)) = 
\frac{|t_1-t_2|}{\sqrt{(x_1-x_2)^2+(y_1-y_2)^2}} \le \epsilon
\]
where $\delta = \frac{\epsilon}{\sqrt{(x_1-x_2)^2+(y_1-y_2)^2}}>0$.

Once we know $f$ is continuous, we can
evaluate $f(0)=(y_1,y_2)\in V$ and $f(1)=(x_1,x_2)\in U$.
Thus $f[(0,1)]$ is disconnected as $U\bigcap V=\emptyset$.
Now since $[0,1]$ is connected, $f$ is continuous, $f([0,1])$
should also be connected; this is a contradiction. Thus $\RR^2$
is connected and thus cannot be written as a disjoint union of
open balls in $\RR^2$.
\end{proof}

\section{Optional Chapter 6 Problem 2}
Show that the only nonempty connected subsets of $\Delta$ are
singletons. (We would say that $\Delta$ is \emph{totally 
disconnected}.)

We first need a Lemma.
\begin{lem}
If $A\subset \RR$, then $A$ is connected if and only if $A$
is an interval.
\end{lem}

\begin{proof}
Suppose $A$ is connected, but is not an interval, then 
$\exists (x,y,z)\in \RR$ such that $x<z<y$ and $x,y\in A$ but
$z$ not in $A$. Then we can construct two disjoint sets $G_1=(-\infty,z)$
and $G_2=(z,\infty)$, and thus $A\subset G_1\bigcup G_2$ and
moreover $x\in G_1$ and $y \in G_2$, thus $A$
is not connected. A contradiction.
\end{proof}
\begin{proof}
Using the above Lemma we know that the only connected subsets of $\RR$
are intervals. Since $\Delta$ is also a subset of $\RR$, the above
lemma is applicable, and moreover we know that no intervals are
present in $\Delta$. Singletons are connected by definition, thus
the only nonempty connected subsets of $\Delta$ are
singletons.

\end{proof}

\section{Optional Chapter 6 Problem 10}
True or false? If $A\subset B \subset C \subset M$, where
$A$ and $C$ are connected, then $B$ is connected.
\begin{proof}
False. If $A\subset B \subset C$ and $A$ and $C$ are connected, then
there can still exist $A\subset B\subset C$ such that $B$ is not connected.
Consider $A$ to be a singleton contained in $B_1$;
while $B_2$ is disjoint from $B_1$, and
$B=(B_1\bigcup B_2)\subset C$. Then we have met all the criteria, but $B$ is disconnected.
\end{proof}
\section{Optional Chapter 6 Problem 14}
Let $f:[0,2]\to \RR$ be continuous with $f(0)=f(2)$. Show that
there is some $x$ in $[0,1]$ such that $f(x)=f(x+1)$.
\begin{proof}
Since $f(0)=f(2)$, $f$ is not monotone, thus there exist at least some
$x\in[0,2]$ such that $f(x)=y$ is repeated. Define \emph{girth} of
a set $X$ as $\inf d(a,b):a,b\in X$, or in other words the infemum
of the pairwise distances between the points of the set. This
is the logical inverse of the \emph{diameter}. Define a function
$g(y)=girth(f^{-1}(y))$.  Note that $g$ is continuous.

Note that $g(f(0))=2$ since $f(0)=f(2)$. Now since $f$ is not
monotone, it has either a maximum or minimum value in the domain.
At this extrema, let the value of $f(x_0)=y_0$, then $g(y_0)=0$.
Thus $g$ takes value at least between $[0,2]$, and thus using the
intermediate value theorem, $\exists y_1:g(y_1)=1$. But if the
distance between points for $f^{-1}(y_1)$ is 1, 
this implies $y_1=f(x)=f(x+1)$, as
required. Note that above we have assumed that $y=f(0)=f(2)$ is not
taken on by $f$ at other points. But this can be remedied by modifying
$f$ to take $f(0)+\epsilon$ at $x\ne 0$ and $x\ne 2$, if $f(x)=f(0)$.

\end{proof}
\section{Optional Chapter 6 Problem 17}
Prove that there does not exist a continuous function
$f:\RR \to \RR$ satisfying $f(\QQ)\subset \RR\setminus\QQ$ and
$f(\RR\setminus\QQ) \subset \QQ$.
\begin{proof}
We prove by contradiction. Let us assume that such an $f$ exists, then
consider $g:[0,1]\to\RR$ defined as
\[
g(x) = f(x)-x
\]
We consider three possibilities, (i) $g(x)\in \QQ$, (ii) $g(x)\in\RR\setminus\QQ$
and (iii) $g(x)=c$ constant function. If $g(x)\in\QQ$, then for $x\in[0,1]$
we get if $x\in\QQ$ then $g(x)+x\in\QQ\implies f(x)\in\QQ\implies x\in\RR\setminus \QQ$.
A contradiction. Similarly, if $x\in\RR\setminus\QQ$ then $g(x)-f(x)\in\QQ\implies x\in\QQ$.
A contradiction.

If we assume that $g([0,1])\in\RR\setminus\QQ$, then since $f$ is continuous, 
$g$ is also continuous, we can use the intermediate value theorem to
specify the existence of a rational $r\in[\min g,\max g]$ such that $g(x)=r,x\in[0,1]$.
But this is a contradiction since we assumed $g(x)$ was irrational. Thus let us
consider $g(x)=c$, but then $f(x)=c+x$ and for $f(c)=2c$ is also irrational, which
contradicts the conditions of the map $f$. Thus, no such $f$ can exist.
\end{proof}

\section{Optional Chapter 6 Problem 25}
\begin{enumerate}[(a)]
\item{Give an example of a continuous function having a
connected range but a disconnected domain}:
\begin{proof}
Consider a constant function $f:[a,b]\setminus\{c\}=1$. Since it is a constant
function it is continuous, and since its range is a singleton it is
connected, and its domain is disconnected as it has a point $c$
removed from the interval $[a,b]$.
\end{proof}

\item{Let $D\subset \RR$, and let $f:D\to\RR$ be continuous. Prove
that $D$ is connected if $\{(x,f(x)):x\in D\}$, the graph of $f$,
is a connected subset of $\RR^2$}:
\begin{proof}
Since $\{(x,f(x))\}$ is connected, a parametric representation of
$f(x)=(t\in[0,1],f(t))$ is also connected, thus $f(t)$ is connected, and since
$f$ is continuous, this implies $f^{-1}(x)$ is also connected, but 
this immediately implies that the domain is connected, as required.
\end{proof}

\end{enumerate}
\section{Optional Chapter 6 Problem 26}
Let $f:[0,1]\to\RR$ be defined by $f(x)=\sin(1/x)$ for $x\ne 0$ and
$f(0)=0$. Show that although $f$ is not continuous, the graph of $f$
is a connected subset of $\RR^2$.

Let us write $S(x,y)=(x,f(x))$ as the union of two parts $S_+(x,y)$
where $x\ne 0$ and $y=sin(1/x)$ and $S_0(x,y)=(0,0)$.
Note that $S_+(x,y)$ is connected as $\sin(1/x)$ is continuous in
the domain.
 Next we need a Lemma.
\begin{lem} The closure of $S_+$ in $\RR^2$ is $S$.
\end{lem}
\begin{proof}
Since $(1/x_n)$ and $\sin(1/x_n)$ both tend to 0 when $n\to\infty$,
it is clear that $(0,0)$ is a limit point of $S_+$ and thus $S=S_++S_0$
is the closure of $S$.
\end{proof}

Now that we have $S_+(x,y)\subset S(x,y)$ and $S(x,y)$ is the closure
of $S_+(x,y)$ we can use our previous result which states that if
$A\subset B\subset\bar{A}$, and $A$ is connected, then $\bar{A}$
as well as $B$ are also connected. This implies that $S$ is connected
as well.

\end{document}  
