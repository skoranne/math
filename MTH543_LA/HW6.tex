%%%%%%%%%%%%%%%%%%%%%%%%%%%%%%%%%%%%%%%%%%%%%%%%%%%%%%%%%%%%%%%%%%%%%%%%%%%%%%%%
%% HW6
%% Sandeep Koranne
%%%%%%%%%%%%%%%%%%%%%%%%%%%%%%%%%%%%%%%%%%%%%%%%%%%%%%%%%%%%%%%%%%%%%%%%%%%%%%%%
\documentclass{article}[12pt]
\usepackage{amssymb}
\usepackage{amsmath}
\usepackage{amsfonts}
\usepackage{amsthm}
\newtheorem{lem}{Lemma}

\begin{document}
\title{Fall 2015 MTH 543 Homework 6}
\author{Sandeep Koranne}
\maketitle

\section{Problem 29}
\subsection{Matrix 1}
Consider $A=\begin{pmatrix} 1 & 2 \\ 3 & 2 \end{pmatrix}$ for $F=R$.
\subsubsection{Determine all the eigenvalues of $A$}
The characteristic polynomial of $A$ can be calculated using the determinant 
\[
\textrm{det}(\lambda I-A) = \begin{vmatrix} \lambda-1 & -2 \\ -3 & \lambda-2 \end{vmatrix} \\
\]
The characteristic polynomial $\lambda^2-3\lambda-4$ has 
roots $\lambda=4$ and $\lambda=-1$, and these
are the eigenvalues of $A$. 

\subsubsection{Calculate the eigenvectors of $A$ for each $\lambda$}
For $\lambda=4$ we solve
$(4I-A)x=0$, to get
\[
\begin{pmatrix}
4-1 & -2 \\
-3 & 4-2
\end{pmatrix} \begin{pmatrix} x_1 \\ x_2 \end{pmatrix} = 
\begin{pmatrix} 0 \\ 0
\end{pmatrix}
\]
This linear equation has infinitely many solutions 
of the form $3x_1=2x_2$ and we can pick $(2,3)$.
For $\lambda=-1$ we get $(-I-A)x=0$, to get
\[
\begin{pmatrix}
-1-1 & -2 \\
-3 & -1-2
\end{pmatrix} \begin{pmatrix} x_1 \\ x_2 \end{pmatrix} = 
\begin{pmatrix} 0 \\ 0
\end{pmatrix}
\]
Solving we get $(1,-1)$ as an eigenvector.

\subsection{Basis for $F^n$}
The set of eigenvectors $\beta=\left\{(2,3),(1,-1)\right\}$ is linearly
independent and is thus a basis. Thus
\[
Q = [I]_\beta^\alpha = \begin{pmatrix} 2 & 1 \\ 3 & -1 \end{pmatrix}
\]
and 
\[
Q^{-1}=\frac{1}{5} \begin{pmatrix} 1 & 1 \\ 3 & -2 \end{pmatrix}
\]
where $\alpha$ is the standard basis for $F^2$ or $R^2$ (in this example).
Thus the diagonal matrix $D=Q^{-1}AQ$ is
\[
D = \frac{1}{5} \begin{pmatrix} 1 & 1 \\ 3 & -2 \end{pmatrix}
                \begin{pmatrix} 1 & 2 \\ 3 & 2 \end{pmatrix}
                \begin{pmatrix} 2 & 1 \\ 3 & -1 \end{pmatrix} = 
\begin{pmatrix}
4 & 0 \\
0 & -1
\end{pmatrix}
\]

\subsection{Matrix 2}
Consider $A$ as
\[
A = \begin{pmatrix}
0 & -2 & -3 \\
-1 & 1 & -1 \\
2 & 2 & 5
\end{pmatrix}
\]
Again considering the determinant $\textrm{det}(\lambda I - A)$ the
characteristic polynomial can be calculated as
\[
-\lambda^3 + 6\lambda^2-11\lambda+6
\]
This polynomial has roots $\lambda=1$, $\lambda=2$, and $\lambda=3$.

\subsubsection{Eigenvectors}
The eigenvectors corresponding to $\lambda=1$ can be calculated using
the equation $(I-A)x=0$, as:
\[
\begin{pmatrix}
1 & 2 & 3 \\
1 & 0 & 1 \\
-2 & -2 & -4
\end{pmatrix} x = \begin{pmatrix} 0 \\ 0 \\ 0 \end{pmatrix}
\]
Solving, we get $(1,1,-1)$ as the eigenvector.
Similarly, for $\lambda=2$, we solve $(2I-A)x=0$, as:
\[
\begin{pmatrix}
2 & 2 & 3 \\
1 & 1 & 1 \\
-2 & -2 & -3
\end{pmatrix} x = \begin{pmatrix} 0 \\ 0 \\ 0 \end{pmatrix}
\]
Solving, we get $(1,-1,0)$ as the eigenvector corresponding to $\lambda=2$.
For $\lambda=3$, we solve $(3I-A)x=0$, as:
\[
\begin{pmatrix}
3 & 2 & 3 \\
1 & 2 & 1 \\
-2 & -2 & -2
\end{pmatrix} x = \begin{pmatrix} 0 \\ 0 \\ 0 \end{pmatrix}
\]
Solving, we get $(1,0,-1)$. 

\subsubsection{Basis for $F^n$}
The set $\beta=\left\{(1,1,-1),(1,-1,0),(1,0,-1)\right\}$ 
is linearly independent as the determinant of $Q$
\[
Q = [I]_\beta^\alpha = \begin{pmatrix}
1 & 1 & 1 \\
1 & -1 & 0 \\
-1 & 0 & -1
\end{pmatrix}
\]
is non-zero (in particular $\textrm{det}(Q)=1$), and $Q^{-1}$
\[
Q^{-1} = \begin{pmatrix}
1 & 1 & 1\\
1 & 0 & 1\\
-1 & -1 & -2
\end{pmatrix}
\]
Thus $D=Q^{-1}AQ$ is:
\[
D = \begin{pmatrix}
1 & 1 & 1\\
1 & 0 & 1\\
-1 & -1 & -2
\end{pmatrix}
\begin{pmatrix}
0 & -2 & -3 \\
-1 & 1 & -1 \\
2 & 2 & 5
\end{pmatrix}
\begin{pmatrix}
1 & 1 & 1 \\
1 & -1 & 0 \\
-1 & 0 & -1
\end{pmatrix} 
= \begin{pmatrix}
1 & 0 & 0 \\
0 & 2 & 0 \\
0 & 0 & 3
\end{pmatrix}
\]

\subsection{Matrix 3}
Consider the matrix $A$ given below:
\[
A = \begin{pmatrix}
i & 1 \\
2 & -i
\end{pmatrix}
\]
\subsubsection{Eigenvalues of $A$}
Consider the determinant of $\lambda I - A$ as:
\[
\begin{vmatrix}
\lambda-i & -1 \\
-2 & \lambda+i
\end{vmatrix}
\]
The characteristic polynomial is $\lambda^2-1$ and has roots $\lambda=-1$
and $\lambda=1$.

\subsubsection{Eigenvectors of $A$}
The eigenvector corresponding to $\lambda=-1$ can be calculated by
solving $(-I-A)x=0$
\[
\begin{pmatrix}
-1-i & -1 \\
-2 & -1+i
\end{pmatrix}x = \begin{pmatrix} 0 \\ 0 \end{pmatrix}
\]
Solving, we get $(1,-i-1)$ as the corresponding eigenvector.
Similarly for $\lambda=1$,  we get $(1,-i+1)$ as the eigenvector.

\subsubsection{Basis for $F^n$}
We can construct the $Q$ matrix as
\[
Q = \begin{pmatrix} 1 & 1 \\ -i-1 & -i+1 \end{pmatrix}
\]
and since $\textrm{det}(Q)=2$ these eigenvectors form a basis and $Q^{-1}$ is
\[
Q^{-1} = \frac{1}{2}\begin{pmatrix} -i+1 & -1 \\ i+1 & 1 \end{pmatrix}
\]
Thus $D=Q^{-1}AQ$ is
\[
D = \frac{1}{2}\begin{pmatrix} -i+1 & -1 \\ i+1 & 1 \end{pmatrix}
\begin{pmatrix}  i & 1 \\ 2 & -i \end{pmatrix}
\begin{pmatrix} 1 & 1 \\ -i-1 & -i+1 \end{pmatrix}
= \begin{pmatrix} -1 & 0 \\ 0 & 1 \end{pmatrix}
\]

\subsection{Matrix 4}
Consider the matrix $A$ given below
\[
A = \begin{pmatrix} 2 & 0 & -1 \\ 4 & 1 & -4 \\ 2 & 0 & -1 \end{pmatrix}
\]
\subsubsection{Eigenvalues of $A$}
The eigenvalues values of $A$ can be calculated using the determinant
of $(\lambda I-A)$, as
\[
\begin{vmatrix}
\lambda-2 & 0 & 1\\
-4 & \lambda-1 & 4 \\
-2 & 0 & \lambda+1
\end{vmatrix}
\]
The characteristic polynomial is $-\lambda^3+2\lambda^2-\lambda$, and
the roots are $\lambda=0$, $\lambda=1$ and $\lambda=1$.

\subsubsection{Eigenvectors of $A$}
The eigenvector corresponding to $\lambda=0$ can be calculated by
solving $(-A)x=0$, to get
\[
\begin{pmatrix} 2 & 0 & -1 \\ 4 & 1 & -4 \\ 2 & 0 & -1 \end{pmatrix}x
= \begin{pmatrix} 0 \\ 0 \\ 0 \end{pmatrix}
\]
Solving for the null space of $A$ we get $(1,4,2)$ as the eigenvector
corresponding to $\lambda=0$. For $\lambda=1$ we solve for
$(I - A)x=0$ as:
\[
\begin{pmatrix} -1 & 0 & 1 \\ -4 & 0 & 4 \\ -2 & 0 & 2 \end{pmatrix}x
= \begin{pmatrix} 0 \\ 0 \\ 0 \end{pmatrix}
\]
Solving for the null space of $I-A$ we get two eigenvectors
$(0,1,0)$ and $(1,0,1)$.

\subsubsection{Basis for $F^n$}
Let us construct the $Q$ matrix as $Q=[I]_\beta^\alpha$ as
\[
Q = \begin{pmatrix} 1 & 0 & 1 \\ 4 & 1 & 0 \\ 2 & 0 & 1 \end{pmatrix}
\]
These vectors form a basis as $\textrm{det}(Q)=-1$ is non-zero, thus
the vectors are linearly independent. And indeed, $Q^{-1}$ is:
\[
Q^{-1} = \begin{pmatrix} -1 & 0 & 1\\4 & 1 & -4 \\ 2 & 0 & -1 \end{pmatrix}
\]
Thus $D=Q^{-1}AQ$ is:
\[
D = \begin{pmatrix} -1 & 0 & 1\\4 & 1 & -4 \\ 2 & 0 & -1 \end{pmatrix}
\begin{pmatrix} 2 & 0 & -1 \\ 4 & 1 & -4 \\ 2 & 0 & -1 \end{pmatrix}
\begin{pmatrix} 1 & 0 & 1 \\ 4 & 1 & 0 \\ 2 & 0 & 1 \end{pmatrix}=
\begin{pmatrix}
0 & 0 & 0\\
0 & 1 & 0\\
0 & 0 & 1
\end{pmatrix}
\]
\section{Problem 30}
We first need a Lemma.
Consider a block matrix $M$ of dimension $n\times n$, then
\begin{lem}
\[
\begin{vmatrix}
I_p & C_{n-p} \\
0 & A_{n-p}
\end{vmatrix} = \textrm{det}(A)
\]
\end{lem}
\begin{proof}
If $M=\begin{pmatrix} I_p & C_{n-p} \\ 0 & A_{n-p} \end{pmatrix}$ then
$\textrm{det}(M)=I_{11}|M_{n-1,n-1}|$, and $I_{11}=1$, thus
$\textrm{det}(M)=|M_{n-1,n-1}|$, by cofactor expansion along the diagonal of
$I_{p}$. Continuing $p$ times, we get $\textrm{det}(M)=\textrm{det}A$. As
required.
\end{proof}
\section{Problem 31}

\section{Problem 32}
We have $f(t) = \textrm{det}(tI-A) = t^n + b_{n-1}t^{n-1} + \cdots + b_1 t + b_0$.
\subsection{Show $b_{n-1} = -\textrm{tr}(A)$}
Consider the minor expansion form of the determinant. Every minor
expansion (except diagonal $a_{ii}$) removes one row and one column, thus
the maximum degree of $t$ in the minor expansion of non-diagonal entries
is $n-2$. Thus, the only contribution of degree $n-1$ comes from the diagonal
expansion along $A$.

Consider the expansion along the diagonal:
\[
(a_{11}-t)(a_{22}-t)\cdots(a_{nn}-t)
\]
is a monomial with leading coefficient 1.
We want to find the coefficient of $t^{n-1}$ in the above expansion.
Using Vieta's formula for the sum of roots of a polynomial we conclude
the sum of the roots is $(-1)(a_{11}+a_{22}+\cdots+a_{nn})$, and indeed
this is equal to the coefficient of $t^{n-1}$. Above, we have shown that
the diagonal expansion is the only contribution to $t^{n-1}$, and thus
\[
b_{n-1} = (-1)(a_{11}+a_{22}+\cdots+a_{nn}) = (-1)\textrm{tr}(A)
\]
As required.

\subsection{Show $b_0 = (-1)^n\textrm{det}(A)$}
Evaluate $f$ at $t=0$, as $f(0) = \textrm{det}(-A) = b_0$.
\[
\textrm{det}(-A) = (-1)^n\textrm{det}(A) = b_0
\]
As required.

\end{document}
