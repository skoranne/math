%%%%%%%%%%%%%%%%%%%%%%%%%%%%%%%%%%%%%%%%%%%%%%%%%%%%%%%%%%%%%%%%%%%%%%%%%%%%%%%%
%% MTH611 Complex Analysis HW1 TeX source
%% Author : Sandeep Koranne
%% Good luck
%%%%%%%%%%%%%%%%%%%%%%%%%%%%%%%%%%%%%%%%%%%%%%%%%%%%%%%%%%%%%%%%%%%%%%%%%%%%%%%%
\documentclass{article}[12pt]
\usepackage{mathptmx,amssymb,amsmath}
\usepackage{breqn}
\usepackage[normalem]{ulem}
\usepackage{enumerate}
%\usepackage{graphicx}
\usepackage[mathscr]{euscript}
\usepackage{enumerate}
%% \setlength{\textwidth}{16.5cm}
%% \setlength{\oddsidemargin}{-0.1cm}
%% \setlength{\evensidemargin}{-0.1cm}
%% \setlength{\textheight}{23cm}
%% \setlength{\topmargin}{-1.3cm}
\newtheorem{lem}{Lemma}
\newtheorem{proof}{Proof}
% Some handy shortcuts. 
\def\ge{\geqslant}
\def\le{\leqslant}
\def\phi{\varphi}
\def\to{\rightarrow}
\def\mapsto{\longmapsto}
\def\la{\langle}
\def\ra{\rangle}
\def\Aut{\operatorname{Aut}}
\def\diam{\operatorname{diam}}
\def\Image{\operatorname{Image}}
\def\Ker{\operatorname{Ker}}
\def\GL{\operatorname{GL}}
\def\SL{\operatorname{SL}}
\def\Perm{\operatorname{Perm}}
\def\nor{\vartriangleleft}
\def\nnor{\vartriangleright}
\def\lcm{\operatorname{lcm}}
\def\gcd{\operatorname{gcd}}
\def\li{\operatorname{li}}
\def\min{\operatorname{min}}
\def\max{\operatorname{max}}
\renewcommand{\mod}{\,\operatorname{mod}\,}
\newcommand{\norm}[1]{\left\lVert#1\right\rVert}

\def\CC{\mathbb C}
\def\FF{\mathbb F}
\def\NN{\mathbb N}
\def\QQ{\mathbb Q}
\def\RR{\mathbb R}
\def\ZZ{\mathbb Z}

\pagestyle{myheadings}
\begin{document}

\title{Complex Analysis Spring 2016 Homework 1}
\author{Sandeep Koranne}
\date{}
\maketitle

\section{Problem 1}
\subsection{Problem 1(a)}
Prove that
\[
1-\left|\frac{z-w}{1-z\bar{w}}\right|^2 = 
\frac{(1-|z|^2)(1-|w|^2)}{|1-\bar{z}w|^2}
\]
provided $1-\bar{z}w\ne 0$.
\begin{proof}
Expand $\left|\frac{z-w}{1-z\bar{w}}\right|^2$ using the identity
$|z|^2=z\bar{z}$, to get:
\[
\left|\frac{z-w}{1-z\bar{w}}\right|^2 = 
\frac{z-w}{1-z\bar{w}}\overline{\left(\frac{z-w}{1-z\bar{w}}\right)} =
\frac{z-w}{1-z\bar{w}}\frac{\overline{z-w}}{\overline{1-z\bar{w}}} =
\frac{z-w}{1-z\bar{w}}\frac{\bar{z}-\bar{w}}{1-\bar{z}w} = 
\frac{|z|^2-z\bar{w}-w\bar{z}+|w|^2}{1-\bar{z}w-z\bar{w}+|z|^2|w|^2}
\]
Thus the left-hand side of the expression to be proved can now be written as:
\[
\frac{1-\bar{z}w-z\bar{w}+|z|^2|w|^2-(|z|^2-z\bar{w}-w\bar{z}+|w|^2)}
{1-\bar{z}w-z\bar{w}+|z|^2|w|^2}
\]
Simplifying the numerator:
\[
\frac{1+|z|^2|w|^2-|z|^2-|w|^2}
{1-\bar{z}w-z\bar{w}+|z|^2|w|^2}
\]
Also the denominator can be written as $|1-\bar{z}w|^2$, thus we get
\[
\frac{(1-|z|^2)(1-|w|^2)}{|1-\bar{z}w|^2}
\]
As required.
\end{proof}

\subsection{Problem 1(b)}
Prove that for fixed $W$ with $|w|<1$, the mapping
\begin{equation}
z\to \frac{z-w}{1-\bar{w}z} \label{eqn:1b1}
\end{equation}
maps the unit disk into the unit disk, taking the unit circle to
the unit circle.
\begin{proof}
First consider the unit circle case, then $|z|=1$, and we can write
Equation~\ref{eqn:1b1} as
\[
|z|^2=|\frac{z-w}{1-\bar{w}z}|^2 = \left(\frac{z-w}{1-\bar{w}z}\right)
\left(\frac{\bar{z}-\bar{w}}{1-\bar{z}w}\right) =
\frac{|z|^2-z\bar{w}-w\bar{z}+|w|^2}{1-w\bar{z}-\bar{w}z+|w|^2|z|^2}=1
\]
As $|z|^2=1$. Thus the unit circle is mapped to the unit circle.
For the unit disk, we have $|z|<1$, then using the relation
$A+B < 1+AB$, for $A,B < 1$, we have
\[
|z|^2+|w|^2 < 1 + |z|^2|w|^2
\]
Using this in the above equation we see that $|z|^2<1$, which
proves that the unit disk is mapped into the unit disk, as required.
\end{proof}


\section{Problem 2}
The following questions pertain to the correspondence between the unit
sphere in $\RR^3$ and the extended complex plane given by stereographic
projection.
\subsection{Problem 2(a)}
Find the mapping of the sphere to itself induced by the following two
mappings of the (extended) complex plane to itself. 

Rather than give detailed algebraic manipulations we give an 
geometric interpretation of the inversion and unit circle inversion
mapping. And moreover, we first consider the case of simple conjugation
mapping $z\to\bar{z}$. %As can be seen in Figure~\ref{fig:1}, the 
In the complex plane, the mapping is equivalent to $y=-y$, and thus
even in the Riemann sphere, the coordinates of $\Sigma_y$ only are
modified, it changes from $\frac{2y}{1+|z|^2}$ to $\frac{-2y}{1+|z|^2}$.
Thus conjugation introduces a reflection across the vertical plane 
passing through the
real axis. We shall need this below.


\subsubsection{Inversion in the Unit Circle Mapping}
Given $z\to \frac{1}{\bar{z}}$ for $z\ne 0$, with the usual mapping of
$\infty\to 0$ and $0\to\infty$.
Since $z\to \frac{1}{\bar{z}}$, we have $z\to \frac{\bar\bar{z}}{|z|^2}$.
Thus points which are inside
the unit circle are mapped outside (with the same arg), and points
which are outside are mapped inside the circle. In the Riemann sphere,
inversion in the unit circle maps to a reflection in the equatorial
plane $\CC$, as can be seen in Figure~\ref{fig:equator}.

\begin{figure}
\vspace{5cm}
\caption{Inversion Mapping}
\label{fig:equator}
\end{figure}

\subsubsection{Inversion Mapping}
Given $z\to \frac{1}{z}$ for $z\ne 0$, with the usual mapping of
$\infty\to 0$ and $0\to\infty$.
We know that inversion mappings can be thought of as composition
of inversion in unit circle with conjugation (in that order) since
$z\to \frac{1}{z}$ implies $z\to \frac{\bar{z}}{z\bar{z}}$, which is
same as $z\to\frac{\bar{z}}{|z|^2}$. Thus in the Riemann sphere
inversion mapping is a reflection across the equatorial plane $\CC$, followed
by a reflection across the vertical plane on the real axis. It is therefore
the composition of reflections across orthogonal axis, and moreover
can be combined into a single rotation of $\Sigma$ by $\pi$. It should
be noted that this rotation is performed by an unitary matrix.


\subsection{Problem 2(b)} Show that the two mappings above are 
isometries with respect to the chordal distance.
\begin{proof}
We showed above that both mappings are equivalent to unitary
transformations (reflections). We know that unitary transforms preserve
lengths, thus the mappings are isometric with respect to the Euclidean
chordal distance.
\end{proof}

\subsection{Problem 2(c)}
Show that non-zero complex numbers $z$ and $w$ correspond to 
diametrically opposed points on the sphere if and only if $z\bar{w}=-1$.
\begin{proof}
The first thing to notice is that the unit circle of $\CC$ is a great
circle on $\Sigma$, the Riemann sphere. On the unit circle, the
condition $z\bar{w}=\frac{z}{w}$, and using the polar representation
we have $cos(z_\theta-w_\theta)=-1$, where $z=e^{iz_\theta}$ and
$w=e^{iw_\theta}$, and $z_\theta,w_\theta$ are the angles. Thus we note
that $z_\theta-w_\theta = \cos^{-1}(-1)=\pi$. Thus, the points are
diametrically opposite on the unit circle, when the condition
$z\bar{w}=-1$ is met.

Since the unit circle and Riemann sphere can be rotated while
preserving distance, without loss of generality we can fix one
of the points, say $z$ to a fixed coordinate on the unit sphere.
Let $z=(1,0)\in \CC$, and in $\Sigma$, this maps to $\hat{z}=(1,0,0)\in \Sigma$.
We first show that when $z\bar{w}=-1$, then distance between 
$\hat{z}$ and $\hat{w}$ is maximal. Since the sphere has diameter 2, we
need to show that the distance is exactly 2. Let $z\bar{w}=-1$, this
implies that $\bar{w}=-1/z$. But since $z=(1,0)$ (in particular $z$ is real),
we get $w=-1$. On the sphere $\hat{w}=(-1,0,0)$ using the stereographic
projection formulas. Thus we can see that distance between $\hat{z}$ and
$\hat{w}$ is indeed 2, as required.

To show the other direction, we again fix $\hat{z}=(1,0,0)\in \Sigma$.
Since the sphere can be rotated, this is without any loss of generality.
The diameterically opposed point of $\hat{z}$ can be calculated without
problem since $\hat{z}$ lies in the equatorial plane $\CC$, and indeed
it is denoted $\hat{w}=(-1,0,0)$. It is a simple check to calculate
$w=(-1,0)$ and check that $z\bar{w}=-1$, as required.

If there were other points (say $z',w'$)
for which the chordal distance was 2, but
the condition $z'\bar{w'}=-1$ was not met,
then rotating $\Sigma$ to bring the great circle between $z',w'$ to lie
in $\CC$ we see that $z'_\theta-w'_\theta \ne \pi$, would imply that
the angle subtended between $\hat{z'}$ and $\hat{w'}$ in $\Sigma$
is not the same as the angle between $z'$ and $w'$ in $\CC$, and
this is a contradiction. Thus $z\bar{w}=-1$ is necessary and sufficient
condition for the points to be diameterically opposed on $\Sigma$.
\end{proof}
\section{Problem 3} 
Show that three distinct points $z_1,z_2,z_3$ in the complex plane
are the vertices of an equilateral triangle if and only if they
satisfy the equation:
\begin{equation}
z_1^2 + z_2^2 + z_3^2 = z_1z_2 + z_2z_3 + z_3z_1 \label{eqn:tri}
\end{equation}
To do this, first show that Equation~\ref{eqn:tri} is equivalent to
\begin{equation}
(z_1-z_2)^2 + (z_2-z_3)^2 + (z_3-z_1)^2 = 0 \label{eqn:tri2}
\end{equation}
\begin{proof}
We expand Equation~\ref{eqn:tri2} to get:
\[
2(z_1^2 + z_2^2 + z_3^2) - 2(z_1z_2 + z_2z_3 + z_3z_1)=0 \label{eqn:tri3}
\]
Thus Equation~\ref{eqn:tri2} is clearly equal to Equation~\ref{eqn:tri}.
Let the three sides of the triangle formed by the points be
represented by the complex numbers $z=z_1-z_2$, $w=z_2-z_3$, and
$\gamma=z_3-z_1$. Written in order, as they close the boundary of
a triangle we get:
\[
z+w+\gamma=0 \implies \gamma= -(z+w) \implies \gamma^2 = z^2+w^2+2zw
\]
Substituting from Equation~\ref{eqn:tri2}, we get
\begin{eqnarray}
z^2 + w^2 + \gamma^2 & = & 0  \\
z^2 + w^2 + z^2 + w^2 + 2zw & = & 0 \\
z^2 + w^2 + zw & = & 0 \\
\frac{z}{w} + \frac{w}{z} + 1 & = & 0 \label{eqn:tri4}
\end{eqnarray}
Let $x=\frac{z}{w}$, then Equation~\ref{eqn:tri4} is $x+\frac{1}{x}+1=0$,
which we rewrite as $x^2+x+1=0$. This quadratic can be solved to get
$\alpha_{1,2}=\frac{-1\pm i\sqrt{3}}{2}$. Specifically we get
$\alpha_1=e^{i\frac{2\pi}{3}}$. 
This is a cube-root of identity,
distinct from 1, since it is conjugate complex. We also note that
$|\alpha_1|=|\alpha_2|=\sqrt{\alpha_1\alpha_2}$. We compute this as:
\[
|\alpha_1| = (\frac{-1+i\sqrt{3}}{2})(\frac{-1-i\sqrt{3}}{2})
= \frac{1+3}{4} = 1
\]
Therefore $|\frac{z}{w}|=1$, and thus $|z|=|w|$.
In the above equation, we could also eliminate $z$ instead of $\gamma$, thus
by symmetry, we have $|z|=|w|=|\gamma|$. Next, we consider the angle between
$z$ and $w$. Since $\frac{z}{w}=\alpha_1$, writing it in polar complex form
we have $\arg(\frac{z}{w})=\frac{\pi}{3}$. Similarly, the angle between
$w$ and $\gamma$ is $2\frac{\pi}{3}$ and also for $z$ and $\gamma$. Thus, 
the angle between the sides of the triangle must have angle equal to 
$2\frac{\pi}{6}$. Now we have proven that the triangle has equal sides (in
fact that would have been sufficient), and also the angles are appropriate,
as required.
\end{proof}

\section{Problem 4}
Suppose that $f$ is holomorphic on an open set $U$ and non-vanishing there.
Find (and prove) formulas for $\frac{\partial |f|}{\partial z}$ and
$\frac{\partial |f|}{\partial \bar{z}}$.
\begin{proof}
We begin by noting the definition for $\frac{\partial f}{\partial z}$
and $\frac{\partial f}{\partial \bar{z}}$:
\begin{eqnarray}
\frac{\partial f}{\partial z} & = & \frac{1}{2} 
\left(\frac{\partial f}{\partial x} -
i\frac{\partial f}{\partial y}
\right) \\ \label{eqn:del1}
\frac{\partial f}{\partial \bar{z}} & = & \frac{1}{2} 
\left(\frac{\partial f}{\partial x} +
i\frac{\partial f}{\partial y}
\right) \label{eqn:del2}
\end{eqnarray}
Using the cotangent expansion formula given in the textbook we have
\[
\langle dz,\bar{\partial} \rangle = \langle d\bar{z}, \partial \rangle
\]
Using the above we have
\begin{equation}
\frac{\partial \bar{f}}{\partial z} = \frac{\partial f}{\partial \bar{z}}
\label{eqn:del_main}
\end{equation}

Next we write $|f|=\sqrt{f\bar{f}}$ and use the product chain rule
to differentiate $|f|$ with respect to $z$. We get:
\[
\frac{\partial |f|}{\partial z} = \frac{1}{2|f|} 
\frac{\partial f\bar{f}}{\partial z} = \frac{1}{2|f|} \left[
\frac{\partial f}{\partial z}\bar{f} + \frac{\partial \bar{f}}{\partial z}f
\right]
\]
We subsitute from Equation~\ref{eqn:del1} and Equation~\ref{eqn:del_main}
to get:
\[
\frac{\partial |f|}{\partial z} = \frac{1}{2|f|} \left[
\frac{\bar{f}}{2}\left( \frac{\partial f}{\partial x} 
-i \frac{\partial f}{\partial y}\right) +
\frac{f}{2}\left( \frac{\partial f}{\partial x} +i\frac{\partial f}{\partial y}
\right)
\right]
\]
Almost by symmetry we get the following:
\[
\frac{\partial |f|}{\partial \bar{z}} = \frac{1}{2|f|} \left[
\frac{\bar{f}}{2}\left( \frac{\partial f}{\partial x} 
+i \frac{\partial f}{\partial y}\right) +
\frac{f}{2}\left( \frac{\partial f}{\partial x} -i\frac{\partial f}{\partial y}
\right)
\right]
\]

\end{proof}
\section{Problem 5}
Given $f=u+iv$ is holomorphic on an open connected set $U\subseteq \CC$ and 
satisfies the given conditions, we have to prove the following.
\subsection{Prove $f$ is constant on $U$}
Since $f$ is holomorphic (analytical) on $U$, the Cauchy-Riemann
equations are satisfied, in particular, we have
$u_x=v_y$ and $v_x=-u_y$, and if $f'(z)=0$ everywhere in $U$, then
in addition we have $f'(z)=u_x+iv_x = v_y-iu_y=0$.
Thus all partials ($u_x,u_y,v_x,v_y$) are zero on $U$.
Consider $z_0=(x_0+iy_0)\in U$, then in particular $u_x(z_0)=0$.
Consider a horizontal line $L=\{x+iy_0:a<x<b\}$, for $a,b\in\RR$ and
consider the part of $L$ inside $U$. Then we can construct a function
$\phi(x)=u(x,y_0)$, which represents the traversal on the line $L$. And
on the line $L$, we have $\phi'(x)=u_x(x,y_0)=0$. Now since $u$ is a real
valued function with zero derivative, we have $u$ is constant on $L$.
Since $y_0$ was arbitrary, we can conclude that $u$ is constant on each
horizontal line in $L$. Similarly, $u$ is constant for each vertical line
in $U$. And similarly, $v$ is also constant for horizontal and vertical lines.
But $U$ is open and connected, thus every pair of points in $U$ can be
path connected by series of horizontal and vertical lin segments lying
inside $U$. Thus we conclude that $f=u+iv$ is constant inside $U$.

Next consider a Lemma:
\begin{lem}
If a function $f$ and its conjugate $\bar{f}$ are both analytical
on $U$, then $f$ is constant on $U$.
\end{lem}
\begin{proof}
Let $f(z)=u(x,y)+iv(x,y)$ and $\overline{f(z)}=u(x,y)-iv(x,y)$. 
Due to the Cauchy-Riemann
equations we have: $u_x=v_y, u_y=-v_x$ for $f$ and
$u_x=-v_y$ and $u_y=v_x$ for $\bar{f}$. Together these imply that
$u_x=u_y=v_x=v_y=0$. Thus, we can use the previous part of the
problem and conclude that $f$ is constant on $U$.
\end{proof}
\subsection{Conclude as special case $f$ is real-valued}
Using the previous lemma we see that $f$ and $\bar{f}$ are both
analytical, hence $f$ must be constant.

\subsection{If the modulus of $f$ is constant}
We can write $|f(z)|=c$, and then $f(z)\overline{f(z)}=c^2$, but
this implies that $\overline{f(z)}=\frac{c^2}{f(z)}$. Since $f(z)$
is analytical, this implies that $\overline{f(z)}$ is also analytical, and
thus again using the previous lemma, we conclude that $f$ is constant.



\end{document}  
