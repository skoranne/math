%%%%%%%%%%%%%%%%%%%%%%%%%%%%%%%%%%%%%%%%%%%%%%%%%%%%%%%%%%%%%%%%%%%%%%%%%%%%%%%%
%% File   : skoranne_complex_hw3.tex
%% Author : Sandeep Koranne
%% Purpose: HW3 for complex analysis, good luck
%%%%%%%%%%%%%%%%%%%%%%%%%%%%%%%%%%%%%%%%%%%%%%%%%%%%%%%%%%%%%%%%%%%%%%%%%%%%%%%%
\documentclass{article}[12pt]
\usepackage{amssymb}
\usepackage{amsmath}
\usepackage{amsfonts}
\usepackage{amsthm}
\usepackage{a4wide}
%\usepackage[margin=1.5in]{geometry}
\usepackage{tikz}
\newtheorem{lem}{Lemma}
\def\RR{\mathbb R}
\def\QQ{\mathbb Q}
\def\Aut{\mathrm{Aut}}
\def\Inn{\mathrm{Inn}}
\def\Out{\mathrm{Out}}
\def\CC{{\mathbb C}}
\def\ZZ{{\mathbb Z}}
\def\Fix{\mathrm{Fix}}
\def\Spec{\mathrm{Spec}}
\def\D{\mathrm{d}}
\def\nsub{{\trianglelefteq}}
\DeclareMathOperator{\sign}{sign}
\begin{document}
\title{Spring 2016 Complex Analysis HW3}
\author{Sandeep Koranne}
\date{April 20, 2016}
\maketitle


\section*{Problem 1}
Let $f$ be an entire function, and let $a_n(w)$ denote the
coefficient of $(z-w)^n$ in the power series expansion of $f$
centered at $w$. Assume that for each $w\in \CC$ there is at
least one $n$ (which may depend on $w$) such that $a_n(w)=0$. Prove
that $f$ must be a polynomial.
\begin{proof}
  It is given that $f$ is expressed as a power series around $w$
  \[
  f(z) = \sum_{n=0}^\infty a_n (z-w)^n
  \]
  and the $a_n$ are denoted $a_n(w)$ (to show dependence on $w$). We can
  also write $a_n$ as
  \begin{equation}
  a_n = \frac{f^{(n)}(w)}{n!}  \label{eqn:1a}
  \end{equation}
  Thus $a_n=0\implies f^{(n)}(w)=0$.
  Since $f$ is entire, in a small disk around $w$, $f(z)$ is given
  by the same unique power series (using the Uniqueness Theorem).
  Then using analytic continuation, we have that this power series
  representation is unique over $\CC$. Thus, if there exists $n$ such that
  $a_n(w)=0$, then $a_n(z)=0, \forall z\in \CC$. The nth-derivative $f^{(n)}(z)$ of
  $f$ is also analytic, and thus by analytic continuation argument
  similar to above, $f^{(n)}(w)=0\implies f^{(n)}(z)=0, \forall z\in \CC$.
  This implies that $f^{(n)}=0$, which in turn implies that $f^{(n+k)}=0, k\ge 0$.
  Thus $a_n(z)=0\forall z\in \CC\implies a_n=0$, and 
  using Equation~\ref{eqn:1a} and the above argument, we have
  $a_{n+k}=0, k\ge 0$. Thus $f(z)=\sum_{k=0}^{n-1} a_k(z-w)^n$, which is a polynomial.
\end{proof}
\section*{Problem 2(a)} Evaluate $\int_\gamma \frac{\bar{z}}{z-w}\,\D z$,
where $\gamma$ is the unit circle traversed once in the counterclockwise
direction, and $w$ is a point inside the unit disk.[Hint: On the unit disk,
  $|z|^2=1$.]
\begin{proof}
  We can write the integral as
  \[
  \int_\gamma \frac{\bar{z}}{z-w}\,\D z = \int_\gamma \frac{z\bar{z}}{z(z-w)}\,\D z =
  \int_\gamma \frac{1}{z(z-w)}\,\D z =
  \int_\gamma \left( \frac{1}{w(z-w)} -\frac{1}{wz}\right)\,\D z
  \]
  Now we can integrate these separately as:
  \begin{eqnarray}
    & = &\int_\gamma  \frac{1}{w(z-w)}\,\D z -\int_\gamma \frac{1}{wz} \,\D z \nonumber \\
    & = &\frac{1}{w} \int_\gamma  \frac{1}{(z-w)}\,\D z -\frac{1}{w}\int_\gamma \frac{1}{z} \,\D z \nonumber \\
   & = & \frac{2\pi i}{w} - \frac{2\pi i}{w} = 0 \label{eqn:2a}
  \end{eqnarray}
  Where Equation~\ref{eqn:2a} is based on Cauchy's integral theorem which
  specifies that $\oint_\gamma \frac{1}{z-a}\, \D z = 2\pi i$, when $a$ lies
in the circle.
\end{proof}
\section*{Problem 2(b)}Evaluate $\int_\gamma \frac{\overline{z-w}}{z-w}\,\D z$,
where $\gamma$ is the unit circle traversed once in the counterclockwise
direction, and $w$ is a point inside the unit disk.[Hint: On the unit disk,
  $|z|^2=1$.]
\begin{proof}
  We can write the integral as
  \[
  \int_\gamma \frac{\overline{z-w}}{z-w}\,\D z =
  \int_\gamma \frac{\bar{z}-\bar{w}}{z-w}\,\D z =
  \int_\gamma \frac{\bar{z}}{z-w}\,\D z -  \int_\gamma \frac{\overline{w}}{z-w}\,\D z =
  0 - \int_\gamma \frac{\overline{w}}{z-w}\,\D z 
  \]
  based on part~(a) of this problem. We can evaluate the second term as:
  \[
    \int_\gamma \frac{\overline{w}}{z-w}\,\D z = \overline{w} \int_\gamma \frac{1}{z-w}\,\D z =
    \bar{w}2\pi i
  \]
  Thus the integral is $-2\bar{w}\pi i$.
\end{proof}

\section*{Problem 3} Let $\gamma$ be a $C^1$ closed contour, and let
$g$ be a $C^1$ real valued function. Using the chain rule, the fundamental
theorem of calculus, and some manipulation, show that
$\int_\gamma \frac{\partial}{\partial z}g\,\D z$ is pure imaginary. Conclude
as a special case that if $f$ is holomorphic, then $\int_\gamma \bar{f}f'(z)\,\D z$
is pure imaginary
\begin{proof}
  %% Let $g(z)=\frac{f(z)+\overline{f(z)}}{2}$, where $f(z)$ is a complex
  %% valued function. Then the integral can be written as:
  %% \[
  %% \int_\gamma \frac{\partial}{\partial z}g\,\D z =
  %% \int_\gamma \frac{\partial}{\partial z}\frac{f(z)+\overline{f(z)}}{2}\,\D z
  %% \]
  %% Note the Real Fundamental Theorem of Calculus, which asserts that:
  %% \[
  %% \int_a^b F'(t)\,\D t = F(b) - F(a)
  %% \]
  %% The complex analogue of this would be:
  %% \[
  %% \int_\gamma F'(z)\,\D z = F(\gamma(1)) - F(\gamma(0))
  %% \]
  %% This follows from the real fundamental theorem and the chain rule.
  %% Also note that the contour integral is defined as:
  %% \begin{equation}
  %% \int_\gamma f(z)\,\D z = \int_0^1 f(\gamma(t))\gamma'(t)\, \D t \label{eqn:3a}
  %% \end{equation}
  We also know that for $g(x,y)=u(x,y)+iv(x,y)$, where $u,v$ are real
  valued functions, we have:
  \[
  \frac{\partial g}{\partial z}(z) = \frac{1}{2}\left(\frac{\partial u}
       {\partial x}(z) + \frac{\partial v}{\partial y}(z)\right) + \frac{i}{2}
       \left(\frac{\partial v}{\partial x}(z) - \frac{\partial u}{\partial y}(z)
       \right)
       \]
       Since we know $g$ is real valued, here $v=0$, and we have:
       \[
  \frac{\partial g}{\partial z}(z) = \frac{1}{2}\left(\frac{\partial u}
       {\partial x}(z) -i\frac{\partial u}{\partial y}(z)
       \right)
       \]
       In particular, $\frac{\partial g}{\partial z}$ is also real valued,
       and thus:
       \begin{equation}
         \frac{\partial g}{\partial z}(z) =
         \frac{1}{2}\frac{\partial u}{\partial x}(z) \label{eqn:3b}
       \end{equation}
       Substituting Equation~\ref{eqn:3b}, we get:
       \begin{equation}
         \int_\gamma \frac{\partial}{\partial z}g\,\D z =
         \int_\gamma  \frac{1}{2}\frac{\partial u}{\partial x}(z) \,\D z 
       \end{equation}
       It should be noted, that $u$ is real valued.
       We can use Cauchy's integral theorem, which asserts that the
       integral over $\gamma$ is equal to the same integral taken over
       an arbitrarily small circle $C$ around $z_0$.
       Now using the parameterization of the circle
       as $\gamma=e^{it}$, we can write the above equivalent integral as:
         \[
         \frac{1}{2}\int_0^{2\pi} \frac{\partial u}{\partial x}(\gamma(t))\gamma'(t)\, \D t=
         \frac{1}{2}\int_0^{2\pi} \frac{\partial u}{\partial x}(\gamma(t))e^{it}i\, \D t =
         \frac{i}{2}\int_0^{2\pi}\frac{\partial u}{\partial x}(\gamma(t))e^{it}\, \D t =
         \frac{i}{2}\int_0^{2\pi}|\frac{\partial u}{\partial x}(\gamma(t))|e^{it}\, \D t
         \]
       which is pure-imaginary, because, $u$ is a real-valued function, and
       integral of $e^{it}\,\D t$ is also real-valued from $0$ to $2\pi$.

       {\bf Conclusion}:~conclude as a special case that if $f$ is
       holomorphic, then $\int_\gamma \bar{f}f'(z)\,\D z$ is pure imaginary.
       Consider $g(z)=|f(z)|^2=(\bar{f}(z)f(z))$, which is a real valued
       function, with real valued derivative. Now consider, $\bar{f}f'$,
       \[
       \bar{f}\frac{\partial}{\partial z} f(z) =g'(z)- f(z)\bar{f}'(z)
       \]
       Since $f$ is holomorphic, we have $f(z)\bar{f}'(z)=0$. Thus,
       the term on the left is real valued, and thus
       $\int_\gamma \bar{f}f'(z)\,\D z$ can
       be written as $\int_0^{2\pi} |\bar{f}f'(\gamma(t))|\gamma'(t)\,\D t$
       which is pure imaginary, as we proved above.

       
\end{proof}

\section*{Problem 4}Let $f$ be holomorphic on the unit disk $D$, and let
$\delta$ be the diameter of the image of the unit disk $f$, that is,
$\delta=\sup_{z,w \in D}|f(z)-f(w)|$. Using that
\begin{equation}
2f'(0) = \frac{1}{2\pi i} \int_{|z|=r} \frac{f(z)-f(-z)}{z^2}\, \D z\label{eqn:4a}
\end{equation}
prove that
\[
2|f'(0)| \le \delta
\]
\begin{proof}
  We need the following result on diameteric opposite points on the
  unit disk, which we proved in HW1. In that, $\overline{w}z=-1$,
  is necessary and sufficient condition for $|z-w|$ to be maximal.
  Rearranging we have, $\overline{w}=\frac{-1}{z}$, taking conjugates,
  $w=\frac{-1}{\overline{z}}=-z$, when $|z|=1$. 
  Therefore, the above result for opposing
  points holds for Equation~\ref{eqn:4a} also, and we have:
  \begin{equation}
    \delta|_{|z|=r}=\frac{|f(z)-f(-z)|}{r^2} \label{eqn:4b}
  \end{equation}
  This is because
  on the image of the unit disk $D$, for the diameter $\delta$, the points
  $z$ and $w$ (which form the diameter), also lie on a circle of radius
  $\delta/2$.
  Next we use Cauchy's theorem:
  \begin{equation}
    C_\gamma \phi(z) = \frac{1}{2\pi i} \int_\gamma \frac{\phi(w)}{w-z}\,\D z \label{eqn:cauchy}
  \end{equation}
  Let $z=0$, and $\gamma$ be a contour $|z|=r$ around 0.
  \begin{equation}
    C_\gamma \phi(0) = \frac{1}{2\pi i} \int_\gamma \frac{\phi(w)}{w}\,\D z \label{eqn:cauchy2}
  \end{equation}
  
  Next we use Cauchy's Transform Theorem:
  \[
  \frac{\D}{\D z} C_\gamma \phi(z) = \frac{1}{2\pi i} \int_\gamma
  \frac{\phi(w)}{(w-z_0)^2}\,\D w
  \]
  Let $z_0=0$,$\gamma:=\{|z|=r\}$,
  and $\phi(z)=f(z)-f(-z)$, and substituting above we have:
  \[
  2f'(0) = \phi'(0) = 
  \frac{1}{2\pi i} \int_{|z|=r} \frac{f(z)-f(-z)}{z^2}\,\D z
  \]
  In the class it was discussed that we can assume $r\le 1$, and thus
  we have to consider $0 \le r \le 1$, and we note on $\gamma, |z|=r$.
  Thus we get by taking modulus on both sides an inequality:
  \[
  2|f'(0)| \le \frac{1}{2\pi} \int_{|z|=r} \frac{|f(z)-f(-z)|}{|z|^2}\,\D z =
  \frac{2 \pi r\delta }{2\pi r^2}  = \frac{\delta}{r}
  \]
  The result follows by letting $r$ converge to 1.

\end{proof}

\section*{Problem 5} Suppose that $f$ is $C^1$ on the region $\Omega$ and
that for every circle $C$ lying in $\Omega$ it holds that $\int_C f\,\D z=0$.
Prove that $f$ is holomorphic.
\begin{proof}
Instead of proving the existence of a single valued primitive $F$ of $f$ defined
by 
\[
F(z) = \int_{z_0}^z f(s)\,\D s
\]
where $z_0$ is some fixed point in $\Omega$ and the integral is taken over
any rectifiable curve joining $z_0$ to $z$,
we use the hint given to consider
the complex Green's theorem and look at limits of 
the average values over small disks of shrinking radius. 
Fix $z_0\in \Omega$ and then
according to the hypothesis we have
\[
0 = \int_{C(z_0,r)} f(s)\,\D s
\]

We know the real form of Green's theorem is defined as
\[
\int_\Gamma P\,\D x + Q\,\D y = \int \int_\Omega \left( 
\frac{\partial Q}{\partial x} - \frac{\partial P}{\partial y}
\right) \,\D x\, \D y
\]
where $\Omega$ is a domain with boundary $\Gamma$, and $P$ and
$Q$ are continuous functions defined on a larger set, which contains
both $\Omega$ and $\Gamma$.
Let the complex function $f(z)=u(x,y)+iv(x,y)$, then we can write
$\int_\Gamma f(s)\,\D s$. Expanding this:
\[
\int_\Gamma f(s)\,\D s = \int_\Gamma u\,\D x - v\,\D y + i\int_\Gamma v\,\D x + u\,\D y
\]
Applying the real form of Green's theorem to the two integrals on the right:
\[
= - \int \int_\Omega \left( 
\frac{\partial v}{\partial x} + \frac{\partial u}{\partial y}
\right) \,\D x\, \D y + \int \int_\Omega \left( 
\frac{\partial u}{\partial x} - \frac{\partial v}{\partial y}
\right) \,\D x\, \D y
\]
Since $f$ is holomorphic, the two integrals on the right are identically
zero, which confirms $\int_\Gamma f(s)\,\D s=0$.
The above Green's Theorem, can be written in compact notation as
follows.
The complex form of Green's theorem, we have
\begin{equation}
0 = \int_{C(z_0,r)} f(s)\,\D s = 2i\int\int_{D(z_0,r)\in C(z_0,r)} 
\frac{\partial f}{\partial \overline{z}} \,\D x\,\D y \label{eqn:50}
\end{equation}
where the differential operator for $\bar{z}$ is defined as usual by
\[
\frac{\partial f}{\partial \overline{z}} = \frac{1}{2}\left( 
\frac{\partial f}{\partial z}+i\frac{\partial f}{\partial y}\right)
\]
Dividing Equation~\ref{eqn:50} by $2i\pi r^2$ we get:
\begin{equation}
\frac{1}{\pi r^2} \int\int_{D(z_0,r)\in C(z_0,r)} 
\frac{\partial f}{\partial \overline{z}} \,\D x\,\D y = 0 \label{eqn:5a}
\end{equation}
This is the average of the continuous function $\frac{\partial f}{\partial \overline{z}}$ over the disk $D(z_0,r)$, and is equal to zero.
Essentially, this completes the argument, as by letting $r\to 0$ we have
\[
\lim_{r\to 0}\ \frac{\partial f}{\partial \overline{z}}(z_0)=0
\]
Since this holds for each point $z_0\in \Omega$, we have
$\frac{\partial f}{\partial \overline{z}}=0$ identically in $\Omega$.
We know this is identical to Cauchy-Riemann condition, and demonstrates
that $f$ is holomorphic in $\Omega$. Since $f\in C^1$, it is sufficient
to show that $\frac{\partial f}{\partial \overline{z}}$ is identically zero
on a dense set, and for each point of $\Omega$ we have met this criteria
by using a disk shrinking (with the limit of $r\to 0$) to it (the point).
%[This proof of Morera's theorem 
%is based in part on the 1974 paper by Lawrence Zalcman, ``Real
%Proofs of Complex Theorems''].
\end{proof}

\end{document}
