%%%%%%%%%%%%%%%%%%%%%%%%%%%%%%%%%%%%%%%%%%%%%%%%%%%%%%%%%%%%%%%%%%%%%%%%%%%%%%%%
%% File   : skoranne_complex_hw8.tex
%% Author : Sandeep Koranne
%% Purpose: HW8 for complex analysis, good luck
%%%%%%%%%%%%%%%%%%%%%%%%%%%%%%%%%%%%%%%%%%%%%%%%%%%%%%%%%%%%%%%%%%%%%%%%%%%%%%%%
\documentclass{article}[12pt]
\usepackage{amssymb}
\usepackage{amsmath}
\usepackage{amsfonts}
\usepackage{amsthm}
\usepackage{a4wide}
\usepackage{polynom}
\usepackage{tikz}

%\usepackage[margin=1.5in]{geometry}
\newtheorem{lem}{Lemma}
\def\RR{\mathbb R}
\def\QQ{\mathbb Q}
\def\DD{\mathbb D}
\def\Aut{\mathrm{Aut}}
\def\Inn{\mathrm{Inn}}
\def\Out{\mathrm{Out}}
\def\CC{{\mathbb C}}
\def\NN{{\mathbb N}}
\def\ZZ{{\mathbb Z}}
\def\Fix{\mathrm{Fix}}
\def\Spec{\mathrm{Spec}}
\def\D{\mathrm{d}}
\def\nsub{{\trianglelefteq}}
\DeclareMathOperator{\sign}{sign}
\begin{document}
\title{Spring 2016 Complex Analysis HW8}
\author{Sandeep Koranne}
\date{June 3, 2016}
\maketitle


\section*{Problem 1}Using the linear fractional transform and powers,
construct a conformal map from the upper half disk to the upper
half plane.
\begin{proof}
The map $f(z)\to z^2$ maps the first quadrant to the upper half plane
with $\mbox{Im}z>0$. Thus if we can map the upper half disk to the
first quadrant, and compose with $z^2$, we will be done.

Since we know that the map $\frac{1+z}{1-z}$ maps the unit disk to the
right half plan, the behaviour of this map on the semi-disk can be
inferred as below.
The M\"obius map:
\[
f(z) = \frac{1+z}{1-z}
\]
maps the upper half disk to the first quadrant, with $-1<\mbox{Re}z<1$ mapping
to the positive $X$ axis, and the image of the semicircular disk
mapping to the positive $Y$ axis. Therefore the required transform
to construct a conformal map from the upper half disk to the upper
half plane is:
\[
f(z) = \left(\frac{1+z}{1-z}\right)^2
\]
\end{proof}

\section*{Problem 2}
Let $\Gamma$ a circle contained inside the unit circle with center
$\sigma$ and radius $\rho$. The aim of this problem is to construct
a M\"obius transformation $T$ which takes the unit circle to itself
and $\Gamma$ to a circle centered at the origin. By a rotation in $z$
we may suppose that $\sigma$ satisfies $0\le \sigma < 1$. The transformation
sought must have the form $z\to \gamma\frac{a-z}{1-\overline{a}z}$ for
some $a$ of modulus less than 1 and $\gamma$ with $|\gamma|=1$.
Since the $\gamma$ factor maps circles centered at origin to
themselves, it may be omitted.

\subsection*{Problem 2(a)}Using the symmetry of $\sigma$
and $\infty$ with respect to $\Gamma$, and properties of M\"obius
maps, derive an equation relating $a,\sigma$ and the radius $R$,
of the image circle which shows that $a$ must be real.
\begin{proof}
Let $\Gamma'$ denote the transformed $\Gamma$ with radius $R$.
We know that for the circle $\Gamma'$, we have $\Gamma'=\{z:|z|=R\}$,
and the symmetry condition for $\Gamma'$ gives the map
\[
S(z) = \left\{ \begin{array}{cc}
                \frac{R^2}{\overline{z}} & \mbox{if}\ z\ne 0 \\
                \infty & \mbox{if}\ z=0 \\
                0 & \mbox{if}\ z=\infty \end{array}\right.
\]
From the properties of M\"obius maps, we know that these maps
transform a pair of symmetric points with respect to $\Gamma$ into
a pair of symmetric points with respect to $\Gamma'$. Thus $\sigma$
and $\infty$ were symmetric with respect to $\Gamma$, thus they
remain symmetric with respect to $\Gamma'$ defined through
$z\to \frac{a-z}{1-\overline{a}z}$. We know that $\sigma$ is mapped
to the origin, substituting above we get $a=\sigma$, and thus $S(a)$
is defined as $R^2/\bar{a}$, but symmetric lines have same arg, thus
$S(a)$ is real, $R^2$ is real, therefore $\bar{a}$ has to be real,
therefore $a$ is real.
\end{proof}

\subsection*{Problem 2(b)}Show that that $T$ must take the points
$1,a,-1$ to $-1,0,1$ respectively, and so the images of
$\sigma \pm \rho$ are $\pm R$.
\begin{proof}
By direct substitution in the map, we see that $T(a)=\frac{a-a}{1-\bar{a}a}=0$.
Similarly, for $T(1)=\frac{a-1}{1-\bar{a}}=-1$ (since $a$ is real).
Similarly, for $T(-1)=\frac{a+1}{1+\bar{a}}=1$ (since $a$ is real).
Therefore, since $\rho<1$, and $T(\sigma)=0$, the images of
$\sigma\pm\rho$ are $0\pm R$.
\end{proof}

\subsection*{Problem 2(c)}Use this to solve for $a$ in terms of
$\sigma$ and $\rho$
\begin{proof}
We have shown that $T(\sigma)=0$, thus $a=\sigma$. The scaling
can be provided by $\rho$ using $R$.
\end{proof}


\section*{Problem 3} Suppose that $f$ is holomorphic in $D$
(the unir disk), injective, $f(0)=0$, $|f'(0)|\le 1$,
and $D\subseteq f(D)$. Prove that $f(z)=cz$ for some
$c$ of modulus one.
\begin{proof}
Using problem 2 above we know there exist a M\"obius map which
maps the unit circle to itself. Consider the M\"obius map
\[
M(z) = \frac{z_1-z}{1-\bar{z_1}z}
\]
and the associated function  for a fixed $z_1$
\[
\phi(z)=\frac{f(z_1)-z}{1-\overline{f(z_1)}z}
\]
Since $M(z_1)=0$, and the map is invertible, $f$ is injective, we have
the composed map $\phi(f(M^{-1}(z)))$ maps 0 to 0, and $D$ to itself
and moreover $|\phi(f(M^{-1}(z)))|\le |z|$.
Thus we can apply Schwarz-Pick Lemma, and have $f(z)=\phi^{-1} \circ I_c
\circ \phi \in \mbox{Aut}(D)$.
But this implies that $|c|=1$, and thus $f(z)=cz$ with $|c|=1$.
\end{proof}



\section*{Problem 4}Let $C$ be a circle in $\CC$ and
$w_1,w_2$ two (distinct) points in the complement of $C$.
Prove that there is a fractional linear transformation which takes
$C$ to itself and takes $w_1$ to $w_2$.
\begin{proof}
Without loss of generality (or by appropriate translation) assume that
$C$ is centered at the origin. Since both $w_1$ and $w_2$ are in the complement
we have $|w_1|>R$ and $|w_2|>R$, where $R$ is the radius of $C$.
Perform a scaling transformation such that $K_1=|w_1|/R$.
Now $w_1$ is on the circle of radius $K_1R$, and we construct
a M\"obius map such that $0,\infty,w_1$ are mapped to $0,\infty,K_1w_2$.
Thereafter we construct the reverse scaling map $1/K_1$.
Composing these maps 
\[
(1/K_1) \circ (w,0,\infty,K_1w_2)=(z,0,\infty,w_1) \circ K_1
\]
gives a map which takes $C$ to itself, and $w_1$ to $w_2$, where
the notation $(w,0,\infty,K_1w_2)=(z,0,\infty,w_1)$ is a FLT map.
\end{proof}

\section*{Problem 5}Suppose that $f$ is analytic and non-constant on the
unit disk $D$ and satisfies $\mbox{Re}\ f(z)\ge 0$ there.
By the open mapping theorem then actually 
$\mbox{Re}\ f(z)> 0$ on the disk.


\subsection*{Problem 5(a)} By using an appropriate M\"obius map and
the Schwarz lemma, prove that if $f(0) = 1$, then
\[
|f(z)| \le \frac{1+|z|}{1-|z|}
\]
on $D$. What holds if $f(0)\ne 1$ ?
\begin{proof}
We note that $\mbox{Re}\ f(z) > 0$ on the unit disk by the 
open mapping theorem. Consider the M\"obius map 
\[
\phi(w) = \frac{w-1}{w+1}
\]
Then $\phi$ maps the positive half plane conformally onto the
unit disk, and therefore $g=\phi \circ f$ maps $D$ conformally
onto itself with $g(0)=0$. Now using Schwarz lemma we have
\[
|\phi(f(z))|=|g(z)| \le |z|
\] 
for all $z\in D$.
Moreover, 
\[
|\phi'(f(0))f'(0)| = |g'(0)| \le 1
\]
Substituting with the definition of $\phi$ we have
\begin{equation}
\frac{|f(z)|-1}{|f(z)|+1} \le |\frac{f(z)-1}{f(z)+1}| \le |z| \label{eqn:5a}
\end{equation}
Simplifying we have
\[
\frac{|f(z)|-1}{|f(z)|+1} \le |z| \implies 
\frac{-2}{|f(z)|+1} \le |z|-1 \implies 
\frac{2}{|f(z)|+1} \ge 1-|z| \implies 
\frac{2}{1-|z|} \ge 1+|f(z)| \implies 
|f(z)| \le \frac{1+|z|}{1-|z|}
\]
If $f(0)\ne 1$, we still have
\[
|\phi'(f(0))f'(0)| = |g'(0)| \le 1
\]
from Schwarz lemma.
\end{proof}

\subsection*{Problem 5(b)}Show also that if $f(0)=1$ then f also
satisfies
\[
|f(z)| \ge \frac{1-|z|}{1+|z|}
\]
\begin{proof}
Consider Equation~\ref{eqn:5a}, we simplify it with 
\[
\frac{1-|f(z)|}{1+|f(z)|} \le |z|
\]
\[
\frac{2}{1+|f(z)|} \le |z|+1 \implies 1+|f(z)| \ge \frac{2}{1+|z|}
\]
which implies
\[
|f(z)| \ge \frac{1-|z|}{1+|z|}
\]
\end{proof}

\end{document}
