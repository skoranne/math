%%%%%%%%%%%%%%%%%%%%%%%%%%%%%%%%%%%%%%%%%%%%%%%%%%%%%%%%%%%%%%%%%%%%%%%%%%%%%%%%
%% File   : skoranne_complex_hw3.tex
%% Author : Sandeep Koranne
%% Purpose: HW4 for complex analysis, good luck
%%%%%%%%%%%%%%%%%%%%%%%%%%%%%%%%%%%%%%%%%%%%%%%%%%%%%%%%%%%%%%%%%%%%%%%%%%%%%%%%
\documentclass{article}[12pt]
\usepackage{amssymb}
\usepackage{amsmath}
\usepackage{amsfonts}
\usepackage{amsthm}
\usepackage{a4wide}
%\usepackage[margin=1.5in]{geometry}
\usepackage{tikz}
\newtheorem{lem}{Lemma}
\def\RR{\mathbb R}
\def\QQ{\mathbb Q}
\def\Aut{\mathrm{Aut}}
\def\Inn{\mathrm{Inn}}
\def\Out{\mathrm{Out}}
\def\CC{{\mathbb C}}
\def\NN{{\mathbb N}}
\def\ZZ{{\mathbb Z}}
\def\Fix{\mathrm{Fix}}
\def\Spec{\mathrm{Spec}}
\def\D{\mathrm{d}}
\def\nsub{{\trianglelefteq}}
\DeclareMathOperator{\sign}{sign}
\begin{document}
\title{Spring 2016 Complex Analysis HW4}
\author{Sandeep Koranne}
\date{April 27, 2016}
\maketitle


\section*{Problem 1}
Use Cauchy's integral formula to evaluate the following contour integrals:

\subsection*{Problem 1(a)}
\[
\frac{1}{2\pi i} \int_{C_r} \frac{e^z}{z^2+iz+6}\,\D z
\]
where $C_r$ is the circle of radius $r$ centered at the origin oriented
counterclockwise, for both $r=1$ and $r=4$.
\begin{proof}
Using partial fraction decomposition we can write:
\[
\frac{1}{z^2+iz+6} = \frac{i}{5(z+3i)} - \frac{i}{5(z-2i)}
\]
thus the above integral can be written as:
\[
\frac{1}{2\pi i} \int_{C_r} \frac{e^z}{z^2+iz+6}\,\D z = 
\frac{1}{2\pi i} \int_{C_r} \frac{ie^z}{5(z+3i)}\,\D z -
\frac{1}{2\pi i} \int_{C_r} \frac{ie^z}{5(z-2i)} \,\D z
\]
Now we consider $r=1$, in this disk, $z+3i$ does not have any zeros,
and neither does $z-2i$, thus the integral is zero (using Cauchy's integral
theorem), since the functions are analytic, without singularities, and
contour is closed.

For $r=4$, the contour now encloses both the singularities (at $2i$ and
$-3i$) and thus using Cauchy's integral formula, we have
\[
\frac{1}{2\pi i} \int_{C_4} \frac{e^z}{z^2+iz+6}\,\D z = 
\frac{1}{2\pi i} \int_{C_4} \frac{ie^z}{5(z+3i)}\,\D z -
\frac{1}{2\pi i} \int_{C_4} \frac{ie^z}{5(z-2i)} \,\D z =
f(-3i)-f(2i)
\]
where $f(z)=\frac{ie^z}{5}$, substituting, we 
have $I=\frac{i}{5}[e^{-3i}-e^{2i}]$.
\end{proof}

\subsection*{Problem 1(b)}
\[
\frac{1}{2\pi i} \int_{C_r} \frac{e^z}{(z-i)^4} dz
\]
where $C_r$ is the circle of radius $r$ centered at the origin oriented
counterclockwise, for both $r=2$.

\begin{proof}
We first note that with $r=2$, the disk does not pass through
any singularity of the integrand function. And moreover, the disk
contains the obvious singularity at $z=i$.
Using Cauchy's differentiation formula we have
\begin{equation}
f^{(n)}(a) = \frac{n!}{2\pi i} \int_C \frac{f(z)}{(z-a)^{n+1}}\, \D z \label{eqn:1b}
\end{equation}
Using Equation~\ref{eqn:1b}, with $n=3$, we have the integral equal to
\[
I = \frac{f^{(3)}e^z}{3!}|_{z=i} = \frac{-ie^i}{6}
\]
\end{proof}

\section*{Problem 2} Prove that $\sum_{n=1}^\infty \frac{1}{z^2+n^2}$ is
holomorphic on $\CC\setminus \pm i\NN$, by proving that
the partial sums converge uniformly on compact subsets.
\begin{proof}
We begin by noting that on the domain $\CC\setminus \pm i\NN$, the 
general term of the series $\frac{1}{z^2+n^2}$ is well defined and
bounded on the domain (since $\frac{1}{n^2}$ is bounded). By the
limit comparison test (LCT), we have
\[
\lim_{n\to\infty} \frac{|1/n^2|}{|1/(z^2+n^2)|} = \frac{|z^2+n^2|}{|n^2|}
= 1 + |(\frac{z}{n})^2| = 1
\]
thus the series $1/n^2$ and $1/(z^2+n^2)$, have the same disposition with
respect to convergence. But we know that the series $1/n^2$ converges, thus
we can conclude that the series $1/(z^2+n^2)$ also converges
uniformly and absolutely. Therefore for all $\epsilon>0$ there exists
$N(\epsilon)$ such that
\begin{equation}
\sum_{N(\epsilon)}^\infty \frac{1}{|(z_0)^2+n^2|} < \epsilon \label{eqn:2a}
\end{equation}

On any compact subset $A\subseteq \CC\setminus \pm\NN$, there exists
$z_0$ such that $|z_0|$ is minimal. This follows from the maximum modulus
principle, which has a corrolary to this effect. Therefore, on $A$, we have
$|z|\ge |z_0|$ for all $z\in A$. Now using Equation~\ref{eqn:2a} we have
\[
\sum_{N(\epsilon)}^\infty \frac{1}{|z^2+n^2|} < \sum_{N(\epsilon)}^\infty \frac{1}{|(z_0)^2+n^2|} < \epsilon \label{eqn:2b}
\]
Thus, the series converges on compact subsets, thus $f(z)$ represented
by this series represents a holomorphic function on this domain.
\end{proof}


\section*{Problem 3}
Let $\Omega$ be a region in $\CC$ and let $f$ be a holomorphic mapping
from $\Omega$ to itself. 
Let $f_1=f, f_2=f\circ f, f_2 = f\circ f_1,\ldots,f_{j+1}=f\circ f_j$ for
all $j\in \NN$, and suppose that the sequence $f_j$ converges
uniformly on compact subsets of $\Omega$. Let $g=\lim f_j$.

\subsection*{Problem 3(a)}
Prove that $g\circ g=g$.
\begin{proof}
Since $g=\lim f_j$, and it is given that the sequence of functions 
converges, therefore it is a Cauchy sequence using the uniform convergence
metric. Thus for all $\epsilon>0$, there exists $N(\epsilon)$ such that
$|g-f_j|<\epsilon, j>N(\epsilon)$, and in particular this is true for
all $z\in\Omega$ (by uniform convergence). This implies that
$|g\circ g - f_i \circ f_j| < \epsilon$, for $i,j > N(\epsilon)$.
Now let $\lim i,j\to\infty$, we have $|g\circ g-g|<\epsilon$.
This is in the uniform convergence norm, and now expanding $g$ using
power series around $z_0$ we can show 
that $|(g\circ g)(z_0)-g(z_0)|<\epsilon/n$.
Since $\epsilon$ is arbitrary, we have $g\circ g=g$.
\end{proof}

\subsection*{Problem 3(b)}
Prove then that either $g=\mathrm{constant}$ or $g=\mathrm{id}$.
\begin{proof}
We write $g(z)$ as a power series around $z_0$:
\[
g(z) = g(z_0) + a_1(z-z_0)+a_2(z-z_0)^2 + \cdots + 
\]
Composing the above with $g$ we have $(g\circ g)(z)$
\[
(g\circ g)(z) = g(z_0) + a_1( g(z_0) + a_1(z-z_0)+a_2(z-z_0)^2 + \cdots -z_0 ) 
+ a_2(g(z_0) + a_1(z-z_0)+a_2(z-z_0)^2 + \cdots -z_0)^2 + \cdots
\]
but we have shown above that $(g\circ g)(z)=g(z)$. Thus
\begin{eqnarray}
g(z_0) + a_1(z-z_0)+a_2(z-z_0)^2 + \cdots + & = & g(z_0) + a_1( g(z_0) + a_1(z-z_0) +a_2(z-z_0)^2 + \cdots -z_0 ) \nonumber \\  
& & + a_2(g(z_0) + a_1(z-z_0)+a_2(z-z_0)^2 + \cdots -z_0)^2 + \cdots \nonumber
\end{eqnarray}
Comparing the left hand side series and right hand series for powers of
$(z-z_0)$ we have that either $a_1=a_2=\cdots=a_n=0$, and then 
$g(z)=g(z_0)=\mathrm{constant}$. Or, $g(z_0)=0$, and of all the coefficients
$a_1=$, and all other $a_j=0,j>1$.
These correspond to the conditions that either $g(z)$ is a constant
function or $g(z)=z$, the identity function.
\end{proof}

\section*{Problem 4} Let $\Omega$ be a bounded region, $f$
a holomorphic mapping from $\Omega$ to itself, and assume that
there exists $a\in\Omega$ which is a fixed point for $f$.
Prove that $|f'(a)|\le 1$.
\begin{proof}
Using the hint, we first show that for $h(z)\in H(\Omega)$ 
(holomorphic on $\Omega$), we have $|h'(a)|\le M$, where $M$
depends on $\Omega$. If $\sup |f(z)| \le M_1$
for all $z \in \Omega$, and this is true, since $f:\Omega\to\Omega$, and
$\Omega$ is a bounded domain, then using Cauchy estimate we have:
\begin{eqnarray}
f^{(n)}(a) & = & \frac{n!}{2\pi i}\int_\Omega \frac{f(z)}{(z-a)^{n+1}}\, \D z \nonumber \\
|f^{(n)}(a)| & = & \frac{n!}{2\pi }|\int_\Omega \frac{f(z)}{(z-a)^{n+1}}\, \D z |\nonumber \\
& \le & \frac{n!}{2\pi }\int_\Omega \frac{|f(z)|}{(|z-a|)^{n+1}}\, \D z |\nonumber \\
& \le & \frac{n!M_1}{2\pi} |\int_\Omega \frac{1}{(|z-a|)^{n+1}}\, \D z |\nonumber \\
& \le & \frac{n!M_1}{2\pi} |\int_\Omega \frac{1}{(R)^{n+1}}\, \D z |\nonumber \\ 
& \le & \frac{n!M_1}{2\pi R^{n+1}} |\int_\Omega \, \D z |\nonumber \\ 
& \le & \frac{n!M_1}{R^{n+1}} \nonumber \\ 
\end{eqnarray}
where $R$ denotes the distance between $a$ and $\Omega$. Therefore we now have
a bound on $|f^{(n)}(a)|$ dependent only on the region $\Omega$.

Next consider the identity for the nth-derivative of $f$ at a fixed point $a$
\[
f^{(n)}(a) = (f'(a))^n
\]
this can be proven using the chain rule and the fact that $f(a)=a$.
Using the chain rule 
\[
(f\circ f)'(a) = (f'\circ f(a))\cdot f'(a)=f'(a)\cdot f'(a) = (f'(a))^2
\]
Using induction, we have
\[
(f_k)^{(k)}(a) = (f'(a))^k
\]
composing and differentiating once more, we have
\[
(f\circ f_k)^{(k+1)}(a) = f'\circ (f'(a))^k f'(a) = (f'(a))^{k+1}
\]
Thus, we have $f^{(n)}(a)=(f'(a))^n$.

Now, since on $\Omega$ we have shown that $f^{(n)}(a) \le M$, this
implies that $(f'(a))^n \le M$, and from geometric power series, we know
that this convergence (for arbitrary $n$) 
happens only when $|f'(a)|\le 1$, thus we have
the result.
\end{proof}

\section*{Problem 5} Continuing with the hypothesis of the
preceding problem, assume now that $f'(a)=1$. Consider the
Taylor expansion of $f$ at $a$, $f(z)=a+(z-a)+b_n(z-a)^n + o((z-a)^n)$
where $b_n$ denotes the first non-zero coefficient of index greater than one.
Prove that the Taylor expansion of $f_k$ is $f_k(z)=a+(z-a)+kb_n(z-a)^n + o(z^n)$
and conclude, using the Cauchy estimates, that in fact $b_n=0$ and that
$f$ is the identity map.
\begin{proof}
We write the Taylor series as given:
\[
f(z) = a + (z-a) + b_n(z-a)^n + o((z-a)^n)
\]
composing with $f$ we have
\[
(f\circ f)(z) = a + ( a + (z-a) + b_n(z-a)^n + o((z-a)^n) -a) + b_n( a + (z-a) + b_n(z-a)^n + o((z-a)^n) -a)^n + o(z^n)
\]
Simplifying the above:
\[
(f\circ f)(z) = a + (z-a) + b_n(z-a)^n + b_n(z-a)^n + o(z^n)
\]
Using induction, let the hypothesis be true for $k$, then
\[
f_k(z) = a + (z-a) + kb_n(z-a)^n + o(z^n)
\]
composing again with $f$, we have:
\[
(f \circ f_k)(z) = f_{k+1}(z) = a + (a + (z-a) + kb_n(z-a)^n + o(z^n)-a) +
kb_n(a + (z-a) + kb_n(z-a)^n + o(z^n) -a)^n + o(z^n)
\]
Simplifying:
\begin{eqnarray}
f_{k+1}(z) & = & a + (z-a) + kb_n(z-a)^n + b_n(z-a)^n + o(z^n) \nonumber \\
          & = & a + (z-a) + (k+1)b_n(z-a)^n + o(z^n)
\end{eqnarray}
Thus we have verified the expansion for all $\NN$.

Using the Cauchy estimate we have:
\[
f^{(n)}_k(z) = n!kb_m + n!\cdot o(z)
\]
At $z=a$, the Cauchy estimate for $f^{(n)}(a)$ is bounded,
we have $n!kb_n\le M$, this implies that $b_n=0$, and that $f$
is given as $f(z)=z$, the identity map.
\end{proof}


\end{document}
