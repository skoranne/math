%%%%%%%%%%%%%%%%%%%%%%%%%%%%%%%%%%%%%%%%%%%%%%%%%%%%%%%%%%%%%%%%%%%%%%%%%%%%%%%%
%% MTH611 Complex Analysis HW2B TeX source
%% Author : Sandeep Koranne
%% Good luck
%%%%%%%%%%%%%%%%%%%%%%%%%%%%%%%%%%%%%%%%%%%%%%%%%%%%%%%%%%%%%%%%%%%%%%%%%%%%%%%%
\documentclass{article}[12pt]
\usepackage{mathptmx,amssymb,amsmath}
\usepackage{breqn}
\usepackage[normalem]{ulem}
\usepackage{enumerate}
%\usepackage{graphicx}
\usepackage[mathscr]{euscript}
\usepackage{enumerate}
%% \setlength{\textwidth}{16.5cm}
%% \setlength{\oddsidemargin}{-0.1cm}
%% \setlength{\evensidemargin}{-0.1cm}
%% \setlength{\textheight}{23cm}
%% \setlength{\topmargin}{-1.3cm}
\newtheorem{lem}{Lemma}
\newtheorem{proof}{Proof}
% Some handy shortcuts. 
\def\ge{\geqslant}
\def\le{\leqslant}
\def\phi{\varphi}
\def\to{\rightarrow}
\def\mapsto{\longmapsto}
\def\la{\langle}
\def\ra{\rangle}
\def\Aut{\operatorname{Aut}}
\def\diam{\operatorname{diam}}
\def\Image{\operatorname{Image}}
\def\Ker{\operatorname{Ker}}
\def\GL{\operatorname{GL}}
\def\SL{\operatorname{SL}}
\def\Perm{\operatorname{Perm}}
\def\nor{\vartriangleleft}
\def\nnor{\vartriangleright}
\def\lcm{\operatorname{lcm}}
\def\gcd{\operatorname{gcd}}
\def\li{\operatorname{li}}
\def\min{\operatorname{min}}
\def\max{\operatorname{max}}
\renewcommand{\mod}{\,\operatorname{mod}\,}
\newcommand{\norm}[1]{\left\lVert#1\right\rVert}

\def\CC{\mathbb C}
\def\FF{\mathbb F}
\def\NN{\mathbb N}
\def\QQ{\mathbb Q}
\def\RR{\mathbb R}
\def\ZZ{\mathbb Z}

\pagestyle{myheadings}
\begin{document}

\title{Complex Analysis Spring 2016 Homework 1}
\author{Sandeep Koranne}
\date{}
%\maketitle

%\section{Problem 4}
\subsection*{Problem 4(a)}
Suppose that for each $n$, $\sum_k a_k^{(n)}z^k$ converges on $D(0,r)$ and let
$f_n(z)$ denote the sum. Suppose also that $\sum_n f_n(z)$ converges uniformly
on compact subsets of $D(0,r)$, and let $F(z)$ denote the sum.
\subsection{Problem 4(a)}
Supposing that $F(z)$ has a convergent power series expansion on
$D(0,r): F(z) = \sum b_kz^k$. Prove that $b_k=\sum_n a_k^{(n)}$.
\begin{proof}
Since it is given that $f_n(z)$ denotes the sum of a convergent
power series, we can calculate the coefficients using a Taylor expansion
at $(z=0)$,
to get:
\begin{equation}
a^{(n)}_k = \frac{1}{k!} \frac{d^k}{dz^k}f_n(z) \label{eqn:4a1}
\end{equation}
It is also given that $F(z)$ is a convergent power series
\[
F(z) = \sum_n f_n(z)
\]
which has an expansion using $b_k$ as coefficients, we can expand them
using Taylor series as well, to get:
\begin{equation}
b_k = \frac{1}{k!} \frac{d^k}{dz^k}F(z) \label{eqn:4a2}
\end{equation}
Using the definition of $F(z)$ and $f_n(z)$ 
in Equation~\ref{eqn:4a2}, we have:
\begin{equation}
b_k = \frac{1}{k!} \frac{d^k}{dz^k}\sum_{k=0}^\infty \sum_{n=0}^\infty a^{(n)}_kz^k
\label{eqn:4a3}
\end{equation}
Since the series for $F(z)$ on the right is uniformly convergent, 
we can interchange the order of summation to get:
\[
b_k = \frac{1}{k!} \frac{d^k}{dz^k} \sum_{n=0}^\infty  \sum_{k=0}^\infty a^{(n)}_kz^k
\]
We can move the first summation outside the differentiation:
\[
b_k = \sum_{n=0}^\infty \frac{1}{k!} \frac{d^k}{dz^k} \sum_{k=0}^\infty a^{(n)}_kz^k
\]
Thereafter, we can carry out the term-by-term differentiation inside the
first summation, which can be rearranged to get:
\[
b_k = \sum_{n=0}^\infty \frac{1}{k!} \frac{d^k}{dz^k} f_n(z)
\]
But using Equation~\ref{eqn:4a1} we can simplify this to:
\[
b_k = \sum_{n=0}^\infty a^{(n)}_k
\]
As required.
\end{proof}

\subsection*{Problem 4(b)}Show that $\sum \frac{z^n}{1-z^n}$ converges
uniformly on compact subsets of the unit disk, and determine the
coefficients of its power series representation (centered at $z=0$).
What is the number-theoretic interpretation of the coefficients?
\begin{proof}
Let us denote the original series by $L=\sum \frac{z^n}{1-z^n}$.
Obviously, for every $z$ such that $1-z^n$ is zero, $L$ is
not defined, thus we exclude the boundary (circumference) of the
unit disk, even though $L$ may converge at $z^n\ne \pm 1$ on the
circumference, these are excluded as the boundary is not a compact
subset.

Consider the series $\sum z^n$ which is convergent only for $|z|<1$,
and is equivalent to 
\[
\sum z^n = \sum \frac{z^n}{1-z^n} (1-z^n)
\]
Using this series we can write the $L$ as a series of the product
of two terms $z^n$ and $\frac{1}{1-z^n}$.
Using Abel's partial sum test, we have to establish convergence of
\[
\sum |(1-z^{n+1})-(1-z^n)| = |1-z|\sum |z^n|
\]
which is convergent on the unit disk.
The second series to be verified is
\[
\sum |\frac{1}{1-z^{n+1}} - \frac{1}{1-z^n}| = |1-z|\sum 
\frac{|z^n|}{|(1-z^{n+1})(1-z^n)}|
\]
For large $n$, we have $|1-z^n|>\frac{1}{2}$, thus the above series
is also convergent. Now we have verified the Abel's test, and thus
$L$ is convergent.

To show uniform convergence, consider the fact that all terms of $L$
are analytic functions regular in $|z|<1$, and as such for every $\rho < 1$, 
the series is uniformly convergent in $|z|<\rho$, which includes all
compact subsets of the unit disk.

\subsubsection{Coefficients of $L$ as a power series}
Note that we have
\[
\frac{1}{1-z} = 1 + z + z^2 + \cdots
\]
for $|z|<1$. For $z^n$, we have
\[
\frac{1}{1-z^n} = 1 + z^n + z^{2n} + \cdots
\]
for $|z^n|<1$.
It is instructive to look at the first few terms of $L$, as $z^n(1-z^n)^{-1}$:
\begin{eqnarray}
\frac{z}{1-z} & = & z + z^2 + z^3 + z^4 + \cdots + \nonumber \\
\frac{z^2}{1-z^2} & = & \quad + z^2 + \quad + z^4 + \quad + z^6 + \cdots +\nonumber \\
\frac{z^3}{1-z^3} & = & \quad \quad \quad z^3 + \quad \quad \quad \quad z^6 + \cdots + \nonumber \\
\end{eqnarray}
We can add these term by term.
In the $k$th row, a given power of $z^n$ will occur if and only if $n$ is
a multiple of $k$.
Formally,
\[
L = \sum \frac{z^n}{1-z^n} = \sum_{m=1}^\infty z^m \sum_{n=1}^\infty z^n =  
\sum_{m=1}^\infty\sum_{n=1}^\infty z^{mn} = \sum_{m=1}^\infty 1 \sum_{n:m|n}^\infty z^n
\]
Since we have shown the series is convergent, we can interchange the summation
order to get:
\[
L = \sum_{n=1}^\infty z^n \sum_{m|n}^\infty 1 = \sum_{n=1}^\infty \mbox{d}(n) z^n
\]
where $d(n)$ is the function which is known as the divisor function in
number theory. Given $n$, it returns the number of divisors of $n$.
Note that such series are called Lambert series.
\end{proof}

\end{document}  
