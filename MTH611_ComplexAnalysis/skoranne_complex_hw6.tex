%%%%%%%%%%%%%%%%%%%%%%%%%%%%%%%%%%%%%%%%%%%%%%%%%%%%%%%%%%%%%%%%%%%%%%%%%%%%%%%%
%% File   : skoranne_complex_hw3.tex
%% Author : Sandeep Koranne
%% Purpose: HW6 for complex analysis, good luck
%%%%%%%%%%%%%%%%%%%%%%%%%%%%%%%%%%%%%%%%%%%%%%%%%%%%%%%%%%%%%%%%%%%%%%%%%%%%%%%%
\documentclass{article}[12pt]
\usepackage{amssymb}
\usepackage{amsmath}
\usepackage{amsfonts}
\usepackage{amsthm}
\usepackage{a4wide}
\usepackage{polynom}

%\usepackage[margin=1.5in]{geometry}
\newtheorem{lem}{Lemma}
\def\RR{\mathbb R}
\def\QQ{\mathbb Q}
\def\Aut{\mathrm{Aut}}
\def\Inn{\mathrm{Inn}}
\def\Out{\mathrm{Out}}
\def\CC{{\mathbb C}}
\def\NN{{\mathbb N}}
\def\ZZ{{\mathbb Z}}
\def\Fix{\mathrm{Fix}}
\def\Spec{\mathrm{Spec}}
\def\D{\mathrm{d}}
\def\nsub{{\trianglelefteq}}
\DeclareMathOperator{\sign}{sign}
\begin{document}
\title{Spring 2016 Complex Analysis HW6}
\author{Sandeep Koranne}
\date{May 13, 2016}
\maketitle


\section*{Problem 1}
Compute the three non-zero lowest order terms Laurent expansion for the following functions at the given point $z_0$.

\begin{enumerate}
\item{}$f(z)=\frac{z}{(z+1)^3}$, $z_0=-1$
\begin{proof}
Let $z+1=u$, then $z=u-1, f(z)=\frac{u-1}{u^3}$, which can be
written as
\[
\frac{1}{u^2}-\frac{1}{u^3} = \frac{1}{(z+1)^2} - \frac{1}{(z+1)^3}
\]
\end{proof}
\item{}$f(z)=\frac{z}{(z-1)(z-3)(z-4)}$, $z_0=3$
\begin{proof}
Let $z-3=u$, then $z=u+3$, and $f(z)$ is as
\[
f(z) = \frac{u+3}{(u+2)u(u-1)} = -\frac{u+3}{u\cdot 2(1+u/2)\cdot (1-u)}=
-\frac{(u+3)(1+u+u^2+u^3+\cdots+)}{u\cdot 2(1- (-u/2))} 
\]
\[
-\frac{(u+3)(1+u+u^2+u^3+\cdots+)(1-u/2+u^2/4-u^3/8+\cdots+)}{2u}
\]
Now we can collect terms of $u$
\[
-\frac{3}{2u} -\frac{5}{4} - \frac{11}{8}u + \cdots+
\]
Now substitute back $u=z-3$ to get the terms of the Laurent expansion
\[
-\frac{3}{2(z-3)} - \frac{5}{4} - \frac{11}{8}(z-3)
\]
\end{proof}
\item{}$f(z)=\coth(z)$, $z_0=0$
\begin{proof}
We begin by noting
$\coth(z)=\frac{\cosh(z)}{\sinh(z)}$, which can be expanded using
series expansion as
\[
\coth(z) = \frac{1+1/2z^2+1/24z^4+1/720z^6+1/40320z^8+O(z^{10})}
{z+z^3/6+z^5/120+z^7/5040+O(z^{9})}
\]
%\polylongdiv{1+1/2x^2+1/24x^4+1/720x^6}{x+1/6x^3+1/120x^5+1/5040x^7}
\[
\polylongdiv{1+1/2*x^2+1/24*x^4+1/720*x^6+1/40320*x^8}{x+1/6*x^3+1/120*x^5}
\]
But this does not account for the infinite terms, thus, doing
calculation by hand (doing long division)
we get the Laurent expansion of $\coth(z)$ around
$z_0=0$ as
\[
\frac{1}{z}+\frac{z}{3}-\frac{z^3}{45}
\]
NB: there is a formula for this series using the Bernoulli numbers.
\end{proof}

\end{enumerate}

\section*{Problem 2}Compute the residue at the given point $z_0$
\begin{enumerate}
\item{}$f(z)=\frac{z^2+1}{z(z+3)^2}$, $z_0=-3$
\begin{proof}
The function has a pole of order 2 at $z_0=-3$, thus we compute the
residue as
\[
\lim_{z\to -3} \frac{d}{dz}\left( (z+3)^2 f(z)\right)=
\lim_{z\to -3} \frac{d}{dz}\left(z+\frac{1}{z}\right)=
\lim_{z\to -3}  1 - \frac{1}{z^2} = 8/9
\]
\end{proof}

\item{}$f(z)=\frac{e^z}{(z-i-1)^2}$, $z_0=1+i$
\begin{proof}
Function has a simple pole of second order, thus using the
Residue formula, we get $\frac{d}{dz}e^z=e^z$, at $z_0$, thus
the residue is $e^{1+i}$.
\end{proof}


\item{}$f(z)=\frac{\sin z}{z^3(z-2)(z+1)}$, $z_0=0$
\begin{proof}
Function has pole of order 3, thus using the Residue formula
we have
\[
\lim_{z\to 0} \frac{1}{2} \frac{d^2}{dz^2} \left( z^3 f(z)\right) =
\lim_{z\to 0} \frac{1}{2} \frac{d^2}{dz^2} \frac{\sin z}{(z-2)(z+1)}
\]
\[
\frac{1}{2}\lim_{z\to 0} -\frac{2\cos z}{(z-2)^2(z+1)} - \frac{2\cos z}{(z-2)(z+1)^2}=\frac{1}{2}(\frac{-2}{4}+1)=\frac{1}{4}
\]
\end{proof}


\item{}$f(z)=exp(z+z^{-1})$, $z_0=0$
\begin{proof}
The function has an essential singularity at $z=0$.
We cannot write the Laurent expansion of $f(z)$ directly, so
we need to use the Cauchy formula to compute the coefficients.
\[
f(z) = exp(z+z^{-1}) = \sum_{n=-\infty}^\infty a_nz^n
\]
And we know $a_n=\frac{1}{2\pi i}\int_C \frac{f(s)}{s^{n+1}}ds$.
Let us use the unit circle as the contour of integration, then
$|z|=1$, and we use the parameterization $z=e^{i\theta}$, and thus
calculate $a_n$ as
\[
a_n = \int_0^{2\pi} \frac{e^{e^{i\theta}+e^{-i\theta}}}{e^{i(n+1)\theta}}ie^{i\theta}d\theta
\]
We know that $\cos z = \frac{e^{iz}+e^{-iz}}{2}$, thus the numerator
of the integral can be simplified using this identity as
\[
a_n = i \int_0^{2\pi} e^{2\cos \theta-n\theta}d\theta
\]
Since we are only interested in the residue, let us put $n=-1$
and simplify even further to get
\[
a_{-1} = i\int_0^{2\pi} e^{2\cos \theta + \theta}d\theta
\]
%Let $y=2\cos\theta + \theta$, then $\frac{dy}{d\theta}=-2\sin\theta+1$.
%Substituting above
%\[
%\int \frac{e^y}{1-2\sin\theta}dy 
%\]
This integral should be zero, thus $a_{-1}=0$, and the residue is zero.
\end{proof}

\end{enumerate}

\section*{Problem 3 (Simon p.~131, Problem 1)}
Let $f(z)=P(z)/Q(z)$ be a rational function. Show that it has
finitely many poles.
\begin{proof}
We are assuming from the notation that $P(z)$ and $Q(z)$ are
polynomials with no common factor 
(as thats what a rational function for complex is
understood to be), therefore we can write the function $P(z)$
as
\[
P(z)=a_0(z-\alpha_1)(z-\alpha_2)\ldots(z-\alpha_m)
\]
where $P(z)$ has degree $m$. Similarly for $Q(z)$ we write it as
\[
Q(z)=b_0(z-\beta_1)(z-\beta_2)\ldots(z-\beta_n)
\]
where $Q$ has degree $n$, and moreover $\alpha_i\ne\beta_j$ (since
$P$ and $Q$ have no common factor).
Therefore, $Q$ has finitely many zeros, and these are the poles
of $f$. We can therefore write $R(z)$ as
\begin{equation}
f(z) = \frac{a_0(z-\alpha_1)(z-\alpha_2)\ldots(z-\alpha_m)}
{b_0(z-\beta_1)(z-\beta_2)\ldots(z-\beta_n)}z^{m-n} \label{eqn:3a}
\end{equation}
We consider the three cases corresponding to $m>n$, $m<n$ and $m=n$,
and in each case we can easily conclude that the number of poles
of $f(z)$ is finite.
\end{proof}

\subsection*{Problem 3(b)}
Let $p_1,p_2,\ldots,p_l$ be the poles and $p_1(z),p_2(z),\ldots,p_l(z)$
be the principal parts. Show that
\[
f(z) = \sum_{j=1}^l p_j(z) + R(z)
\]
where $R$ is a polynomial of degree exactly $\mbox{deg}(P)-\mbox{deg}(Q)$
(with the convention that $R=0$, if this difference is negative).
In particular, if $\mbox{deg}(P) < \mbox{deg}(Q)$, then 
$f=\sum_{j=1}^l p_j$.
\begin{proof}
Based on the Hint, we use Theorem 3.1.9, which states that
if $f$ is an entire function and for some $\alpha \ge 0$ and
$C<\infty$, we have that for all $z\in\CC$, $|f(z)| \le C(|z|+1)^\alpha$,
then $f$ is a polynomial of degree at most $[\alpha]$, the largest
integer less than $\alpha$.
As we have written $f(z)$ in Equation~\ref{eqn:3a}, we see that when
$m\le n$, then $R(z)=0$. If $m>n$, then $f(z)$ has a pole at $\infty$,
and using the division algorithm we can write $f(z)=\sum_{j=1}^l p_j(z)+R(z)$
and the degree of $R(z)$ has to be less than $n$ and moreover
$R(z)$ is finite at $\infty$.
\end{proof}

\section*{Problem 4 (Simon p.~215, Problem 1 c,e,f)}
\subsection*{Problem (c)}Evaluate
\[
\int_0^{2\pi} \frac{\cos \theta}{a+\cos \theta} dx\qquad \mbox{for}\ a>1
\]
\begin{proof}
Looks like a trick question or misprint, as the measure is $dx$, so
$\frac{\cos \theta}{a+\cos\theta}$ are independent of $x$, therefore
can be moved out of the intergral, to get 
\[
\frac{\cos \theta}{a+\cos\theta} \int_0^{2\pi} dx = 2\pi\frac{\cos \theta}{a+\cos\theta}
\]
\end{proof}

\subsection*{Problem (e)}Evaluate
\[
\int_{-\pi}^{\pi} \frac{x\sin x}{1-2a\cos x+a^2} dx\qquad \mbox{for}\ 0<a<1
\]
\begin{proof}

\end{proof}

\subsection*{Problem (f)}Evaluate
\[
\int_{0}^{2\pi} \frac{\cos mx}{1-2a\cos x+a^2} dx\qquad \mbox{for}\ 0<a<1
\]
\begin{proof}

\end{proof}



\section*{Problem 3 (Simon p.~131, Problem 5 a,b,d)}
Evaluate the following assuming
\[
0 < \alpha < \beta, 0 < \alpha = \beta, \mbox{and}\ 0 < \beta < \alpha
\]
\subsection*{Problem (a)}Evaluate
\[
\int_{-\infty}^{\infty} \frac{\sin \alpha x\cos \beta x}{x} dx
\]
\begin{proof}

\end{proof}

\subsection*{Problem (b)}Evaluate
\[
\int_{-\infty}^{\infty} \frac{\sin^2 x}{x^2} dx
\]
\begin{proof}

\end{proof}

\subsection*{Problem (d)}Evaluate
\[
\int_{-\infty}^{\infty} \frac{\sin x}{x(x^2-\pi^2)} dx
\]
\begin{proof}

\end{proof}






\end{document}
