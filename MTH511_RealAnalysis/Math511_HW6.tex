\documentclass{article}
\usepackage{mathptmx,amssymb,amsmath}
\usepackage{amsthm}
\usepackage{breqn}
\usepackage[normalem]{ulem}
\usepackage{enumerate}
\usepackage[mathscr]{euscript}
\setlength{\textwidth}{16.5cm}
\setlength{\oddsidemargin}{-0.1cm}
\setlength{\evensidemargin}{-0.1cm}
\setlength{\textheight}{23cm}
\setlength{\topmargin}{-1.3cm}
\newtheorem{lem}{Lemma}
% Some handy shortcuts. 
\def\ge{\geqslant}
\def\le{\leqslant}
\def\phi{\varphi}
\def\to{\rightarrow}
\def\mapsto{\longmapsto}
\def\la{\langle}
\def\ra{\rangle}
\def\Aut{\operatorname{Aut}}
\def\diam{\operatorname{diam}}
\def\Image{\operatorname{Image}}
\def\Ker{\operatorname{Ker}}
\def\GL{\operatorname{GL}}
\def\SL{\operatorname{SL}}
\def\Perm{\operatorname{Perm}}
\def\nor{\vartriangleleft}
\def\nnor{\vartriangleright}
\def\lcm{\operatorname{lcm}}
\def\gcd{\operatorname{gcd}}
\def\li{\operatorname{li}}
\def\min{\operatorname{min}}
\def\max{\operatorname{max}}
\renewcommand{\mod}{\,\operatorname{mod}\,}
\newcommand{\norm}[1]{\left\lVert#1\right\rVert}

\def\CC{\mathbb C}
\def\FF{\mathbb F}
\def\NN{\mathbb N}
\def\QQ{\mathbb Q}
\def\RR{\mathbb R}
\def\ZZ{\mathbb Z}

\pagestyle{myheadings}
\begin{document}

% Replace Name. This is for the headers on all but the first page.
\markright{\hspace{10pt} Math 511 - Fall 2015 \hspace{100pt} Homework 6 - Sandeep Koranne }
\thispagestyle{empty}

\textbf{Math 511 - Fall 2015 \hfill Real Analysis  \hfill Instructor: Ossiander}

\hrulefill 
\medskip 

% Replace Name. This is for the first page header.
 {Sandeep Koranne \hfill  Homework 6 \hfill Due: November 16, 2015}
 \begin{center}
Ch 4(3, 5, 19, 33, 48) \quad Ch 5(17, 20, 25, 36) \quad Ch 7(5, 10)\\ \textit{Optional:  Ch 4(6, 7, 14, 34, 46, 58) \quad Ch 5(30, 42) \quad Ch 7(9)}
 \end{center}
\medskip

\begin{enumerate}

%%%%%%%%%%%%%%%%%%%%%%%%%%%%%%%%%%%%%%%%%%%%%%%%%%%%%%

\item (Car. 4.3) Some authors say that two metrics $d$ and $\rho$ on a set $M$ are equivalent if they generate the same open sets. Prove this. (Recall that we have defined equivalence to mean that $d$ and $\rho$ generate the same convergent sequences. See Exercise 3.42.)

\textbf{Solution:} We have to prove this using the definition of equivalence of convergent sequences.
We first need a Lemma.
\begin{lem}
Equivalent metrics generate the same closed sets.
\end{lem}
\begin{proof}
The metrics $d$ and $\rho$ generate the same convergent sequences. In particular if $(x_n) \rightarrow x$ for
$d$ then $(x_n) \rightarrow x$ for $\rho$. Consider the subset $X \subseteq M$ and $(x_n) \subseteq X$, then
closure of $X$ under $d$ is $X+L_d$ where $L_d$ denotes the limit points under $d$. Similarly, closure of
$X$ under $\rho$ is $X+L_\rho$ where $L_\rho$ denotes the limit points under $\rho$. But by our definition
of metric equivalence $L_d = L_\rho$. Thus closure of $X$ is identical under $d$ and $\rho$, thus the
closed sets generated by $d$ and $\rho$ are same.
\end{proof}
To prove the equivalence condition for open sets, we only need to note that open sets are the complement of
closed sets. Thus if $d$ and $\rho$ generate same closed sets, they generate the same open sets.
%%%%%%%%%%%%%%%%%%%%%%%%%%%%%%%%%%%%%%%%%%%%%%%%%%%%%%

\item (Car. 4.5) Let $f: \RR \to \RR$ be continuous. Show that $\{ x :\ f(x) > 0 \}$ is an open subset of $\RR$ and that $\{x:\ f(x) = 0 \}$ is a closed subset of $\RR$.

\textbf{Solution:} Since $f$ is continuous then $\forall \epsilon > 0$ we know $\exists \delta > 0$ such that
$f(B_\delta(x)) \subset B_\epsilon(f(x))$. Now $f(x) > 0$ is open (since if $f(x) > 0$, then $f(x)+\epsilon > 0$.
Then we can use Theorem 5.1(iv) of the textbook to conclude $\{ x :\ f(x) > 0 \}$ is also open.
Similarly, we know that for $f(x)=0$ the right hand side is a closed set (any finite set is closed), thus
using Theorem 5.1(iii) we can conclude $\{x:\ f(x) = 0 \}$ is a closed subset of $\RR$.

%%%%%%%%%%%%%%%%%%%%%%%%%%%%%%%%%%%%%%%%%%%%%%%%%%%%%%

\item (Car. 4.19) Show that $\diam \left (A \right ) = \diam \left (\bar{A} \right )$.

\textbf{Solution:} 
We know $A \subseteq \bar{A}$. Thus 
\begin{equation}
\diam \left (A \right ) \le \diam \left(\bar{A}\right ) \label{eqn:dia1}
\end{equation}
Let $x,y \in A$ and $x',y' \in \bar{A}$ such that $d(x,x') < \epsilon/2$
and $d(y,y') < \epsilon/2$. Using the triangle inequality and the definition
of diameter on $A$ we have
\[
d(x',y') \le d(x,x') + d(x,y) + d(y,y') < \epsilon/2 + \diam \left(A\right) + \epsilon/2
\]
We know:
\[
\diam\left(\bar{A}\right) = \sup_{x',y' \in \bar{A}} d(x',y') 
                          \le \epsilon + \diam\left(A\right)
\]
Since $\epsilon > 0$ is arbitrary, we have $\diam\left({\bar{A}}\right) 
\le \diam\left(A\right)$.
Using Eqn~\ref{eqn:dia1} we get $\diam\left(A\right) = \diam\left({\bar{A}}\right)$.

%%%%%%%%%%%%%%%%%%%%%%%%%%%%%%%%%%%%%%%%%%%%%%%%%%%%%%

\item (Car. 4.33) Let $A$ be a subset of $M$. A point $x \in M$ is called a \textbf{limit point} of $A$ if every neighborhood of $x$ contains a point of $A$ that is different from $x$ itself, that is, if $(B_{\epsilon}(x) \backslash \{ x \} ) \cap A \neq \emptyset$ for every $\epsilon > 0$. If $x$ is a limit point of $A$, show that every neighborhood of $x$ contains infinitely many point of $A$.

\textbf{Solution:} We need the definition of the closure of $A$ as the set of all \emph{contact points} of $A$.
A point $x \in M$ is a  contact point of $A$ if $(B_{\epsilon}(x)) \cap A \neq \emptyset$. Thus every limit point is a contact
point (but the converse is not true). Next we need a Lemma.
\begin{lem} A necessary condition for a point $x$ to be a contact point of $A$ is that there exist a
sequence $(x_n) \subset A $ converging to $x$.
\end{lem}
\begin{proof}
Consider the sequence of nested balls $B_{1/n}(x)$. Since $x$ is a contact point each ball contains a point of $A$, 
let that point be $x_n$, and since the radius shrinks to zero, the sequence converges to $x$.
\end{proof}
Indeed, if $x$ is in addition a limit point, then at every step of the above Lemma, the neighborhood $B_{1/n}(x) \cap A$
will contain points distinct from $x$ (by definition of limit point). And as the sequence is infinite, we conclude
that every neighborhood of $x$ contains infinitely many points of $A$.
%%%%%%%%%%%%%%%%%%%%%%%%%%%%%%%%%%%%%%%%%%%%%%%%%%%%%%

\item (Car. 4.48) A metric space is called \textbf{seperable} if it contains a countable dense subset. Find example of countable dense sets in $\RR$, in $\RR^2$, and in $\RR^n$.

\textbf{Solution:} The example of countable dense set in $\RR$ is $\QQ$. The set of all points $x=(x_1,x_2)$ with
rational coordinates is countable dense in $\RR^2$, and similarly, the set of all points $x=(x_1,x_2,\ldots,x_n)$ with
rational coordinates is countable dense in $\RR^n$.

%%%%%%%%%%%%%%%%%%%%%%%%%%%%%%%%%%%%%%%%%%%%%%%%%%%%%%

\item (Car. 5.17) Let $f,g:(M, d) \to (N, \rho)$ be continuous, and let $D$ be a dense subset of $M$. If $f(x) = g(x)$ for all $x \in D$, show that $f(x) = g(x)$ for all $x \in M$. If $f$ is onto, show that $f(D)$ is dense in $N$.

\textbf{Solution:} Proof by contradiction. Assume that $\exists x \in M$ such that $f(x) \ne g(x)$. We can
construct open neighborhoods of $Q_1=B(f(x))\in N$ and $Q_2=B(g(x))\in N$ such that $Q_1 \cap Q_2 = \emptyset$.
Note that such neighborhoods can always be constructed since $(N,\rho)$ is a metric space.
By continuity, there exist open neighborhoods $P_1=B(x)$ and $P_2=B(x)$ such $f(P_1) \subseteq Q_1$
and  $f(P_2) \subseteq Q_2$. In particular, since $f(x) \ne g(x), x\in M$, $(P_1 \cap P_2) \cap D = \emptyset$, as
by definition $f(x) = g(x), \forall x \in D$. But this contradicts the definition of $D$ dense in $M$. Thus
we have $f(x) = g(x)$ for all $x \in M$.

If $f$ is onto then for all $y\in N, \exists x \in M$ such that $y=f(x)$. 
%Since $D$ is dense in $M$, $f(D)$ is dense in $N$. 
Consider $x_1 \in D$ and $y_1 = f(x_1)$.
Let  $Q=B_\epsilon(y)$, and if $Q \cap y_1 = \emptyset$, then by continuity we have open neighborhood in $M$ such that
$P=f(B(x)) \subseteq Q, x\in M$, and moreover $P \cap x_1 = \emptyset$. But by definition $D$ is dense in $M$, thus
this is a contradiction. Thus $f(D)$ is dense in $N$.

%%%%%%%%%%%%%%%%%%%%%%%%%%%%%%%%%%%%%%%%%%%%%%%%%%%%%%

\item (Car. 5.20) If $d$ a metric on $M$, show that $|d(x, z) - d(y, z)| \le d(x,y)$ and conclude that the function $f(x) = d(x, z)$ is continuous on $M$ for any fixed $z \in M$. This says that $d(x,y)$ is \textit{separately continuous} -  continuous in each variable separately.

\textbf{Solution:} Consider 
\[
d(x,z) \le d(x,y) + d(y,z)
\]
Thus
\[
d(x,z)-d(y,z) \le d(x,y) 
\]
Similarly
\[
d(y,z) \le d(y,x) + d(x,z)
\]
\[
d(y,z) - d(x,z) \le d(x,y)
\]
Combining the above we get
\[
|d(x,z) - d(y,z)| \le d(x,y)
\]
Consider $f(x)=d(x,z)$. To show $f$ is continuous on $M$ for fixed $z$, we have to show $\forall \epsilon > 0$, 
$\exists \delta > 0$ such that when $d(x,y)\le\delta$ we have $|f(x)-f(y)|\le \epsilon$. 
Using the definition of $f$ we have to show
\[
|d(x,z)-d(y,z)| \le \epsilon
\]
But we have shown above that $|d(x,z)-d(y,z)| \le d(x,y)$, thus choosing $\delta = \epsilon$, we have
\[
d(x,y) \le \epsilon \implies |d(x,z)-d(y,z)| \le \epsilon
\]
Thus $f$ is continuous on $M$ for fixed $z$, as required.

%%%%%%%%%%%%%%%%%%%%%%%%%%%%%%%%%%%%%%%%%%%%%%%%%%%%%%

\item (Car. 5.25) A function $f : (M, d) \to (N, \rho)$ is called \textbf{Lipschitz} if there is a constant $K < \infty$ such that $\rho(f(x), f(y)) \le Kd(x,y)$ for all $x,y \in M$. Prove that a Lipschitz mapping is continuous.

\textbf{Solution:} To show that the Lipschitz mapping is continuous we have to show that $\forall \epsilon > 0, \ 
\exists \delta > 0$ such that $d(x,y) \le \delta \implies \rho(f(x), f(y)) \le \epsilon$. Note that the choice of
$\delta$ can depend on $f$ as well as $\epsilon$ (and $x,y$). In particular, choose $\delta$ such that
$\delta \le \epsilon/K$.
Then using the condtion of Lipschitz mapping and the choice of $\delta$ we have
\[
\rho(f(x),f(y)) \le Kd(x,y) \le K\frac{\epsilon}{K} \le \epsilon
\]
As required.

%%%%%%%%%%%%%%%%%%%%%%%%%%%%%%%%%%%%%%%%%%%%%%%%%%%%%%

\item (Car. 5.36) Suppose that we are given a point $x$ and a sequence $(x_n)$ in a metric space $M$, and suppose that $f(x_n) \to f(x)$ for every continuous, real-valued function $f$ on $M$. Does it follow that $x_n \to x$ in $M$? Explain.

\textbf{Solution:} In the previous problem we showed that Lipschitz mappings are continuous, thus consider a choice
of $\delta$ such that
\[
d(x_n,x) \le \delta \implies |(f(x_n)-f(x)| \le \epsilon
\]
We can invert the relationship between $\epsilon$ and $\delta$ using a suitable Lipschitz constant $K'$.
Specifically, we can choose $K'$ such that
\[
\frac{1}{K'}d(x_n,x) \le |f(x_n)-f(x)|
\] for all continuous functions $f$. We can construct a sequence of $K'$ (one for each function) and choose the
infemum.
Since we are given that $f$ is continuous, the open neighborhood around $f(x)$ contains infinitely many 
points of $f(x_n)$, and correspondingly there is an open neighborhood in $M$ such that around $x$ there
are infinitely many points of $x_n$ (using Theorem 5.1(iv)). Thus $x_n \to x$ in $M$.

%%%%%%%%%%%%%%%%%%%%%%%%%%%%%%%%%%%%%%%%%%%%%%%%%%%%%%

\item (Car. 7.5) Prove that $A$ is totally bounded if and only if $\bar{A}$ is totally bounded.

\textbf{Solution:} We first show that if $\bar{A}$ is totally bounded then this implies that $A$ is
also totally bounded. By the definition of totally boundedness, $\bar{A}$ can be covered by a finite
collection of balls $B_1,B_2,\ldots,B_n$ of radius $\epsilon$ centered at $x_1,x_2,\ldots,x_n$.
Moreover, by the definition of closure, $A \subseteq \bar{A}$, thus these balls also cover all of $A$; thus $A$ is also totally bounded.

Next, we show that if $\bar{A}$ is not totally bounded, then $A$ is also not totally bounded. 
Note that this is equivalent to showing that the closure of a totally bounded set is totally bounded.
Let $B_i$ be
a collection of balls of radius $\epsilon/2$ such that $A = \cup B_{\epsilon/2}(x_i)$. Moreover, by the
definition of closure, if $z \in \bar{A}$, then $\exists x \in A$ such that $d(z,x) \le \epsilon/2$ (as the
$\epsilon/2$ neighborhood of $z$ must contain a point from $A$). But, we also have that $d(x,x_i)$ for some $i$
is less than $\epsilon/2$ (as $A$ is covered by $\cup B_{\epsilon/2}(x_i)$. Using the triangle inequality
\[
d(z,x_i) \le d(z,x) + d(x,x_i) \le \epsilon/2 + \epsilon/2 = \epsilon
\]
As required. Thus $\bar{A}$ is also covered by $\cup B_{\epsilon/2}(x_i)$ and is totally bounded.
%%%%%%%%%%%%%%%%%%%%%%%%%%%%%%%%%%%%%%%%%%%%%%%%%%%%%%

\item (Car. 7.10) Prove that a totally bounded metric space $M$ is separable. [Hint: For each $n$, let $D_n$ be a fininte $(1/n)$-net for $M$. Show that $D = \bigcup_{n=1}^{\infty} D_n$ is a countable dense set.]

\textbf{Solution:} Since $M$ is totally bounded we can find a finite $\epsilon$-net for $M$. 
Let $D_n$ be  finite $(1/n)$-net for $M$. Specifically,
\[
M \subseteq \cup_{x \in D_n} B_{1/n}(x)
\]
Let $D=\cup_{n=1}^{\infty} D_n$. Since the countable union of countable sets is countable, we have $D$ is countable.
Next we show that $D$ is dense in $M$. Consider $x\in M$. Let $U(x) \in M$ be an open neighborhood such that
$x\in U(x)$. By definition, there is a $n$ such that $B_{1/n}(x) \subseteq U$. Similarly, there is a member $i$ of
the net $D_n$ such that $x \in B_{1/n}(D_i)$. Thus, 
\[
B_{1/n}(x) \cap B_{1/n}(D_i) \ne \emptyset
\]
for some $i, 1 \le i \le n$. Thus $D$ is dense in $M$ as required.


%%%%%%%%%%%%%%%%%%%%%%%%%%%%%%%%%%%%%%%%%%%%%%%%%%%%%%

\begin{center}
\hrulefill  \quad \large{\textit{Optional Exercises}} \quad \hrulefill 
\end{center}

%%%%%%%%%%%%%%%%%%%%%%%%%%%%%%%%%%%%%%%%%%%%%%%%%%%%%%

\item (Car. 4.6 -optional) Give an example of an infinite closed set in $\RR$ containing only irrationals. Is there an open set consisting entirely of irrationals?

\textbf{Solution:} Consider the set consisting of $\frac{n}{\sqrt{2}}$. This set is is infinite, irrational and as it
consists of isolated points, is closed. There is no open set consisting entirely of rationals, as $\QQ$ is dense in
$\RR$ and thus also dense in $\RR \setminus \QQ$.

%%%%%%%%%%%%%%%%%%%%%%%%%%%%%%%%%%%%%%%%%%%%%%%%%%%%%%

\item (Car. 4.7 -optional) Show that every open set in $\RR$ is the union of (countably many) open intervals with \textit{rational} endpoints. Use this to show that the collection $\mathscr{U}$ of all open subsets of $\RR$ has the same cardinality as $\RR$ itself.

\textbf{Solution:} Consider $E$ as an open set in $\RR$. If $E=\emptyset$, then there is nothing to prove. So we
assume $E\ne\emptyset$. Consider $x\in E$. Let $u=\inf\left\{a \in E|(a,x) \subseteq E\right\}$ and let 
$v=\sup\left\{a \in E|(x,a) \subseteq E\right\}$. Then $E$ is the union of all such segment $(u,v)$. Since
$\QQ$ are dense in $\RR$, each segment must contain atleast one rational, and since the set of rationals is
countable, the set of segments is countable.

Next, we show that each point $x\in\RR$ can be associated with two rationals $q$ and $r_q$ such that
$x\in (q-r_q,q+r_q)$. Above we have shown that each open set is the union of countable open intervals, and each
interval is associated with a rational (let that be $q$) and the length of the interval containing $x$ be
$r_q$. Thus there is a bijective map between $2^{|\QQ\times\QQ|}$ and $\RR$, as required (since $2^{|\QQ\times\QQ|}$
has the same cardinality as $2^{\aleph_0}$ which is the cardinality of $\RR$).


%%%%%%%%%%%%%%%%%%%%%%%%%%%%%%%%%%%%%%%%%%%%%%%%%%%%%%

\item (Car. 4.14 -optional) Show that the set $A = \{ x \in \ell_2 : |x_n| \le 1/n,\ n = 1, 2, \dots \}$ is a closed set in $\ell_2$ but that $B = \{ x \in \ell_2 : |x_n| < 1/n,\ n = 1, 2, \dots \}$ is \textit{not} an open set. [Hint: Does $B \supset B_{\epsilon}(0)$?]

\textbf{Solution:} The set $A$ consists of points $x$ which converge in $\ell_2$ and $|x_n| \le 1/n$, the limit
point of this sequence is 0 which is contained in $A$, thus $A$ is closed.
As the hint specifies, for $B$ infinitely many points will have $|x_n|=0$ which is not less than  $1/n$ (as 
limit of $1/n$ is zero), thus the neighborhood
of the limit will not be open (as it does not contain infinitely many points of the sequence).

%%%%%%%%%%%%%%%%%%%%%%%%%%%%%%%%%%%%%%%%%%%%%%%%%%%%%%

\item (Car. 4.34 -optional) Show that $x$ is a limit point of $A$ if and only if there is a sequence $(x_n)$ in $A$ such that $x_n \to x$ and $x_n \neq x$ for all $n$.

\textbf{Solution:} We have proved this using Lemma 2 in problem 4 above.

%%%%%%%%%%%%%%%%%%%%%%%%%%%%%%%%%%%%%%%%%%%%%%%%%%%%%%

\item (Car. 4.46 -optional) A set $A$ is said to be \textbf{dense} in $M$ (or, as some authors say, \textit{everywhere dense} if $\bar{A} = M$. For example, both $\QQ$ and $\RR \backslash \QQ$ are dense in $\RR$. Show that $A$ is dense in $M$ if and only if any of the following hold:
\begin{enumerate}
\item Every point in $M$ is the limit of a sequence from $A$.
\item $B_{\epsilon}(x) \cap A \neq \emptyset$ for every $x \in M$ and every $\epsilon > 0$.
\item $U \cap A \neq \emptyset$ for every nonempty open set $U$.
\item $A^c$ has empty interior.
\end{enumerate}

\textbf{Solution:} We first show $1 \iff 2$. Since the sequence converges the $\epsilon$ neighborhood of the limit
must contain points of $A$ (by the $\epsilon$ definition of limit). Moreover, as we proved in Lemma 2, this sequence
of balls $B_{1/n}(x)$ around the limit can be treated as a sequence which converges to $x$.

Next we shown $2 \iff 3$. Every nonempty $U\in M$ contains $B_\epsilon(x)$ for every $\epsilon$.
Thus if $B_{\epsilon}(x) \cap A \neq \emptyset$ for every $x \in M$ and every $\epsilon > 0$ then 
$U \cap A \neq \emptyset$ for every nonempty open set $U$. To show the reverse, simply note that $U$ contains
$B_\epsilon$. 

Next we show $1 \iff 4$. By definition, $M$ is the closure of $A$, in particular $M=\bar{A}$, thus the open set
of $A^c$ has interior equal to $M-\bar{A}$ which is empty set. The reverse if also true, if $Int(A^c)=\emptyset$, then
closure of $A$ is $M$, which is same as $M=\bar{A}$. As required.

%%%%%%%%%%%%%%%%%%%%%%%%%%%%%%%%%%%%%%%%%%%%%%%%%%%%%%

%% \item (Car. 4.58 -optional) Let $(r_n)$ be an enumeration of $\QQ$. For each $n$, let $I_n$ be the open interval centered at $r_n$ of radius  $2^{-n}$, and let $U = \bigcup_{n=1}^{\infty} I_n$. Prove that $U$ is a proper, open, dense subset of $\RR$ and that $U^c$ is nowhere dense in $\RR$.

%% \textbf{Solution:} 

%% %%%%%%%%%%%%%%%%%%%%%%%%%%%%%%%%%%%%%%%%%%%%%%%%%%%%%%

%% \item (Car. 5.30 -optional) Let $f:(M,d) \to (N ,\rho)$. Prove that $f$ is continuous if and only if $f(\bar{A}) \subset \overline{f(A)}$ for every $A \subset M$ if and only if $f^{-1}(B^{\circ}) \subset \left ( f^{-1}(B) \right )^{\circ}$ for every $B \subset N$. Give an example of a continuous $f$ such that $f(\bar{A}) \neq \overline{f(A)}$ for some $A \subset M$.

%% \textbf{Solution:} 

%% %%%%%%%%%%%%%%%%%%%%%%%%%%%%%%%%%%%%%%%%%%%%%%%%%%%%%%

%% \item (Car. 5.42 -optional) Suppose that $f: \QQ \to \RR$ is Lipschitz. Show that $f$ extends to a continuous function $h: \RR \to \RR$. Is $h$ unique? Explain. [Hint: Given $x \in \RR$, choose a sequence of rationals $(r_n)$ converging to $x$ and argue that $h(x) = \lim_{n \to \infty} f(r_n)$ exists and is actually independent of the sequence $(r_n)$.]

%% \textbf{Solution:} 
%%%%%%%%%%%%%%%%%%%%%%%%%%%%%%%%%%%%%%%%%%%%%%%%%%%%%%

\item (Car. 7.9 -optional) Give an example of a closed bounded subset of $\ell_{\infty}$ that is not totally bounded.

\textbf{Solution:} Consider the set of points with coordinates equal to the unit vectors
\[
x=(e_1,e_2,\ldots,)
\]
where only finitely many $e_i$ are non-zero. This is a closed bounded subset of $\ell_{\infty}$, but it
cannot be totally bounded for all $\epsilon$ (choose $\epsilon=0.5$).
%%%%%%%%%%%%%%%%%%%%%%%%%%%%%%%%%%%%%%%%%%%%%%%%%%%%%%
\end{enumerate}
\end{document}

