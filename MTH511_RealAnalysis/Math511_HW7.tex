\documentclass{article}
\usepackage{mathptmx,amssymb,amsmath}
\usepackage{breqn}
\usepackage[normalem]{ulem}
\usepackage{enumerate}
\usepackage[mathscr]{euscript}
\setlength{\textwidth}{16.5cm}
\setlength{\oddsidemargin}{-0.1cm}
\setlength{\evensidemargin}{-0.1cm}
\setlength{\textheight}{23cm}
\setlength{\topmargin}{-1.3cm}
\newtheorem{lem}{Lemma}

% Some handy shortcuts. 
\def\ge{\geqslant}
\def\le{\leqslant}
\def\phi{\varphi}
\def\to{\rightarrow}
\def\mapsto{\longmapsto}
\def\la{\langle}
\def\ra{\rangle}
\def\Aut{\operatorname{Aut}}
\def\diam{\operatorname{diam}}
\def\Image{\operatorname{Image}}
\def\Ker{\operatorname{Ker}}
\def\GL{\operatorname{GL}}
\def\SL{\operatorname{SL}}
\def\Perm{\operatorname{Perm}}
\def\nor{\vartriangleleft}
\def\nnor{\vartriangleright}
\def\lcm{\operatorname{lcm}}
\def\gcd{\operatorname{gcd}}
\def\li{\operatorname{li}}
\def\min{\operatorname{min}}
\def\max{\operatorname{max}}
\renewcommand{\mod}{\,\operatorname{mod}\,}
\newcommand{\norm}[1]{\left\lVert#1\right\rVert}

\def\CC{\mathbb C}
\def\FF{\mathbb F}
\def\NN{\mathbb N}
\def\QQ{\mathbb Q}
\def\RR{\mathbb R}
\def\ZZ{\mathbb Z}

\pagestyle{myheadings}
\begin{document}

% Replace Name. This is for the headers on all but the first page.
\markright{\hspace{10pt} Math 511 - Fall 2015 \hspace{100pt} Homework 7 - Sandeep Koranne }
\thispagestyle{empty}

\textbf{Math 511 - Fall 2015 \hfill Real Analysis  \hfill Instructor: Ossiander}

\hrulefill 
\medskip 

% Replace Name. This is for the first page header.
 {Sandeep Koranne \hfill  Homework 7 \hfill Due: November 25, 2015}
 \begin{center}
Ch. 7(16, 21, 22*, 23*) Ch. 8(1,2,7,15,17,31,33*,34*,36*)
 \end{center}
\medskip

\begin{enumerate}

%%%%%%%%%%%%%%%%%%%%%%%%%%%%%%%%%%%%%%%%%%%%%%%%%%%%%%

\item (Car. 7.16) Prove that $\RR^n$ is complete under any of the norms $\norm{\cdot}_1$, $\norm{\cdot}_2$, or $\norm{\cdot}_{\infty}$. [This is interesting because completeness is not usually preserved by the mere equivalence of \textit{metrics}. Here we use the fact that all of the metrics involved are generated by \textit{norms}. Specifically, we need the norms in question to be equivalent as functions: $\norm{\cdot}_{\infty} \le \norm{\cdot}_2 \le \norm{\cdot}_1 \le n \norm{\cdot}_{\infty}$. As we will see later, \textit{any} two norms on $\RR^n$ are comparable in this way.]

\textbf{Solution:} We use the ordering inequalities of the norms
we proved in previous homeworks:
\[
\norm{x}_\infty \le \|x\|_2 \le \norm{x}_1 \le n \|x\|_\infty
\]
Without loss of generality, assume that $\|x\|_p$ is an arbitrary 
norm on $\RR^n$,
then we have $\|x\|_p \le L \|x\|_2$, where $L$ is an arbitrary constant.
This implies that any $\|\cdot\|_p$ ball contains a $\|\cdot\|_2$ ball with
the same center. Thus, since $\RR^n$ is complete (hence, Cauchy) in any one
norm induced metric $d$, we have
\[
(\forall \epsilon)(\exists N \in \NN)(m,n > N, m,n \in \NN)\ d(x_m,x_n) \le \epsilon
\]
Since $d_p(x_m,x_n) \le L d(x_m,x_n)$, by choosing $\epsilon'=\epsilon/L$, we
have Cauchy convergence even in $d_p$. Thus $\RR^n$ is complete in any norm, as
required.

%%%%%%%%%%%%%%%%%%%%%%%%%%%%%%%%%%%%%%%%%%%%%%%%%%%%%%

\item (Car. 7.21) If $(M, d)$ is complete, prove that two Cauchy sequences $(x_n)$ and $(y_n)$ have the same limit if (and only if) $d(x_n, y_n) \to 0$.

\textbf{Solution:} Since $(x_n)$ is Cauchy,
\[
(\forall \epsilon)(\exists N_x \in \NN)(m_x,n_x \ge N_x; m_x,n_x\in \NN) 
d(x_{m_x},x_{n_x}) \le \epsilon
\]
and similarly since $(y_n)$ is Cauchy,
\[
(\forall \epsilon)(\exists N_y \in \NN)(m_y,n_y \ge N_y; m_y,n_y\in \NN)
d(y_{m_y},y_{n_y}) \le \epsilon
\]
Let $x_n\to x$ and $y_n \to y$ (since $x_n$ and $y_n$ are Cauchy, this
is well defined).
Choosing $N=max(N_x,N_y)$ we have $d(x_N,x) \le \epsilon$ and
$d(y_N,y) \le \epsilon$.  Thus using the triangle inequality we have,
\begin{equation}
(n>N) d(x_n,y_n)  \le d(x_n,x) + d(x,y) + d(y,y_n) \le 2\epsilon+d(x,y) \label{eqnA}
\end{equation}
Similarly,
\begin{equation}
d(x,y) \le d(x,x_n) + d(x_n,y_n) + d(y_n,y) \le 2\epsilon+d(x_n,y_n)\label{eqnB}
\end{equation}
Thus, using Eqn~\ref{eqnB} we can show that $x_n$ and $y_n$ have the same limit
since $d(x,y)\to 0$ when $d(x_n,y_n)\to 0$, as $\epsilon$ is arbitrary. 
Similarly, using Eqn~\ref{eqnA} we can show that if $d(x,y)\to 0$, then
this implies that $d(x_n,y_n)\to 0$. As required.

%%%%%%%%%%%%%%%%%%%%%%%%%%%%%%%%%%%%%%%%%%%%%%%%%%%%%%
 - - - - - -\textit{Unless otherwise stated, $(M,d)$ denotes a generic metric space.} - - - - - -

\item (Car. 8.1) If $K$ is a nonempty compact subset of $\RR$, show that $\sup K$ and $\inf K$ are elements of $K$.


\textbf{Solution:} 
By the definition of $\sup K$ and $\inf K$, we know
there exist sequences $(x_n),(y_n)$ in $K$ such that $\lim x_n \to \sup K$
and $\lim y_n \to \inf K$. By definition of compactness (Theorem 8.2), 
$K$ must contain
all limit points of all infinite sequences in $K$, thus $\lim x_n \in K$
and $\lim y_n \in K$, thus $\sup K \in K$ and $\inf K\in K$. As required.
%%%%%%%%%%%%%%%%%%%%%%%%%%%%%%%%%%%%%%%%%%%%%%%%%%%%%%

\item (Car. 8.2) Let $E = \{ x \in \QQ\ :\ 2 < x^2 < 3 \}$, considered as a subset of $\QQ$ (with its usual metric). Show that $E$ is closed and bounded but \textit{not} compact.

\textbf{Solution:} Obviously, $E$ is bounded, for example by $[0,2]$. To
show $E$ is closed, consider $E^c=(\-\infty,\sqrt{2}] \cup [\sqrt{3},\infty)$, 
since this set $E^c$ is open, $E$ must be closed (in $\QQ$). For $E$ to
be compact (Theorem 8.2) it must contain the limit 
point of all infinite sequences
in $E$, but consider the infinite sequence $(x_n)=\sqrt{3}-\frac{1}{n}$.
Obviously, $x_n\in E$ and $x_n \to \sqrt{3}$ as $n \to \infty$, but $\sqrt{3} \notin \QQ$, thus
$E$ does not contain the limit point of this infinite sequence, and is thus
not compact. 

%%%%%%%%%%%%%%%%%%%%%%%%%%%%%%%%%%%%%%%%%%%%%%%%%%%%%%

\item (Car. 8.7) If $K$ is a compact subset of $\RR^2$, show that $K \subset [a,b] \times [c,d]$ for some pair of compact intervals $[a,b]$ and $[c,d]$.

\textbf{Solution:} Since $K$ is compact, it is closed and bounded in $\RR^2$.
Moreover, let $C_K$ be a finite open cover for $K$, then $U=\RR^2\setminus K$ is an
open set and thus $C_K\cup U$ is a cover not just for $K$, but also for
any arbitrary set $T=[a,b]\times[c,d] \in \RR^2 \supset K$.
Since $T$ is also compact (any bounded, closed subset of a compact set
is compact), there exists a finite sub-cover $C_K'$ of $C_K$ such that
$ K \subseteq\  \cup\  C_K'$,
as required.

%%%%%%%%%%%%%%%%%%%%%%%%%%%%%%%%%%%%%%%%%%%%%%%%%%%%%%

\item (Car. 8.15) If $A$ is a totally bounded subset of a complete metric space $M$, show that $A$ is compact in $M$. For this reason, totally bounded sets are sometimes call \textit{precompact} or \textit{conditionally compact}. In fact, any set with compact closure might be labeled precompact.

\textbf{Solution:} Consider an infinite sequence $(x_n)\in A \subset M$. 
Since $M$
is complete $(x_n)$ has a Cauchy subsequence which converges to a limit in $M$.
Since $A$ is totally bounded, there exists a finite
cover of $A$, let the set of balls $B$ of radius 1 cover $A$.
Since $(x_n)\in A$, atleast one of the balls above must have an
infinite number of $x$ in it. Let that ball be $B_1$ and let 
$S_1$ denote the set of indices of $i$ such 
that $(x_i)_{i \in S_1} \subset B_1$.
Now we use induction. At the $k$-th step, a ball $B_k$ of radius $1/k$
contains an infinite number of $x_i, i\in S_k$. The sets 
$S_1 \supset S_2 \supset \ldots \supset S_k$, are thus 
nested and the $(x_j)_{j\in S_i}, 1\le j\le k$ form a subsequence. Since $A$
is complete, using the
Nested Intersection Theorem, these balls have non-zero intersection
\[
\bigcap_{n=1}^\infty S_n \ne \emptyset
\]
In fact there is a point $x_0 \in A$ such that
\[
\bigcap_{n=1}^\infty S_n = \{x_0\}
\]
This point $x_0$ is the limit point of the original sequence $(x_n)$
since every neighborhood of $x_0$ contains some sphere $S_k$ which
contains an infinite subsequence.
Thus every infinite sequence $(x_n)\in A$ has a limit point in $A$, thus
$A$ is compact in $M$, as required.
 
%%%%%%%%%%%%%%%%%%%%%%%%%%%%%%%%%%%%%%%%%%%%%%%%%%%%%%

\item (Car. 8.17) If $M$ is compact, show that $M$ is separable.

\textbf{Solution:} Since $M$ is compact, we can construct a
finite $1/n$-net for each $n$, and then consider the union of these
nets. The union of these nets is a dense countable subset, thus $M$
is separable, as required.

%%%%%%%%%%%%%%%%%%%%%%%%%%%%%%%%%%%%%%%%%%%%%%%%%%%%%%

\item (Car. 8.31) Given an arbitrary metric space $M$, show that a decreasing sequence of nonempty \textit{compact} sets in $M$ has nonempty intersection.

\textbf{Solution:} Let $F$ denote the sets of compact sets in $M$ such 
that the sequence of $\{F_\alpha\}$ is non-empty and decreasing. Consider
the complement of $F_\alpha$ denoted $G_\alpha=M\setminus F_\alpha$. 
Note that $G_\alpha$ is open for each $\alpha$. By the problem, $F_\alpha$ is
non-empty and decreasing, thus $\bigcap_{k=1}^n F_k \ne \emptyset$ for finite $n$.
But this implies that no finite system of 
$G_k=M\setminus F_k$ covers $M$. But then
the whole collection of $G_k = \cup G_k$ cannot cover $M$. This
implies that $\bigcap_\alpha F_\alpha \ne \emptyset$. Thus if $F_\alpha$ are compact,
then their decreasing sequence has nonempty intersection.

%%%%%%%%%%%%%%%%%%%%%%%%%%%%%%%%%%%%%%%%%%%%%%%%%%%%%%

\begin{center}
\hrulefill  \quad \large{\textit{Optional Exercises}} \quad \hrulefill 
\end{center}

%%%%%%%%%%%%%%%%%%%%%%%%%%%%%%%%%%%%%%%%%%%%%%%%%%%%%%

\item (Car. 7.22- optional) Let $D$ be a dense subset of a metric space $M$, and suppose that every Cauchy sequence from $D$ converges to some point of $M$. Prove that $M$ is complete.

\textbf{Solution:}We use the definition of \textbf{dense} as 
$B_\epsilon(x) \bigcap D \ne \emptyset$ for all $x \in M$. Thus every infinite 
sequence in $M$ has an atmost $\epsilon$-distance equivalent sequence in $D$.
Moreover, if this equivalent sequence in $D$ is Cauchy, then according to
the problem, the sequence converges to a point of $M$. Using the triangle
inequality:
\[
d(x_m,x_n) \le d(x_m,x'_m) + d(x'_m,x'_n) + d(x'_n,x_n) \le \epsilon + d(x'_m,x'_n) + \epsilon \le 3\epsilon
\]
where $x'_m,x'_n \in D$ are $\epsilon$ distance from $x_m,x_n \in M$, 
respectively, 
and $(\forall \epsilon)(\exists N\in\NN)(m,n > N)d(x'_m,x'_n)<\epsilon$, since
the sequence considered in $D$ is Cauchy. Thus, the sequence in $M$ is also
Cauchy.
Let the sequence in $D$ converge to a point $\{x\}\in M$, then
\[
(\forall \epsilon)(\exists N\in\NN)(m>N,m\in\NN) d(x_m,x) \le d(x'_m,x)+ 2\epsilon \le 3\epsilon
\]
Thus every Cauchy sequence in $M$ also coverges to
some point in $M$, and thus $M$ is complete, as required.

%%%%%%%%%%%%%%%%%%%%%%%%%%%%%%%%%%%%%%%%%%%%%%%%%%%%%%

\item (Car. 7.23- optional) Prove that $M$ is complete if and only if every sequence $(x_n)$ in $M$ satisfying $d(x_n, x_{n+1}) < 2^{-n}$, for all $n$, converges to a point $M$.

\textbf{Solution:} We use the definition of Cauchy completeness of $M$.
Cauchy criteria implies
\[
(\forall \epsilon)(\exists N\in\NN)(m,n > N)d(x_m,x_n)\le\epsilon
\]
By the problem, $d(x_n, x_{n+1}) < 2^{-n}$, for all $n$, given any
$\epsilon$ we can find $N$ such that $2^{-n} \le \epsilon, \forall n > N$, thus
if $M$ is complete, then every Cauchy sequence will converge to a point in $M$.
Conversely, if every sequence of the form above converges to a point in $M$, 
then every Cauchy sequence also converges to a point in $M$, thus $M$ is
complete.
%%%%%%%%%%%%%%%%%%%%%%%%%%%%%%%%%%%%%%%%%%%%%%%%%%%%%%

\item (Car. 8.33- optional) Let $(M,d)$ be compact. Suppose that $(F_n)$ is a decreasing sequence of nonempty closed sets in $M$, and that $\bigcap_{n=1}^{\infty} F_n$ is contained in some open set $G$. Show that $F_n \subset G$ for all but finitely many $n$.

\textbf{Solution:} Since $\bigcap_{n=1}^{\infty} F_n \subseteq G$, and $F_n$ 
is a decreasing sequence of nonempty closed sets in $M$ we
know $\exists N \in \NN$ such that 
\[
(x_m,x_n\in F_n, m,n>N)\ \sup\ d(x_m,x_n)  \le \textrm{diam}\ G
\]
Since $M$ is compact, by Nested Intersection Theorem $\bigcap_{n=1}^{\infty} F_n
= \{x_0\} \ne \emptyset$. Moreover, $B_\epsilon(x_0) \subseteq G$, and since
$G$ is open $\epsilon >0$. By construction of this ball, only finitely
many (upto $N$) elements of $F_n$ can be outside $B_\epsilon(x_0)$, and thus
$F_n \subset G$ for all but finitely many $n$, as required.

%%%%%%%%%%%%%%%%%%%%%%%%%%%%%%%%%%%%%%%%%%%%%%%%%%%%%%

\item (Car. 8.34- optional) Let $A$ be a subset of a metric space $M$. Prove that $A$ is closed in $M$ if and only if $A \cap K$ is compact for every compact set $K$ in $M$. [Hint: If $(x_n)$ converges to $x$, then $\{x \} \cup \{ x_n\ :\ n \ge 1 \}$ is compact. (Why?)]

\textbf{Solution:} As the hint specifies we can consider the set
\[
(x_n) \to x \implies \{x \} \cup \{ x_n\ :\ n \ge 1 \} \ \textrm{is compact}
\]
Assume that $A \cap K$ is compact for every compact set $K$ in $M$, then we
have to show that $A$ is closed in $M$. But, if $A \cap K$ is compact
then $A \cap K$ must contain the limit points of all convergent subsequences
in $A$ (since we consider the set of all compact sets $K$). 
Thus, $A$ must contain all the limit points, and thus by definition of
closed sets, $A$ is closed in $M$.

In the other direction, assume that $A$ is closed in $M$, then $A$ contains
all limit points, and is therefore compact in $M$. Since $K$ is already
compact, and intersection of compact sets is compact, $A \cap K$ is compact 
in $M$ as required.

%%%%%%%%%%%%%%%%%%%%%%%%%%%%%%%%%%%%%%%%%%%%%%%%%%%%%%

\item (Car. 8.36- optional) Let $F$ and $K$ be disjoint, nonempty subsets of a metric space $M$ with $F$ closed and $K$ compact. Show that $d(F,K) = \inf \{ d(x,y)\ :\ x \in F,\ y \in K \} > 0$. Show that this may fail if we assume only that $F$ and $K$ are disjoint closed sets.

\textbf{Solution:} 

%%%%%%%%%%%%%%%%%%%%%%%%%%%%%%%%%%%%%%%%%%%%%%%%%%%%%%

\end{enumerate}
\end{document}

