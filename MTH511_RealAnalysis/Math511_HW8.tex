%%%%%%%%%%%%%%%%%%%%%%%%%%%%%%%%%%%%%%%%%%%%%%%%%%%%%%%%%%%%%%%%%%%%%%%%%%%%%%%%
%% MTH511 HW8 TeX source
%% Author : Sandeep Koranne
%% Good luck
%%%%%%%%%%%%%%%%%%%%%%%%%%%%%%%%%%%%%%%%%%%%%%%%%%%%%%%%%%%%%%%%%%%%%%%%%%%%%%%%
\documentclass{article}
\usepackage{mathptmx,amssymb,amsmath}
\usepackage{breqn}
\usepackage[normalem]{ulem}
\usepackage{enumerate}
\usepackage[mathscr]{euscript}
\setlength{\textwidth}{16.5cm}
\setlength{\oddsidemargin}{-0.1cm}
\setlength{\evensidemargin}{-0.1cm}
\setlength{\textheight}{23cm}
\setlength{\topmargin}{-1.3cm}
\newtheorem{lem}{Lemma}

% Some handy shortcuts. 
\def\ge{\geqslant}
\def\le{\leqslant}
\def\phi{\varphi}
\def\to{\rightarrow}
\def\mapsto{\longmapsto}
\def\la{\langle}
\def\ra{\rangle}
\def\Aut{\operatorname{Aut}}
\def\diam{\operatorname{diam}}
\def\Image{\operatorname{Image}}
\def\Ker{\operatorname{Ker}}
\def\GL{\operatorname{GL}}
\def\SL{\operatorname{SL}}
\def\Perm{\operatorname{Perm}}
\def\nor{\vartriangleleft}
\def\nnor{\vartriangleright}
\def\lcm{\operatorname{lcm}}
\def\gcd{\operatorname{gcd}}
\def\li{\operatorname{li}}
\def\min{\operatorname{min}}
\def\max{\operatorname{max}}
\renewcommand{\mod}{\,\operatorname{mod}\,}
\newcommand{\norm}[1]{\left\lVert#1\right\rVert}

\def\CC{\mathbb C}
\def\FF{\mathbb F}
\def\NN{\mathbb N}
\def\QQ{\mathbb Q}
\def\RR{\mathbb R}
\def\ZZ{\mathbb Z}

\pagestyle{myheadings}
\begin{document}

% Replace Name. This is for the headers on all but the first page.
\markright{\hspace{10pt} Math 511 - Fall 2015 \hspace{100pt} Homework 8 - Sandeep Koranne }
\thispagestyle{empty}

\textbf{Math 511 - Fall 2015 \hfill Real Analysis  \hfill Instructor: Ossiander}

\hrulefill 
\medskip 

% Replace Name. This is for the first page header.
 {Sandeep Koranne \hfill  Homework 8 \hfill Due: December 7, 2015}
 \begin{center}
Ch.~16(1*,5*,6*,8,16,17*,18,22,40,42(a),45,47*,48*,49*,52*,53) and
Ch.~18(3,4,10,11,38)
 \end{center}
\medskip

\begin{enumerate}

%%%%%%%%%%%%%%%%%%%%%%%%%%%%%%%%%%%%%%%%%%%%%%%%%%%%%%

\item (Car. 16.1) Let $f$ be a nonnegative bounded function
on $[a,b]$ with $0 \le f\le M$. Let
\[
E_{n,k} = \left\{\frac{kM}{2^n} \le f < \frac{(k+1)M}{2^n} \right\}
\]
for each $n=1,2,\ldots$ and $k=0,1,\ldots,2^n$, and set
$\phi_n = \sum_{k=0}^{2^n} (kM/2^n) \chi_{E_{n,k}}$. Prove that
$0 \le \phi_n \le \phi_{n+1} \le f$ and that $0 \le f-\phi_n \le 2^{-n}M$
for each $n$. Thus $(\phi_n)$ converges uniformly to $f$ on $[a,b]$.

\textbf{Solution:} We have
\[
E_{n,k} = \left\{\frac{kM}{2^n} \le f < \frac{(k+1)M}{2^n} \right\} = E_{n+1,2k} \bigcup E_{n+1,2k+1}
\]
On $E_{n+1,2k}$, we have $\phi_n=(kM)/2^n = (2kM)/2^(n+1) = \phi_{n+1}$.
On $E_{n+1,2k+1}$ also we have $\phi_n<\phi_{n+1}$, because
\[
\phi_n = (kM)/2^n < (2k+1)M/2^{(n+1)} = \phi_{n+1}
\]
Combining,
it is obvious that
$\phi_n \le \phi_{n+1}$. To show $\phi_{n+1}\le f$, consider the
definition of $E_{n,k}$. Moreover, if $f(x) < \infty$, then
\[
0 \le f(x) -\phi_n(x) < \frac{(k+1)M}{2^n} - \frac{kM}{2^n} = \frac{M}{2^n}
\]
Thus $f(x) \ge \phi_n(x)$ and as $n\to\infty$, 
$(\phi_n)$ converges uniformly to $f$ on $[a,b]$.

\item (Car. 16.5) If we define $rE = \{ rx: x \in E\}$, what is
$m^*(rE)$ in terms of $m^*(E)$.

\textbf{Solution:} Using the definition of $m^*(E)$
\[
m^*(E) = \inf\ \left\{ \sum_{n=1}^\infty \ \ell(I_n) : E \subset \bigcup_{n=1}^\infty I_n \right\}
\]
With the scaling of $rE$, each $\ell(I_n)=r\ell(I_n)$. Thus
$m^*(rE)=r\ m^*(E)$.

\item (Car. 16.6) If $E$ has non-empty interior, show that $m^*(E)>0$.

\textbf{Solution:} Use the definition of interior of $E$ such that
there exists an open ball of radius $\epsilon>0$ such that 
$U_\epsilon(x), x\in E$ is completely contained in $E$, thus 
$\diam U_\epsilon(x)>0$ and thus $m^*(E)>2\epsilon>0$.

\item (Car. 16.8) Given $\delta>0$, show that $m^*(E) = \inf 
\ \sum_{n=1}^\infty\ \ell(I_n)$, where the infemum is taken over
all coverings of $E$ by sequences of intervals $(I_n)$, where each
$I_n$ has diameter less than $\delta$.

\textbf{Solution:} By the definition of \textbf{outer measure}
we know $m^*(E)$ is defined as:
\[
m^*(E) = \inf\ \left\{ \sum_{n=1}^\infty \ \ell(I_n) : E \subset \bigcup_{n=1}^\infty I_n \right\}
\]
where the infemum is taken over all coverings of $E$ by countable union
of intervals. By definition of the infemum, we know there exists
a disjoint set of intervals $J_n$ such that $E \subseteq \cup J_n$. Moreover,
if we partition $J_n$ further into disjoint intervals $I_n$,
where each
$I_n$ has diameter less than $\delta$ then 
\[
m^*(E) = m^*((J_n)) = \sum_n m^*(J_n) = \sum_n m^*(I_n) = m^*((I_n))
\]
as the measure does not change as long as the intervals are disjoint.

\item (Car. 16.16) If $m^*(E)=0$, show that $m^*(E\cup A) = m^*(A) = m^*(A\setminus E)$ for any $A$. 

\textbf{Solution:} We know that $m^*(E\cap A) \le m^*(E)$, thus if 
$m^*(E)=0$, $m^*(E\cap A)=0$. We can write $m^*(E\cup A)$ as
\[
m^*(E\cup A) = m^*(E) + m^*(A) - m^*(E\cap A) = 0 + m^*(A) - 0
\]
thus $m^*(E\cup A) = m^*(A)$. Also,
\[
m^*(A\cup E) = m^*(E \cup A\setminus E ) = m^*(E) + m^*(A\setminus E) - m^*(E \cap A\setminus E)
\]
Since $(E \cap A\setminus E)=\emptyset$, we get using the previous result
and $m^*(E)=0$,
\[
m^*(A\cup E) = m^*(A) = m^*(A\setminus E)
\]

\item (Car. 16.17-optional)

\textbf{Solution:} 

\item (Car. 16.18) If $E$ is a compact set with $m^*(E)=0$, and if $\epsilon>0$, prove that
  $E$ can be covered by finitely many open intervals, $I_1,\ldots,I_n$, satisfying $\sum_{j=1}^n m^*(I_j) < \epsilon$.

\textbf{Solution:} By definition of the measure we know there exist
a set of intervals $(I_n)$ such that $E\subseteq \bigcup_n I_n$ and
\[
\sum_{j=1}^n m^*(I_j) < m^*(E) + \epsilon 
\]
With $m^*(E)=0$ as given in the problem to get the required result.

\item (Car. 16.22) Let $E=\bigcup_{n=1}^\infty E_n$. Show that $m^*(E)=0$ if and only if $m^*(E_n)=0$ for every $n$.

\textbf{Solution:} Consider the definition of measure
\[
m^*(E) = m^*(\bigcup_{n=1}^\infty E_n) \le \sum_{n=1}^\infty m^*(E_n)
\]
The backward direction of the proof is obvious, since if each $m^*(E_n)=0$,
their sum is zero, thus $m^*(E)=0$. To show the forward direction
consider another set $F$ defined as 
 $F_1 = E_1$, $F_2 = E_2\setminus E_1$ and in general
\[
F_n = E_n \bigcap_{i=1}^{n-1} E_i^c
\]
Then $F_i \cap F_j = \emptyset$ for $i\ne j$ and $F_n \subseteq E_n$
and $\bigcup_{i=1}^\infty F_n = \bigcup_{j=1}^\infty E_n$.
Then we can write $m^*(E)$ as
\[
m^*(\bigcup E_n) = m^*(\bigcup F_n) = \sum m^*(F_n) \le \sum m^*(E_n)
\]
Thus if $m^*(E)=0$ then each $m^*(F_n)=0$, and using nonnegativity of
$E_n$ and $F_n$, this implies that each $E_n=0$.
 
\item (Car. 16.40) If $A$ and $B$ are measurable sets, show that $m(A\cup B)+m(A\cap B)=m(A)+m(B)$.

\textbf{Solution:} We can write $(A\cup B)=A\cup (B\setminus A)$ and
$B = (A \cap B) \cup (B\setminus A)$.
Thus we have
\[
m(A\cup B) = m(A\cup (B\setminus A)) = m(A) + m(B\setminus A) - m( A \cap ( B \setminus A ) ) = m(A) + m(B\setminus A) + m(\emptyset)
\]
and
\[
m(B) = m(A\cap B) + m(B\setminus A) 
\]
Here we have used the fact that the partition of $B$ above is disjoint.
Combining and eliminating $m(B\setminus A)$ we get 
\[
m(A\cup B) - m(A) = m(B) - m(A\cap B)
\]
Rearrange to get the required result.

\item (Car. 16.42(a)) Suppose that $E$ is measurable with $m(E)=1$. Show that:
  \begin{enumerate}
    \item There is a measurable set $F \subset E$ such that $m(F)=1/2.$.
  \end{enumerate}

  \textbf{Solution:} As the hint specifies, consider the function
  \[
  f(x) = m(E\cap (-\infty,x])
  \] Since $f(x)$ is continuous and also we can scale $f$ such that 
$f(x_0)=m(\emptyset)=0$ and $f(x_1)=m(E)=1$, then by the indermediate value theorem and
using Theorem 16.23 of the textbook, there
exist $x$ such that $f(x)=1/2$, and thus there exist $F\subset E$ 
such that $m(F)=1/2$.

\item (Car. 16.45) Let $f: X\to Y$ be any function:
  \begin{enumerate}
  \item If $\mathcal{B}$ is a $\sigma$-algebra of subsets of $Y$, show that $\mathcal{A} = \{f^-1(B):
    B \in \mathcal{B}\}$ is a $\sigma$-algebra of subsets of $X$.

\textbf{Solution:}     
Let $E\subset Y$ such that $f^-1(E) \subseteq X$. 
Since $\mathcal{B}$ is a $\sigma$-algebra of subsets of $Y$, it contains
the emptyset and the complete set, thus the inverse image in $X$
also contains the emptyset and $f^-1(E)$. 
We have $f^-1(Y) = X$, and $f^-1(Y-A) = X - f^-1(A)$. Also, 
\[
f^-1( \bigcup_{i=1}^\infty A_i ) = \bigcup_{i=1}^\infty f^-1(A_i) 
\]
Thus $\mathcal{A}$ is closed under countable union, and thus
$\mathcal{A}$ is a $\sigma$-algebra of subsets of $X$.

  \item If $\mathcal{A}$ is a $\sigma$-algebra of subsets of $X$, show that $\mathcal{B} = \{B: f^-1(B)
    \in \mathcal{A}\}$ is a $\sigma$-algebra of subsets of $Y$.

\textbf{Solution:}     
We have $f^-1(Y) = X$, and $f^-1(Y-A) = X - f^-1(A)$. Also, 
\[
f^-1( \bigcup_{i=1}^\infty A_i ) = \bigcup_{i=1}^\infty f^-1(A_i) 
\]
because $\mathcal{A}$ is a $\sigma$-algebra of subsets of $X$.
Hence,
$\emptyset,Y$ belong to $\mathcal{B}$, and also it is closed under
countable union, thus $\mathcal{B}$ is a $\sigma$-algebra of subsets of $Y$.

    \end{enumerate}

\item (Car. 16.47) Given that $\{\emptyset,\RR\}$ and $\mathcal{P}(\RR)$
are both $\sigma$-algebras. 

\textbf{Solution:} Using the definition of $\sigma$-algebra we know
that any other $\sigma$-algebra must contain $\emptyset$ and $\RR$.
Moreover, since it comprises of subsets of $\RR$ it must be contained
in $\mathcal{P}(\RR)$.

\item (Car. 16.48) Let $\mathcal{E}$ be any collection of subsets of $\RR$.
Show that there is always a smallest $\sigma$-algebra $\mathcal{A}$ 
containing $\mathcal{E}$

\textbf{Solution:} 
%Since $\mathcal{E}\subseteq \mathcal{P}(\RR)$ we know
%that $\mathcal{E}\ne\emptyset$. 
Let $\mathcal{A}=\bigcap \mathcal{S}$,
where $\mathcal{S}$ is the set of all $\sigma$-algebras of subsets of
$\RR$ containing $\mathcal{E}$. Since $\mathcal{P}(\RR)$ is one
such $\sigma$-algebra, we know atleast one $\mathcal{S}$ is not empty.
Using the definition of $\sigma$-algebra that it is closed under
complementation, countable union and using DeMorgan's law
\[
(x \cap y)^c = x^c \cup y^c
\]
if $x,y\in\mathcal{S}$, then $x\cap y \in \mathcal{S}$. Thus
$\mathcal{A}=\bigcap\mathcal{S}$ is the smallst $\sigma$-algebra
containing $\mathcal{E}$.

\item (Car. 16.49) The smallest $\sigma$-algebra containing
$\mathcal{E}$ is called \textbf{the $\sigma$-algebra generated by}
$\mathcal{E}$ and is denoted $\sigma(\mathcal{E})$. 
If $\mathcal{E}\subset\mathcal{F}$, prove that
$\sigma(\mathcal{E})\subset \sigma(\mathcal{F})$.

\textbf{Solution:} Since by definition $\sigma(\mathcal{E})$ is
the smallest $\sigma$-algebra containing $\mathcal{E}$, any other
$\sigma$-algebra on the subsets of $\RR$ which contains $\mathcal{E}$
must also contain $\sigma(\mathcal{E})$ (this is 
because $\sigma(\mathcal{E})$ is defined using set intersection).

\item (Car. 16.52) 
\textbf{Solution:}


\item (Car. 16.53) The \textbf{Borel $\sigma$-algebra} $\mathcal{B}$ is defined to be the smallest
  $\sigma$-algebra of subsets of $\RR$ containing the open sets; equivalently, $\mathcal{B}$ is the
  $\sigma$-algebra generated by the (open) intervals. The elements of $\mathcal{B}$ are called the
  \textbf{Borel sets}. Notice that closes sets, $G_\delta$-sets, $F_\sigma$-sets, $G_{\delta\sigma}$-sets, and so
  on, are all Borel sets. From Corollaries 16.17 and 16.20, every Borel set is measurable; that
  is $\mathcal{B} \subset \mathcal{M}$. Show that $\mathcal{B}$ is generated by each of the following:
  \begin{enumerate}
  \item The open intervals $\mathcal{E}_1=\{(a,b): a<b\}$

    \textbf{Solution:} The open interval $(a,b)$ can be written as
\[
(a,b) = \bigcup_{n_0}^\infty \left[ a+\frac{1}{n}, b-\frac{1}{n} \right]
\]
where $\frac{1}{n_0} < (b-a)$. This contains the empty set and the complete
set and it closed under countable union, thus is a $\sigma$-algebra
generated by open intervals.
     
\item The closed intervals $\mathcal{E}_2=\{[a,b]: a<b\}$

  \textbf{Solution:} The closed interval $[a,b]$ can be written as
\[
[a,b] = \{a\} \bigcup (a,b) \bigcup \{b\}
\]
thus $\mathcal{E}_1 \subset \mathcal{E}_2$, as required, and thus
is contained in $\sigma(\mathcal{E}_2)$.
\item The half-open intervals $\mathcal{E}_3=\{(a,b],[a,b): a<b\}$

    \textbf{Solution:} The half-open 
intervals $\mathcal{E}_3=\{(a,b],[a,b): a<b\}$ can be written as
\[
\{(a,b]: a<b\} = (a,b) \bigcup \{b\}
\]
and similarly for $[a,b\}=\{a\} \bigcup (a,b)$, thus
$\mathcal{E}_1 \subset \mathcal{E}_3$, as required, and thus
is contained in $\sigma(\mathcal{E}_3)$.

  \item The open rays $\mathcal{E}_4=\{(a,\infty),(-\infty,b): a,b\in \RR\}$

    \textbf{Solution:} Set $b=\infty$ and similarly set $a=-\infty$, to get
$\mathcal{E}_1 \subset \mathcal{E}_4$, as required, and thus
is contained in $\sigma(\mathcal{E}_4)$.
  \item The closed rays $\mathcal{E}_5=\{[a,\infty),(-\infty,b]: a,b\in \RR\}$.

    \textbf{Solution:}     
Set $b=\infty$ and then $\mathcal{E}_5 = \{a\}\bigcup (a,b)$, similarly
set $a=-\infty$ and then $\mathcal{E}_5 = (a,b)\bigcup \{b\}$. Thus
$\mathcal{E}_1 \subset \mathcal{E}_5$, as required, and thus
is contained in $\sigma(\mathcal{E}_5)$.
  \end{enumerate}

%%%%%%%%%%%%%%%%%%%%%%%%%%%%%%%%%%%%%%%%%%%%%%%%%%%%%%%%%%%%%%%%%%%%%%%%%%%%%%%%

\item (Car. 18.3) Prove that $\int_1^\infty (1/x)\ dx = \infty$ (as a Lebesgue integral).

\textbf{Solution:} We define a simple function
\[
\phi_n (x) = \sum_{i=1}^N \frac{1}{i+1} \chi_{i,i+1} (x)
\]
we can easily verify that $\phi_n(x) \le f(x)$ where $f(x)=1/x$.
Moreover $\phi$ is simple, nonnegative and integrable, 
and then use the definition of Lebesgue integral
\[
\int_1^\infty (1/x)\ dx = \sup \left\{ \int \phi : 0 \le \phi \le f \right\}
\]
Thus
\[
\int_1^\infty (1/x)\ dx \ge \sum_{i=1}^N \frac{1}{i+1} \chi_{i,i+1} (x)
\]
We know $m(\chi_{i,i+1}(x))=1$, but the series $\sum_{i=1}^N \frac{1}{i+1}$
diverges to $\infty$, thus the Lebesgue integral is $\infty$ as well.

\item (Car. 18.4) Find a sequence $(f_n)$ of nonnegative measurable functions such that
  $\lim_{n\to\infty} f_n=0$, but $\lim_{n\to\infty}\int f_n=1$. In fact, show that $(f_n)$ can be
  chosen to converge \emph{uniformly} to 0.

\textbf{Solution:} Consider the function $f_n(x)=\frac{2}{\pi}\frac{n}{1+n^2x^2}$, then
$\lim_{n\to\infty} f_n=0$. Next consider the integral
\[
 \int  \frac{2}{\pi}\frac{n}{1+n^2x^2} \ dx
\]
We know
\[
 \int  \frac{2}{\pi}\frac{n}{1+n^2x^2} \ dx = \frac{2}{\pi}\arctan(nx)
\]
And,
\[
\lim_{n\to\infty} \frac{2}{\pi}\arctan(nx) = \frac{2}{\pi}\arctan({\infty}) = 
\frac{2}{\pi}\frac{\pi}{2}=1
\]
As required. 

To show uniform convergence, we have to show that 
for any $\epsilon>0$, there is an integer $N$
such that 
\[
|f_n(x) - f(x)| < \epsilon
\] for all $x\in\RR$ and $n>N$. Since the question is asking the conditions
for an arbitrary $f_n$, we have to give conditions for uniform
convergence for arbitrary $f_n$. If $f_n$ was uniformly continuous, then
$N$ could be chosen to be independent of $x$ near 0.


\item (Car. 18.10) If $f$ is nonnegative and measurable, show that $\int_{-\infty}^\infty f = \lim_{n\to\infty}
  \int_{-n}^n f = \lim_{n\to\infty} \int_{\{f \ge (1/n)\}} f$.

\textbf{Solution:} Since $f$ is nonnegative and measurable we can write
it as the limit of an increasing sequence of integrable simple functions
$0 \le \phi_1 \le \phi_2 \le \cdots \le f$ such that 
$f=\lim_{n\to\infty} \phi_n$ and moreover $\int f=\lim_{n\to\infty}\int\phi_n$
using Corollary 18.8 of the textbook. Thus
\[
\int_{-\infty}^\infty f = \lim_{n\to\infty}\int_{-\infty}^\infty  \phi_n = \lim_{n\to\infty}\int_{-n}^n f
\]
Since $f>0$, the support of $f$ can be written as:
\[
\{f>0\}=\bigcup_{i=1}^\infty \{ f\ge(1/n)\}
\]
Using this $\phi_n=\sum_{i=1}^n f^{-1}(\chi_i)m(\chi_i)$ where $\chi_i$ is the
above mentioned support. 
Thus $f_n$ is increasing, and we can use the monotone convergence theorem.
Combining this with the previous result we get:
\[
\lim_{n\to\infty}\int_{-n}^n f = \lim_{n\to\infty} \int_{\{f\ge(1/n)\}} f
\]

\item (Car. 18.11) If $f$ is nonnegative and integrable, show that $\int_{-\infty}^\infty f = \lim_{n\to\infty}
  \int_{\{f\le n\}} f$.

\textbf{Solution:} Since $f$ is nonnegative we can use the $\phi_n$ we
used in the previous problem as well as the fact that $f_n$ is increasing
we can use the monotone convergence theorem; 
moreover since $f$ is also integrable, we know
it is finite a.e. Thus, the support can be modified to 
be $\{f>0\} = \bigcup_{i=1}^\infty \{f\le i\}$. Thus using the previous problem,
\[
\int_{-\infty}^\infty f = \lim_{n\to\infty}\int_{-n}^n  f_n = \lim_{n\to\infty} \int_{\{f\le n\} } f
\]

\item (Car. 18.38) If $f\in L_1[0,1]$, show that $x^nf(x) \in L_1[0,1]$ for $n=1,2,\ldots$ and compute
  $\lim_{n\to\infty} \int_0^1 x^n f(x) dx$.

\textbf{Solution:} Since the series $\sum_{i=1}^\infty x^i$ converges for 
$x\in [0,1)$, we can use Lebesgue integration since $f\in L_1[0,1]$ also.
Since $x^n$ can be dominated by the constant function $g(x)=1,x\in[0,1]$, 
we can dominate
$x^nf(x)$ by $f(x)$, which is in $L_1[0,1]$, thus using the dominated
convergence theorem (18.19) we have $x^nf(x)\in L_1[0,1]$.
Moreover since $\lim_{n\to\infty} x^n=0$ for $x<1$, thus the computed 
limit for   $\lim_{n\to\infty} \int_0^1 x^n f(x) dx$ is $f(1)$.


\end{enumerate}
\end{document}

