%%%%%%%%%%%%%%%%%%%%%%%%%%%%%%%%%%%%%%%%%%%%%%%%%%%%%%%%%%%%%%%%%%%%%%%%%%%%%%%%
%% HW4
%% Sandeep Koranne
%% Good Luck
%%%%%%%%%%%%%%%%%%%%%%%%%%%%%%%%%%%%%%%%%%%%%%%%%%%%%%%%%%%%%%%%%%%%%%%%%%%%%%%%
\documentclass{article}[12pt]
\usepackage{amssymb}
\usepackage{amsmath}
\usepackage{amsfonts}
\usepackage{amsthm}
\usepackage{polynom}
\usepackage{tikz}
\usepackage{subfigure}
%\usepackage{mathtools}
\newtheorem{lem}{Lemma}
\def\RR{\mathbb R}
\def\CC{\mathbb C}
\def\QQ{\mathbb Q}
\def\FF{\mathbb F}
\def\ZZ{\mathbb Z}
\DeclareMathOperator{\sgn}{sgn}

\begin{document}
\title{Spring 2016 Gr\"obner Basis Homework 4}
\author{Sandeep Koranne}
\date{May 27 2016}
\maketitle

\section*{Problem 1}A polynomial $f\in R=K[x_1,x_2,\ldots,x_n]$ is
called symmetric if for any permutation $\sigma\in S_n$ (where $S_n$
denotes the permutation group of $n$), it follows that
\[
f(x_1,x_2,\ldots,x_n) = f(x_{\sigma(1)},x_{\sigma(2)},\ldots,x_{\sigma(n)})
\]
Use Theorem 2.4.4 to give a method for deciding whether a given
polynomial $f\in R$ is symmetric.
\begin{proof}
Theorem 2.4.4 of the book states that $f\in k[x_1,x_2,\ldots,x_n]$
is in the image of a map $\phi$, iff, there exists $h\in k[y_1,y_2,\ldots,y_m]$
such that $f\xrightarrow{G}_+ h$, where $G$ is a reduced Gr\"obner basis
for $K$ wrt variables $y$ greater than $x$. 
This gives a method of recognizing symmetric polynomials, since we
can create a Gr\"obner basis of elementary symmetric polynomials of $n$
variables, then use Gr\"obner reduction to calculate $h$. If $f$ is symmetric
then the reduction will be identically 0.

We first discuss elementary symmetric polynomials.
We first give some preliminary lemmas which are related to the study of
evaluation maps as ring homomorphism, and the representation of symmetric
polynomials using \emph{elementary symmetric polynomials}.

\begin{lem}
Given a field $F$, a ring $R$ containing $F$, and 
$\alpha_1,\alpha_2,\ldots,\alpha_n \in R$, and a polynomial 
$f\in R[x_1,x_2,\ldots,x_n]$,
evaluation map
\[
F[x_1,x_2,\ldots,x_n] \to R
\]
is defined by 
\[
f(x_1,x_2,\ldots,x_n)\mapsto f(\alpha_1,\alpha_2,\ldots,\alpha_n)
\]
Moreover, this evaluation map is a ring homomorphism $F[x_1,x_2,\ldots,x_n]\to R$.
\label{lem:ring-homomorphism}
\end{lem}
This follows from results on polynomials that they are closed under
addition, and multiplication, and we have discussed this in class as well.
For the second lemma, we introduce elementary symmetric polynomials, which
are defined as follows. Let $x_1,x_2,\ldots,x_n$ be variables over a field
$F$, then
\begin{eqnarray}
\sigma_1 & = & x_1 + x_2 + \cdots + x_n \nonumber \\
\sigma_2 & = & \sum_{1 \le i \le j \le n} x_ix_j \nonumber \\
 & \vdots &  \nonumber \\
\sigma_n & = & x_1x_2\ldots x_n
\end{eqnarray}

\begin{lem}
Let $x_1,x_2,\ldots,x_n$ be variables over a field
$F$, then given another variable $x$ we have
\begin{equation}
(x-x_1)\cdots(x-x_n) = x^n -\sigma_1x^{n-1}+\cdots+(-1)^r\sigma_r x^{n-r}
+\cdots+(-1)^n\sigma_n \label{eqn:eval}
\end{equation}
Follows from multiplying out LHS and RHS and comparing the coefficients of each
power of $x$.
Now if we evaluate Equation~\ref{eqn:eval} at 
$x_1=\alpha_1,x_2=\alpha_2,\ldots,x_n=\alpha_n$ we get
\begin{eqnarray}
(x-\alpha_1)(x-\alpha_2)\cdots(x-\alpha_n) & = & x^n -\sigma_1(\alpha_1,\alpha_2,\ldots,\alpha_n)x^{n-1} +\cdots+ \nonumber \\
 & & (-1)^{n-1}\sigma_{n-1}(\alpha_1,\alpha_2,\ldots,\alpha_n)x \nonumber \\
 & & + (-1)^n\sigma_n(\alpha_1,\alpha_2,\ldots,\alpha_n) \label{eqn:eval2}
\end{eqnarray}
\end{lem}

\begin{lem}
A symmetric polynomial is a polynomial $f\in F[x_1,x_2,\ldots,x_n]$, such that
its value (at a given point) is independent of the permutation of the variables.
Any symmetric polynomial in $F[x_1,x_2,\ldots,x_n]$ can be written as 
a polynomial in $\sigma_1,\sigma_2,\ldots,\sigma_n$ with coefficients in $F$.
\end{lem}
This follows from Equation~\ref{eqn:eval2}. And thus, using the
above two lemmas the evaluation of a symmetric polynomial at roots of
a monic polynomial in a larger
field $L$ still remains in the ground field. If $f\in F[x]$ be a monic
polynomial with roots $\alpha_1,\alpha_2,\ldots,\alpha_n\in L$, then
given any symmetric polynomial $p(x_1,x_2,\ldots,x_n)$ with coefficients
in $F$, we have
\[
p(\alpha_1,\alpha_2,\ldots,\alpha_n) \in F
\]
\label{lem:ring-homomorphism2}
\begin{proof}
Using Lemma~\ref{lem:ring-homomorphism} and the above Equations~\ref{eqn:eval}
and Equation~\ref{eqn:eval2}.
\end{proof}

Let $G$ denote the reduced Gr\"obner basis of the collection of
elementary symmetric polynomials of $n$ variables, and $h$ denote
the remainder after Gr\"obner reduction of $f$ by $G$, then
using Theorem 2.4.4 and the Lemmas above, we have $f$ is symmetric iff
$h=0$, identically.
\end{proof}


\section*{Problem 2} Generalize our method to determine whether graphs are
3-colorable to determine whether graphs are 4-colorable.
\begin{proof}
The general idea is same as 3-coloring, with coloring assignments
being points in varieties; varieties which are represented by ideals;
and ideals can be computed with Gr\"obner basis.
We know that a graph $G$ is 4-colorable, if the variety $V(I)\ne 0$.
Let $x_i$ denote the color assignment for the vertex $i$, then we have
$x_i^4=x_j^4=1$, since the 4-colors are represented by 4th roots of unity.
This gives us
\begin{equation}
x_i^4-x_j^4 = (x_i-x_j)(x_i+x_j)(x_i^2+x_j^2) = 0 \label{eqn:graph-color}
\end{equation}
Since the ring is an integral domain, we have that $x_i\ne x_j$ (as they
are adjacent in the graph), then we must have $(x_i+x_j)(x_i^2+x_j^2)=0$.
Note: that $x_i \in \CC$, and in particular $x_i = e^{i\pi/2}$.
Therefore, given graph $G$, we add $x_i^4=1$, to the collection of
polynomials (which is an ideal)
for each vertex, and for each edge, we add a polynomial
generated by Eqn~\ref{eqn:graph-color}. Thereafter, we compute the
reduced Gr\"obner basis for this ideal, and using Theorem 2.7.1, we have
that $G$ is $k$-colorable (with $k=4$, in this case), iff $V(I)\ne 0$, which
the Gr\"obner basis is not identically $\{1\}$.
\end{proof}

\section*{Problem 3} Use our Gr\"obner basis to solve the following
Shidoku puzzle.
\begin{proof}
The problem size is 16, thus we have 16 variables, 
we label them using two dimensional coordinates $x_{ij},1\le i,j \le 4$.
We solved this using \textsc{Singular}, the code is uploaded.
\begin{verbatim}
I[1]=x(44)-1
I[2]=3*x(43)-x(44)^2+12*x(44)-23
I[3]=x(42)+x(43)+x(44)-8
I[4]=x(41)+x(42)+x(43)+x(44)-10
I[5]=x(34)-2
I[6]=x(33)+x(34)+x(43)+x(44)-10
I[7]=x(32)-4
I[8]=x(31)+x(32)+x(41)+x(42)-10
I[9]=x(24)+x(34)+x(44)-7
I[10]=x(23)-1
I[11]=x(22)+x(23)+x(24)-7
I[12]=x(21)+x(22)+x(23)+x(24)-10
I[13]=x(14)+x(24)+x(34)+x(44)-10
I[14]=x(13)+x(14)+x(23)+x(24)-10
I[15]=x(12)+x(22)+x(32)+x(42)-10
I[16]=x(11)+x(12)+x(21)+x(22)-10
\end{verbatim}
The Gr\"obner basis computed by \textsc{Singular} is shown, and using this
the solution is as follows:

\begin{centering}
\begin{tabular}{|c|c|c|c|}
\hline
4 & 1 & 2 & 3\\
\hline
3 & 2 & 1 & 4\\
\hline
1 & 4 & 3 & 2\\
\hline
2 & 3 & 4 & 1\\
\hline
\end{tabular}
\end{centering}
\end{proof}


\section*{Problem 4}
Use the method of Lemma 2.8.1 (in book) to solve the followin system of
equations for non-negative integers.
\begin{eqnarray}
3\sigma_1 + 2\sigma_2 + \sigma_3 & = & 10 \\ \nonumber
4\sigma_1 + 3\sigma_2 + \sigma_3 & = & 12
\end{eqnarray}
Let us introduce two variables $x_1,x_2$ for the two equations and
three variables $y_1,y_2,y_3$ for $\sigma_{\{1,2,3\}}$.
We can then write the above system of equation as
\begin{eqnarray}
\QQ[y_1,y_2,y_3,y_4] & \xrightarrow{\phi} & \QQ[x_1,x_2]  \nonumber \\
y_3 & \mapsto & x_1x_2 \nonumber \\
y_2 & \mapsto & x_1^2x_2^3 \nonumber \\
y_1 & \mapsto & x_1^3x_2^4 \nonumber
\end{eqnarray}
Let $K=\{y_1-x_1^3x_2^4,y_2-x_1^2x_2^3,y_3-x_1x_2\}$.
We compute the Gr\"obner basis for $K$ using \textsc{Singular} as
\begin{verbatim}
I[1]=y(1)-y(2)*y(3)
I[2]=x(2)*y(3)^2-y(2)
I[3]=x(1)*y(2)-y(3)^3
I[4]=x(1)*x(2)-y(3)
> reduce(x(1)^10*x(2)^12,I);
y(2)^2*y(3)^6
\end{verbatim}
Thereafter we reduce $x_1^{10}x_2^{12}$ with the Gr\"obner basis, and
obtain that the solution to the system of equation is $(0,2,6)$, such
that $\sigma_1=0,\sigma_2=2,\sigma_3=6$, and indeed, we verify that this
is a solution.
\end{document}

